\chapter{LATIN SQUARES}

\section{Introduction}

\begin{definition}
    A \eax{latin square} of order $n$ is an $n \times n$ array filled with $n$ distinct symbols, each occuring exactly once in each row and exactly once in each column.
\end{definition}

\eax{Euler's generals problem} in latin squares is as follows. Given $n^{2}$ military personnel, each with a rank and regiment of possible $n$ and $n$ options respectively, the problem is to arrange them in an $n \times n$ array such that any choice of a row or a column gives a group of $n$ personnel with $n$ distinct ranks. If we also require that any choice of a row or a column gives a group of $n$ personnel with $n$ distinct regiments, then it is \textit{not} possible to do so for $n = 6$.

\begin{definition}
    Two latin squares $L_{1},L_{2}$ of order $n$, on the same set of row and column indices, are said to be \eax{orthogonal latin squares} if the $n^{2}$ ordered pairs $(l_{ij}^{(1)},l_{ij}^{(2)})$ for $1 \leq i,j \leq n$ are all distinct.
\end{definition}

Euler's generals problem then becomes to find two orthogonal latin squares of order $6$.

\begin{theorem}
    For integer $n \geq 1$, the maximum number $N(n)$ of pairwise mutually orthogonal latin squares of order $n$ satisfies $N(n) \leq n-1$.
\end{theorem}

\begin{proof}
    The key is to convert this problem into a linear algebra one; Let $L_{1},\ldots,L_{t}$ be mutually orthogonal latin squares of order $n$. Let $S$ be their common symbol set. For $s \in S$ and $1 \leq i \leq t$, define $L_{i,s}$ to be the indicator matrix of the symbol $s$ in the latin square $L_{i}$. Then $L_{i,s}$ is a permutation matrix. If $J$ denotes the matrix will all entries $1$, then
    \begin{align}
        A_{i,s} = L_{i,s} - \frac{1}{n} J
    \end{align}
    has row and column sums $0$. As a small lemma, if $V$ denotes the set of matrices with row sum and column sum $0$, then $\dim(V) = (n-1)^{2}$. This is because we can choose the first $n-1$ rows and $n-1$ columns freely, and the last row and column are determined by the previous ones. Consider the Frobenius inner product on $M_{n}(\R)$ defined by $\ip{A,B} = \tr(A^{T}B)$. Note that $\ip{A_{i,s},J} = 0$. Taking two matrices $A_{i,s}$ and $A_{j,s'}$, we have
    \begin{align}
        \ip{A_{i,s},A_{j,s'}} = \ip{L_{i,s} - \frac{1}{n}J, L_{j,s'} - \frac{1}{n}J} = \ip{L_{i,s},L_{j,s'}} - \frac{1}{n}\ip{J,L_{j,s'}} = 1-1 = 0 \text{ for } i \leq j,\; s,s' \in S.
    \end{align}
    Moreover, $\sum_{s \in S} A_{i,s} = 0$ for each $i$. Thus $A_{i,s} \in V$. For a fixed $i$, we get
    \begin{align}
        \dim (\spanof_{s \in S} \{A_{i,s}\}) = n-1
    \end{align}
    by comparing inner products for both $i = j$ and $i \neq j$. So for $1 \leq i \leq t$, the $\spanof_{s \in S} \{A_{i,s}\}$ form mutually orthogonal subspaces of $V$. Therefore,
    \begin{align}
        \sum_{i=1}^{t} \dim(\spanof_{s \in S} \{A_{i,s}\}) \leq \dim V = (n-1)^{2} \implies t(n-1) \leq (n-1)^{2}.
    \end{align}
    We then have $\ip{\sum_{s \in S} \alpha_{s} A_{i,s}, \sum_{s \in S} \alpha_{s} A_{i,s'}} = n\alpha_{s'} - \sum_{s \in S} \alpha_{s} = 0$ which forces $\alpha_{s} = \alpha_{s'}$ for all $s,s' \in S$.
\end{proof}

\begin{theorem}
    If $q$ is a prime power then there exists a projective plane of order $q$.
\end{theorem}

\subsection{Projective Planes and Fields}

We move our discussion on to projective planes.

\begin{definition}
    A \eax{finite projective plane} of order $n$ is an incidence structure of points and lines satisfying
    \begin{enumerate}
        \item any two distinct points lie on exactly one line,
        \item any two distinct lines meet at exactly one points,
        \item there exist four points with no three collinear.
    \end{enumerate}
\end{definition}

\begin{lemma}
    Each line in a finite projective plane contains the same number of points $(n+1)$. Each point is on $(n+1)$ lines. The total number of points is equal to the total number of lines $n^{2}+n+1$.
\end{lemma}
\begin{proof}
    There is a one-to-one correspondence between lines passing through $p_{2}$ and points on $l$ other than $p_{1}$. The number of lines passing through $p_{2}$ equals the number of lines passing through $p_{3}$. There is a one-to-one correspondence between points on $l_{1}$, lines passing through $p$, and points on $l_{2}$. The number of points on $l_{1}$ equals the number of points on $l_{2}$, or $n+1$. The total number of points is $(n+1)^{2}-n = (n+1)n + 1$.
\end{proof}

\begin{theorem}
    There exists a projective plane of order $n$ if and only if there exists a set of $n-1$ mutually orthogonal latin squares of order $n$.
\end{theorem}

\begin{proof}
    Let $P$ be a projective plane of order $n$. Choose a line $L_{\infty}$ and label its $n+1$ points $p_{0},p_{1},\ldots,p_{n}$. There are $n$ lines passing through $p_{0}$ and $p_{1}$ apart from $L_{\infty}$. $P \setminus L_{\infty}$ has $n^{2}$ points. Let $R_{1},\ldots,R_{n}$ be the $n$ lines passing through $p_{0}$ apart from $L_{\infty}$, and let $C_{1},\ldots,C_{n}$ be the $n$ lines passing through $p_{1}$ apart from $L_{\infty}$. Note that $R_{i} \cap C_{j}$ is always a single point. The $n$ lines $l_{1},\ldots,l_{n}$ passing through $p_{2}$ apart from $L_{\infty}$ determine the symbol set $\{s_{1},\ldots,s_{n}\}$. Every point in $l_{1} \setminus \{p_{2}\} = p_{i} \cap c_{i}$ for a unique pair $(i,j)$. So put $r_{1}$ in those $(i,j)$. All points in $R_{1}$ are exhaustively labelled by $\{s_{1}^{(2)},\ldots,s_{n}^{(2)}\}$. We do the same for the $n$ lines passing through $p_{3},\ldots,p_{n}$. This gives $\{s_{1}^{(2)},\ldots,s_{n}^{(2)}\} = \{s_{1}^{(3)},\ldots,s_{n}^{(3)}\}$.
\end{proof}


Let $\K$ be a finite field, with $\Z \hookrightarrow \K$. Note that since this field it finite, for all $a \in \K$ there exists an integer $k$ such that $a = ka$ or $(k-1)a = 0$. The \eax{characteristic} of $\K$ is the smallest $n \in \N$ such that $na = 0$ for all $a \in \K$. The characteristic is always a prime.

If $na = 0$ for $a \neq 0$, then $na a^{-1} = (a + a + \cdots + a) a^{-1} = (1 + 1 + \cdots + 1) = 0$. If we let $n = pq$, then $p(qa) = 0$ for all $a \in \K$ implies either $qa = 0$ for all $a \in \K$ or $p 1 = 0$. In either case, a factor of $n$ is a `smaller' characteristic. This process only ends if $n$ is prime.

\begin{proposition}
    The order of a finite field is a prime power.
\end{proposition}
\begin{proof}
    Let $p$ be the characteristic of $\K$ and $q$ be another prime factor of $\# \K$. Then $qa = 0$ for some $a \neq 0$. By Cauchy's theorem, $pa = 0$ for some $a \neq 0$. So $p = q$. Thus the order of $\K$ is $p^{k}$ for some $k \in \N$.
\end{proof}
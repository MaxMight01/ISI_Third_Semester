\chapter{DISCRETE STRUCTURES}

\section{A Brief Introduction}
\textit{July 22nd.}

Discrete mathematics is primarily the study of tools for reasoning precisely the systematically about digital systems, logical problems, and combinatorial structures such as the integers, graphs, logical statements, and finite automata. Furthermore, combinatorics is the mathematics of counting and configuration; the counting, organizing, and analyzing discrete structures. 

\eax{Bloch's principle}, or Bloch's heuristics, states that every proposition on whose statement the actual infinity occurs can always be considered as a consequence of a proposition where it does not occur as a proposition on finite terms. The \eax{Ramsey principle} states that complete disorder is impossible. In any sufficiently large structure, order or regularity must emerge. These two principles may be considered complimentary to each other.

\section{Useful Methods}
The method of \eax{double counting} can be thought of a creative device or trick. Before strictly showing the statement, we utilise some examples.
\begin{example}
    Suppose we wish to show that $\sum_{k=0}^{n} \binom{n}{k} = 2^{n}$. We first ask how many ways can a subset be chosen from $\{1,2,\ldots,n\}$. The first method is to build a subset by deciding whether we want $i$ to be a part of our subset for $i \in \{1,2,...,n\}$. The second method is find the number of subsets of caridnality $i$ for $i \in \{1,2,\ldots,n\}$ and add up all the results. This leads us to conclude $2^{n} = \sum_{k=0}^{n} \binom{n}{k}$ after equating the answers from both methods.
\end{example}

\begin{theorem}[The \eax{q-binomial theorem}]
    We use the following notation:
    \begin{equation}
        \binom{n}{k}_{q} = \frac{(q^{n}-1) \cdots (q^{n-k+1}-1)}{(q^{k}-1) \cdots (q-1)}.
    \end{equation}
    Simply stated, the $q$-binomial theorem is
    \begin{equation}
        \sum_{k=0}^{n} q^{\binom{k}{2}} \binom{n}{k}_{q} z^{k} = \prod_{i=0}^{n-1} (1+q^{i}z).
    \end{equation}
\end{theorem}
The proof of the above theorem is performed by double counting; counting the number of pairs $(U,B)$ where $U$ is a $k$-dimensional subspace of $\F^{n}_{q}$ and $B$ is the flag of nested subspaces of $U$.

\subsubsection{Reccurrence Relations and Generating Functions}
Perhaps, the most important example of a recurrence relation is the Fibonacci sequence, where the terms in the sequence are defined as $F_{0} = 0$, $F_{1} = 1$, $F_{n+2} = F_{n+1} + F_{n}$ for all $n \geq 0$. Let us find the \eax{generating function} of this sequence; we start by creating
\begin{equation}
    F(t) = F_{0} + F_{1}t + F_{2}t^{2} + F_{3}t^{3} + \cdots,
\end{equation}
the generating function of $(F_{n})_{n=0}^{\infty}$. We can then work as follows---
\begin{align}
    tF(t) &= F_{0}t + F_{1}t^{2} + \cdots,\notag\\
    t^{2}F(t) &= F_{0}t^{2} + \cdots,\notag\\
    \implies (1-t-t^{2})F(t) &= t \implies F(t) = \frac{-t}{t^{2}+t-1}.
\end{align}

If we look at $F_{n+1} = 1 \cdot F_{n} + 1 \cdot F_{n-1}$ and $F_{n} = 1 \cdot F_{n} + 0 \cdot F_{n-1}$, we may notice a matrix as $\begin{bmatrix}
    F_{n+1} \\ F_{n}
\end{bmatrix} = \begin{bmatrix}
    1 & 1 \\ 1 & 0
\end{bmatrix} \begin{bmatrix}
    F_{n} \\ F_{n-1}
\end{bmatrix}$. Back substituting multiple times leads us to conclude $\begin{bmatrix}
    F_{n+1} \\ F_{n}
\end{bmatrix} = \begin{bmatrix}
    1 & 1 \\ 1 & 0
\end{bmatrix}^{n} \begin{bmatrix}
    1 \\ 0
\end{bmatrix}$. We can then diagonalize the centre matrix to decompose it as $P \begin{bmatrix}
    (1+\sqrt{5})^{n}/2^{n} & 0 \\ 0 & (1-\sqrt{5})^{n}/2^{n}
\end{bmatrix} P^{-1}$; thus, the terms of the sequence are really linear combinations of the diagonal elements that appear.

\subsubsection{Principle of Inclusion-Exclusion}
\textit{July 24th.}

Simply stated, for sets $A$ and $B$, $\# (A \cup B) = \# A + \# B + \# (A \cap B)$. For three sets $A, B$ and $C$, we have
$\# (A \cup B \cup C) = \# A + \# B + \# C - \# (A \cap B) - \# (B \cap C) - \# (A \cap C) + \# (A \cap B \cap C)$. This can be extended to any finite number of finite sets.

\begin{theorem}[The \eax{principle of inclusion-exclusion}]
    Let $S$ be an $N$-set ($\#S = N$), and let $E_{1},E_{2},\ldots,E_{r}$ be, not necessarily distinct, subsets of $S$. For any subset $M$ of the indexing set $\{1,2,\ldots,r\}$, let $N(M)$ denote the number of elements of $S$ in $\bigcap_{i \in M} E_{i}$, and for $0 \leq j \leq r$, define
    \begin{align}
        N_{j} = \sum_{\# M = j} N(M).
    \end{align}
    Then the number of elements of $S$ not in any of the $E_{i}$'s is
    \begin{align}
        \#(S \setminus \bigcup_{i = 1}^{r} E_{i}) = N - N_{1} + N_{2} - N_{3} + \cdots + (-1)^{r} N_{r}.
    \end{align}
\end{theorem}
\begin{proof}
    For $x \in S$, define $M:S \to \{0,1\}$ as $M(x) = 1$ if $x \in \bigcap_{i \in M} E_{i}$ and $0$ otherwise. Thus,
    \begin{align}
        \sum_{x \in S} M(x) = \# (\bigcap_{i \in M} E_{i}) = N(M) \implies N_{j} = \sum_{\# M = j} \sum_{x \in S} M(x) = \sum_{x \in S} \sum_{\# M = j} M(x).
    \end{align}
    The alternating sum then becomes
    \begin{align}
        \sum_{x \in S} 1 - \sum_{x \in S} \sum_{\# M = 1} M(x) + \cdots + (-1)^{r} \sum_{x \in S} \sum_{\# M = r} M(x) = \sum_{x \in S} \left( 1 - \sum_{\# M = 1} M(x) + \cdots + (-1)^{r} \sum_{\# M = r} M(x) \right).
    \end{align}
    Call the term within the parentheses as $F(x)$. We deal with cases; if $x \notin \bigcup_{i=1}^{r} E_{i}$, then $F(x) = 1$. If $x$ is in exactly $k \geq 1$ of the sets $E_{1},E_{2},\ldots,E_{r}$, then
    \begin{align}
        F(x) = 1 - \binom{k}{1} + \binom{k}{2} - \binom{k}{3} + \cdots + (-1)^{k} \binom{k}{k} = (1-1)^{k} = 0.
    \end{align}
    This is independent of $k$; we conclude that the alternating sum reduces to the number of elements in $S$ not in any of the $E_{i}$'s.
\end{proof}

\begin{corollary}
    Retaining notation from the previous theorem, if $S = \bigcup_{i=1}^{r} E_{r}$, then
    \begin{align}
        N = N_{1} - N_{2} + \cdots + (-1)^{r-1} N_{r}.
    \end{align}
\end{corollary}

We look at some examples of the principle in use.
\begin{example}
    Let $d_{n}$ be the number of permutations $\pi$ of the set $\{1,2,\ldots,n\}$ such that $\pi(i) \neq i$ for all $1 \leq i \leq n$. Such a permutation is called a \eax{derangement}, where no point is fixed. We wish to count all such permutations. Let the set of all permutations of the set be $S$. Let $E_{i}$ denote the set of all permutations that fix $i$, for $1 \leq i \leq n$. Thus, $S$ without $\bigcup_{i=1}^{n} E_{i}$ would then denote the set of all derangements. Making use of the principle, we have
    \begin{align}
        \#(S \setminus \bigcup_{i=1}^{n} E_{i}) = n! - \binom{n}{1}(n-1)! + \binom{n}{2} (n-2)! - \cdots = n! \left( 1-\frac{1}{1!} + \frac{1}{2!} - \frac{1}{3!} + \cdots + (-1)^{n} \frac{1}{n!} \right)
    \end{align}
    which is approximately $\frac{n!}{e}$ for larger $n$. Thus, the probability of choosing a derangemnet is $e^{-1}$.
\end{example}

\begin{example}
    Suppose we have two sets $X$ and $Y$ with $\# X = n$ and $\# Y = k$. We ask how many surjective maps exists from $X$ to $Y$. The set $S$, this time, is the set of all functions from $X$ to $Y$, being $Y^{X}$. $E_{i}$ denotes the set of functions from $X$ to $Y$ such that $y_{i}$ is not in the image of $X$. The elments within $S$ not in any of the $E_{i}$'s are surjective maps. Clearly, $N_{i} = \binom{k}{i} (k-i)^{n}$, and the cardinality of $S \setminus \bigcup_{i=1}^{k} E_{i}$ is then
    \begin{align}
        \sum_{i=0}^{k} (-1)^{i} \binom{k}{i} (k-i)^{n}.
    \end{align}
\end{example}

\begin{example}
    We wish to show that the expression $\sum_{i=0}^{n} (-1)^{i} \binom{n}{i} \binom{m+n-i}{k-i}$ evaluates to $\binom{m}{k}$ if $m \geq k$, and $0$ otherwise. To this end, fix $X = \{x_{1},\ldots,x_{n}\}$ and $Y = \{y_{1},\ldots,y_{m}\}$ and set $Z = X \cup Y$. We now ask how many $k$-subsets of $Z$ consist of only points form $Y$. Let $S$ be the set of all $k$-subsets of $Z$, and denote $E_{i}$ to be the set of $k$-subsets of $Z$ containing $x_{i}$ for $1 \leq i \leq n$. The left hand side of our inclusion-exclusion principle evaluates to $\binom{m+n}{k}$. Each $N_{i}$ evaluates to $\binom{n}{i} \binom{m+n-i}{k-i}$, proving our expression above.
\end{example}

The next example relates to the Euler totient function.

\begin{example}
    Recall that, from the \eax{fundamental theorem of arithmetic}, each natural number may be expressed uniquely (upto order) as the product of distinct primes raised to values, that is, $n = p_{1}^{a_{1}} \cdots p_{r}^{a_{r}}$. The \eax{Euler totient function} $\phi: \N \to \C$ acts on the naturals and returns $\phi(n)$, the number of positive integers $k \leq n$ such that $\gcd (k,n) = 1$. Certainly, $\phi(p) = p-1$ for a prime $p$. Our task is to find a closed form formula for $\phi(n)$.

    Set $S = \{1,2,\ldots,n\}$, and set $E_{i}$ to be the set of integers in $S$ divisible by $p_{i}$ for $1 \leq i \leq r$. Clearly, the value $\# (S \setminus \bigcup_{i=1}^{r} E_{i})$ returns the set of all numbers in $\{1,2,\ldots,n\}$ coprime to $n$. Note that $N = n$. The value of $N_{1}$ is $\sum_{i=1}^{r} \frac{n}{p_{i}}$ , the value of $N_{2}$ is is $\sum_{1 \leq i \leq j \leq r} \frac{n}{p_{i}p_{j}}$. The cloesd form formula then becomes
    \begin{align}
        \phi(n) = \# (S \setminus \bigcup_{i=1}^{r} E_{i}) = n - n \sum_{i=1}^{r} \frac{1}{p_{i}} + n \sum_{1 \leq i \leq j \leq r} \frac{1}{p_{i}p_{j}} + \cdots = n\left( 1-\frac{1}{p_{1}} \right)\left( 1-\frac{1}{p_{2}} \right) \cdots \left( 1-\frac{1}{p_{r}} \right).
    \end{align}
\end{example}


\subsubsection{Number Theory}
We continue with the Euler totient function.
\begin{theorem}
    $\sum \limits_{d \mid n} \phi(d) = n$.
\end{theorem}
\begin{proof}
    For each integer $m \in \{1, 2, \ldots, n\}$, the value $\gcd(m, n)$ is a divides $n$. Fix a divisor $d$ of $n$. The number of integers $m$ such that $\gcd(m, n) = d$ is equal to the number of integers $m$ such that $\gcd\left(\frac{m}{d}, \frac{n}{d}\right) = 1$, where $\frac{m}{d}$ runs over integers between $1$ and $\frac{n}{d}$.
    Therefore, the number of such $m$ is $\phi\left(\frac{n}{d}\right)$. Summing over all divisors $d$ of $n$, we get:
    \[
        n = \sum_{d \mid n} \phi\left(\frac{n}{d}\right)
    \]
    which is the same as $\sum_{d \mid n} \phi(d)$.
\end{proof}

The \eax{M\"obius function} is defined as
\begin{equation}
    \mu(d) \defeq \begin{cases}
        1, &\text{ if } d \text{ is a product of even number of distinct primes},\\
        -1, &\text{ if } d \text{ is a product of odd number of distinct primes},\\
        0, &\text{ if otherwise; the number } d \text{ is not square-free.}
    \end{cases}
\end{equation}

\begin{theorem}
    \begin{align}
        \sum_{d \mid n} \mu(d) = \begin{cases}
            1, &\text{ if } n = 1,\\
            0, &\text{ if otherwise.}
        \end{cases}
    \end{align}
\end{theorem}

\begin{proof}
    For $n = 1$, it is clear. For $n > 1$, rewriting $n$ as $p_{1}^{a_{1}} \cdots p_{r}^{a_{r}}$ helps us see that
    \begin{align}
        \sum_{d \mid n} \mu(d) = 1 - \binom{r}{1} + \binom{r}{2} - \binom{r}{3} + \cdots = (1-1)^{r} = 0.
    \end{align}
\end{proof}

This property of the M\"obius function proves to be useful.\\ \\
\textit{July 29th.}

\begin{theorem}[The \eax{M\"obius inversion formula}]
    Suppose we have two function $f: \N \to \R$ and $g:\N \to \R$ which relate as
    \begin{align}
        f(n) = \sum_{d \mid n} g(d).
    \end{align}
    Then the function $g$ satisfies
    \begin{align}
        g(n) = \sum_{d \mid n} \mu\left( \frac{n}{d} \right)f(d).
    \end{align}
\end{theorem}

\begin{proof}
    We work as
    \begin{align}
        \sum_{d \mid n} \mu \left( \frac{n}{d} \right) f(d) &= \sum_{d \mid n} \mu\left(  \frac{n}{d} \right) \left( \sum_{d'\mid d} g(d') \right) = \sum_{d \mid n,\; d'\mid d} \mu \left( \frac{n}{d} \right) g(d') = \sum_{d' \mid n,\; m \mid \frac{n}{d'}} g(d') \mu(m) \notag \\
        &= \sum_{d' \mid n} \left( g(d') \left( \sum_{m \mid \frac{n}{d'}} \mu(m) \right) \right) = g(n).
    \end{align}
\end{proof}

\begin{example}
Let $N_n$ denote the number of distinct circular binary sequences of length $n$, up to rotation. That is, two sequences are considered the same if one is a rotation of the other. We aim to compute $N_n$ explicitly.

Let $M(d)$ denote the number of aperiodic circular binary sequences of length $d$, meaning sequences that are not periodic with any smaller period. Note that each such aperiodic sequence of length $d$ contributes to sequences of length $n$ whenever $d \mid n$. Indeed, every binary circular sequence of length $n$ can be viewed as made up of $\frac{n}{d}$ repetitions of a primitive block of length $d$.

Thus, we have:
\[
N_n = \sum_{d \mid n} M(d).
\]

Now consider the total number of binary strings of length $n$, which is $2^n$. Each such string can be arranged in a circle in $n$ different ways, one for each rotation. However, many of these circular sequences are identical under rotation, so we overcounted by a factor of the size of the symmetry group.

Let $f(n) = 2^n$ be the total number of binary strings of length $n$. Each such string is generated by repeating an aperiodic sequence of length $d$ exactly $\frac{n}{d}$ times, for some $d \mid n$. Since each aperiodic circular sequence of length $d$ has $d$ rotations, we get:
\[
f(n) = 2^n = \sum_{d \mid n} d \cdot M(d).
\]

Applying M\"obius inversion to this relation, we obtain:
\[
n \cdot M(n) = \sum_{d \mid n} \mu(d) \cdot 2^{n/d},
\]
and hence,
\[
M(n) = \frac{1}{n} \sum_{d \mid n} \mu(d) \cdot 2^{n/d}.
\]

Substituting back into the formula for $N_n$, we get:
\[
N_n = \sum_{d \mid n} M(d) = \sum_{d \mid n} \frac{1}{d} \sum_{k \mid d} \mu(k) \cdot 2^{d/k}.
\]
Interchanging the order of summation, we arrive at the classical formula:
\[
N_n = \frac{1}{n} \sum_{d \mid n} \phi(d) \cdot 2^{n/d},
\]
where $\phi$ is Euler's totient function.
\end{example}


\begin{lemma}[\eax{Burnside's lemma}]
    Let $G$ be a permutation group acting on some finite set $X$. Let $\psi(g)$ denote the number of points of $X$ fixed by $g \in G$. Then the number of orbits of $G$ is $\frac{1}{\abs{G}} \sum_{g \in G} \psi(g)$.
\end{lemma}
In the above, by the set of points fixed by $g \in G$, we mean the set $\{x \mid g \cdot x = x\}$. By the \eax{orbit} of $x$, we mean the set $\{g \cdot x \mid g \in G\}$.

Such inversion formulae are common in discrete math. The following are some examples.
\begin{example}
    \begin{itemize}
        \item For an integer $n$, $f(n) = \sum_{i=1}^{n} g(i)$ if and only if $g(n) = f(n) - f(n-1)$. This is known as a \eax{telescoping sum}.
        \item For an integer $n$, $f(n) = \sum_{d \mid n} g(d)$ if and only if $g(n) = \sum_{d \mid n} \mu \left( \frac{n}{d} \right) f(d)$. This is the M\"obius inversion formula.
        \item For a set $S$, $f(S) = \sum_{T \subseteq S} g(T)$ if and only if $g(S) = \sum_{T \subseteq S} (-1)^{\abs{S}-\abs{T}} f(T)$.
    \end{itemize}
\end{example}

\subsubsection{Partially Ordered Sets}
\begin{definition}
    A \eax{poset} $S$, or a \eax{paritally ordered set}, is a (countable or finite) set of objects with a binary relation $\leq$ satisfying the following properties.
    \begin{enumerate}
        \item Reflexivity: $x \leq x$ for all $x \in S$.
        \item Antisymmetry: if $x \leq y$ and $y \leq x$, for some $x,y \in S$, then $x = y$.
        \item Transitivity: if $x \leq y$ and $y \leq z$, for some $x,y,z \in S$, then $x \leq z$.
    \end{enumerate}
\end{definition}

Some examples are as follows.

\begin{example}
    \begin{enumerate}
        \item $(\{1,2,\ldots,n\},\leq)$ is a poset, with $a \leq b$ if $b - a$ is a non-negative integer.
        \item $(\{1,2,\ldots,n\},\leq_{1})$ is also a poset, where $a \leq_{1} b$ if $a \mid b$.
        \item $(\cP(\{1,2,\ldots,n\}), \leq_{2})$ is also a poset; here, $S \leq_{2} T$ if $S \subseteq T$. The power set is also denoted as $2^{\{1,2,\ldots,n\}}$.
        \item The set of partitions of $\{1,2,\ldots,n\}$, equippied with the partial ordering of refinement. By a partition, we mean $\{S_{1},S_{2},\ldots,S_{r}\}$ such that $S_{i} \cap S_{j} = \emptyset$ for $i \neq j$, and $\bigcup_{i = 1}^{r} S_{i} = \{1,2,\ldots,n\}$. Similarly, let $\{T_{1},T_{2},\ldots,T_{d}\}$ be another partition. Then $\{S_{1},\ldots,S_{r}\} \leq \{T_{1},\ldots,T_{d}\}$ if for $1 \leq i \leq r$, $S_{i} \subseteq T_{k}$ for some $1 \leq k \leq d$. Here, we term $\{S_{1},\ldots,S_{r}\}$ a \eax{refinement} of $\{T_{1},\ldots,T_{d}\}$.
    \end{enumerate}
\end{example}

The idea of the M\"obius function really comes from posets, where its definition is more generalized. Here, $\mu(d,n)$ can be thought of as a place-in for $\mu \left( \frac{n}{d} \right)$.

\begin{definition}
    The \eax{M\"obius function of a poset} is defined as
    \begin{align}
        \mu(x,y) = \begin{cases}
            0 &\text{ if } x \nleq y,\\
            1 &\text{ if } x = y,\\
            -\sum_{x \leq z < y} \mu(x,z) &\text{ if } x \lneq y.
        \end{cases}
    \end{align}
\end{definition}

Note that we need only compute $\mu(x,y)$ on all intervals $[x,y]$ $(x \leq y)$. We call an element $y$ a \eax{successor} of $x$ if there exists no $z$ satisfying $x \lneq z \lneq y$. Note that the successor may not be unique. Let us denote any successor by $\suc(x)$.

If $g$ and $f$ are two functions defined and related as
\begin{align}
    g(x) = \sum_{y \leq x} f(y).
\end{align}
We now introduce the \eax{zeta function} $\zeta$ defined as $\zeta(x,y) = 1$ if $x \leq y$ and $0$ otherwise. Thus, the above equation can be rewritten as
\begin{align}
    g(x) = \sum_{y \leq x} f(y) = \sum_{y \in S} \zeta(y,x) f(y). 
\end{align}

\noindent \textit{August 5th.}

\begin{lemma}
    A finite partial order can always be embedded in a total ordering; that is, there exists an indexing $S = \{x_{1},\ldots,x_{n}\}$ such that $x_{i} \leq x_{j}$ in $S$ implies $i \leq j$.
\end{lemma}

As a proof outline, pick a maximal element $x$ of $S$. Label it $x_{n}$. Repeat the process with $S\setminus\{x\}$, then proceed inductively. The embedding is then clear.

Thus, using the lemma, we can rewrite the relation between $g$ and $f$ in matrix form as
\begin{align}
    \begin{pmatrix}
        g(x_{1}) & \cdots & g(x_{n})
    \end{pmatrix} = \begin{pmatrix}
        f(x_{1}) & \cdots & f(x_{n})
    \end{pmatrix} \begin{pmatrix}
        \zeta(x_{1},x_{1}) & \cdots & \zeta(x_{1},x_{n}) \\
        \vdots & \ddots & \vdots \\
        \zeta(x_{n},x_{1}) & \cdots & \zeta(x_{n},x_{n})
    \end{pmatrix}.
\end{align}
Since $\zeta(x_{i},x_{j}) = 0$ when $i > j$, the matrix on the right is upper triangular. Also, all the diagonal entries are 1. Denote the above matrix on the right by $Z$. Note that $Z = I+N$ where $I$ is the identity matrix and $N$ is upper triangular with $0$'s on the diagonal. $Z^{-1}$ can be computed by taking a power series, and noting that $N^{n}$ is $0$.
\begin{align}
    Z^{-1} = (I+N)^{-1} = I-N+N^{2}-N^{3} + \cdots + (-1)^{n-1}N^{n-1}
\end{align}

Let $M = [\mu(x_{i},x_{j})]$. We find the $(x_{i},x_{j})$ entry of $MZ$ as
\begin{align}
    \sum_{y \in P} M_{x_{i},y} Z_{y,x_{j}} = \sum_{y \in P} \mu(x_{i},y) \zeta(y,x_{j}) = \sum_{y \leq x_{j}} \mu(x_{i},y) = \sum_{x_{i} \leq y \leq x_{j}} \mu(x_{i},y).
\end{align}
Noting that $\mu(x_{i},x_{j}) = -\sum_{x_{i} \leq z < x_{j}} \mu(x_{i},z)$, we get the above expression to be $1$ if $x_{i} = x_{j}$ and $0$ otherwise. Thus, $MZ = I$. This is the used definition of the the M\"obius functon. $ZM = I$ may be verified similarly.


\begin{theorem}
    Let $(P,\leq)$ be a finite poset, with $f,g:P \to \Z$ functions. Then,
    \begin{enumerate}
        \item $f(x) = \sum_{y \leq x} g(y)$ if and only if $g(x) = \sum_{y \leq x} \mu(y,x) f(y)$, and
        \item $f(x) = \sum_{x \leq y} g(y)$ if and only if $g(x) = \sum_{x \leq y} \mu(x,y)f(y)$. 
    \end{enumerate}
    This is the \eax{M\"obius inversion formula for a poset}.
\end{theorem}

\begin{proof}
    We have
    \begin{align}
        \sum_{y \leq x} \mu(y,x)f(y) = \sum_{y \leq x} \left( \sum_{z \leq y} \mu(y,x) g(z) \right) = \sum_{z \leq x} \sum_{z \leq y \leq x} \mu(y,x)g(z) = \sum_{z \leq x} g(z) \sum_{z \leq y \leq x} \mu(y,x) = g(x)
    \end{align}
    since $\mu(y,x)$ at the end will be zero if $z \neq x$. To show the converse, we have
    \begin{align}
        \sum_{y \leq x} g(y) = \sum_{y \leq x} \sum_{z \leq y} \mu(z,y) f(z) = \sum_{z \leq x} f(z) \left( \sum_{z \leq y \leq x} \mu(z,y) \right) = f(x)
    \end{align}
    since $\mu(z,y)$ is zero if $z \neq x$.
\end{proof}

\begin{example}
    Verify that if the poset is the positive integers with the standard ordering $\leq$, then
    \begin{align}
        \mu(i,j) = \begin{cases}
            1 &\text{ if } i = j,\\
            -1 &\text{ if } i = j-1,\\
            0 &\text{ if otherwise.}
        \end{cases}
    \end{align}
\end{example}

\begin{example}
    Let our poset be $(2^{S},\subseteq)$ for a set $S$. For fixed subsets $U,T \in 2^{S}$, we have
    \begin{align}
        \sum_{U \subseteq R \subseteq T}(-1)^{\#T - \#S} = \begin{cases}
            1 &\text{ if } U = T,\\
            0 &\text{ if otherwise.} 
        \end{cases}
    \end{align}
    To show this, without the loss of generality, we will assume that $U = \emptyset$. Denoting $\#T = n$, we have
    \begin{align}
        \sum_{R \subseteq T} (-1)^{\#R} = \sum_{k=0}^{n} \binom{n}{k}(-1)^{k} = 0.
    \end{align}
    Here, $Z(R,T) = 1$ if $R \subseteq T$ and 0 otherwise. Also, $M(R,T) = (-1)^{\#T-\#R}$ if $R \subseteq T$ and 0 otherwise. Since $MZ = I$, $M(R,T)$ must be the M\"obius function for $(2^{S},\leq)$.
\end{example}

\chapter{VARIATIONAL PRINCIPLES}

\section{Integral Principle}
\textit{August 11th.}

Recall the assumptions we made prior to deriving the equations of motion:
\begin{enumerate}
    \item $\vec{F}_{i}$ is decomposed into $\vec{F}_{i}^{(a)}$ and $\vec{f}_{i}$.
    \item We have $\sum_{i=1}^{N_{p}} \vec{f}_{i} \cdot \delta \vec{r}_{i} = 0$.
    \item Each $\vec{r}_{i}$ is expressed as $\vec{r}_{i}(q_{1},q_{2},\ldots,q_{n},t)$, where the $\delta q_{k}$'s are independent.
\end{enumerate}

From our assumptions, there are no longer any constraint forces. Also recall the generalized forces where
\begin{align}
    Q_{j} = \sum_{i=1}^{N_{p}} \vec{F}_{i}^{(a)} \cdot \frac{\partial \vec{r}_{i}}{\partial q_{j}}, \quad j = 1,2,\ldots,n,
\end{align}
satisfying
\begin{align}
    \frac{d}{dt} \left( \frac{\partial T}{\partial \dot{q}_{i}} \right) - \frac{\partial T}{\partial q_{i}} - Q_{i} = 0, \quad i = 1,2,\ldots,n,
\end{align}
where $T = \sum_{i=1}^{N_{p}} \frac{1}{2} m_{i} \abs{\dot{\vec{r}}_{i}}^{2}$ is the kinetic energy. If a $V$ can be defined such that $\vec{F}_{i} = -\vec{\nabla}_{i} V$, then $L(q,\dot{q},t) = T - V$, and
\begin{align}
    \frac{d}{dt} \left( \frac{\partial L}{\partial \dot{q}_{j}} \right) - \frac{\partial L}{\partial q_{j}} = 0, \quad j = 1,2,\ldots,n,
\end{align}
where $-\frac{\partial V}{\partial q_{j}} = Q_{j}$ and $V$ does not depend on $\dot{q}_{j}$.

We now discuss the \eax{integral principle}, also known as \eax{Hamilton's principle}. As a precursor, we have $Q_{j} = U(\{q_{j}\},\{\dot{q}_{j}\},t)$ where $U$ is a function of the generalized coordinates, velocities, and time, and $L(\{q_{j}\},\{\dot{q}_{j}\},t) = T-U$. The \eax{action} is defined as
\begin{align}
    S = \int_{t_{i}}^{t_{f}} L(\{q_{j}\},\{\dot{q}_{j}\},t)dt.
\end{align}
The principle states that the physical trajectory taken by the system between two times $t_{i}$ and $t_{f}$ is the one for which the action $S$ is stationary, usually an extremum.

\subsection{Variational Calculus}

We set up a function $f(y,\frac{dy}{dx},x)$ where we have the correspondences $f \leftrightarrow x$, $q \leftrightarrow y$, and $L \leftrightarrow f$, which is analogous to writing $L(q,\dot{q},t)$. We also identify $\frac{dy}{dx}$ with $\dot{y}$. Then, correspondingly, we have
\begin{align}
    J \equiv \int_{x_{1}}^{x_{2}} f(y,\dot{y},x)dx.
\end{align}
Setting $y(x_{1}) = y_{1}$ and $y(x_{2}) = y_{2}$, we aim to minimize $J$ with respect to $y$. We ask how does one \textit{vary} $y(x)$, a function? To rectify this issue, we introduce a new parameter $\alpha$ in $y(x,\alpha)$ such that $y(x,0) = y(x)$. In other words,
\begin{align}
    y(x,\alpha) = y(x,0) + \alpha \eta(x),
\end{align}
where $\eta(x)$ is an arbitrary function that vanishes at the endpoints: $\eta(x_{1}) = \eta(x_{2}) = 0$. It is safe to assume here that every possible smooth path from $x_{1}$ to $x_{2}$ can be represented in this form; the neighbouring paths are parametrized by $\alpha$. Thus,
\begin{align}
    J[f] = \int_{x_{1}}^{x_{2}}dx \cdot f(y(x,\alpha),\dot{y}(x,\alpha),x).
\end{align}
We can now differentiate $J$ with respect to $\alpha$ and set it to $0$ at $\alpha = 0$.
\begin{align}
    \frac{dJ}{d\alpha} = \int_{x_{1}}^{x_{2}} \left( \frac{\partial f}{\partial y} \frac{\partial y}{\partial \alpha} + \frac{\partial f}{\partial \dot{y}} \frac{\partial \dot{y}}{\partial \alpha} \right) dx.
\end{align}
Using the fact that $\frac{\partial \dot{y}}{\partial \alpha} = \frac{d}{dx} \left( \frac{\partial y}{\partial \alpha} \right) = \frac{\partial^{2} y}{\partial x \partial \alpha}$ and noticing the $dx$ term, the integral screams out to use integration by parts. Thus, on the rightmost term of the integral
\begin{align}
    \int_{x_{1}}^{x_{2}} dx \cdot \frac{\partial f}{\partial \dot{y}} \frac{d}{dx} \left( \frac{\partial y}{\partial \alpha}\right) = \left[-\int_{x_{1}}^{x_{2}} \frac{\partial y}{\partial x} \frac{\partial}{\partial x} \left( \frac{\partial f}{\partial \dot{y}} \right) dx \right] + \frac{\partial f}{\partial \dot{y}} \frac{\partial y}{\partial \alpha} \bigg|_{x_{1}}^{x_{2}}.
\end{align}
We plug this back in the derivative of $J$, taking note that $\frac{\partial y}{\partial \alpha} = \eta(x)$.
\begin{align}
    \frac{dJ}{d\alpha} = \int_{x_{1}}^{x_{2}} \left( \frac{\partial f}{\partial y} - \frac{d}{dx} \left( \frac{\partial f}{\partial \dot{y}} \right) \right) \left( \frac{\partial y}{\partial \alpha} \right) dx = \int_{x_{1}}^{x_{2}} dx\left( \frac{\partial f}{\partial y} - \frac{d}{dx} \left( \frac{\partial f}{\partial \dot{y}} \right) \right) \eta(x) = 0.
\end{align}
Since this works for \textit{any} $\eta$ following our initial constraints, we must have
\begin{align}
    \frac{d}{dx} \left( \frac{\partial f}{\partial \dot{y}} \right) - \frac{\partial f}{\partial y} = 0.
\end{align}

\begin{example}
    We show that, without any external conditions, the shortest path between two points $(x_{1},y_{1})$ and $(x_{2},y_{2})$ is a straight line. Each path segment is characterized as $(ds)^{2} = (dx)^{2} + (dy)^{2}$. Thus,
    \begin{align}
        J = \int_{x_{1}}^{x_{2}} ds = \int_{x_{1}}^{x_{2}}\sqrt{(dx)^{2}+(dy)^{2}} = \int_{x_{1}}^{x_{2}} dx \sqrt{1 + \left( \frac{dy}{dx} \right)^{2}} = \int_{x_{1}}^{x_{2}} dx (1+\dot{y}^{2})^{1/2}
    \end{align}
    Identifying with our equation, we have $f = \sqrt{1+\dot{y}^{2}}$, and $\frac{\partial f}{\partial \dot{y}} = \frac{\dot{y}}{\sqrt{1+\dot{y}^{2}}}$ and $\frac{\partial f}{\partial y} = 0$. Thus, the equation tell us
    \begin{align}
        \frac{d}{dx} \left( \frac{\dot{y}}{\sqrt{1+\dot{y}^{2}}} \right) = 0 \implies \frac{\dot{y}}{\sqrt{1+\dot{y}^{2}}} = C \implies \dot{y} = \sqrt{\frac{C^{2}}{1-C^{2}}}.
    \end{align}
\end{example}


\begin{example}
    As another example, we show that the shortest path between the north pole $(\theta_{1} = 0)$ and south pole $(\theta_{2} = \pi)$ on a sphere is a great circle. Here, the movement in spherical coordinates is $\vec{dt} = dr \hat{e}_{r} + rd\theta \hat{e}_{\theta} + r \sin \theta d\phi \hat{e}_{\phi}$. For a sphere, $R$ is constant so really we have $\vec{dr} = Rd \theta \hat{e}_{\theta} + R \sin \theta d\phi \hat{e}_{\phi}$. Thus, the integral becomes
    \begin{align}
        J = \int_{\theta_{1}}^{\theta_{2}} \sqrt{\vec{dr} \cdot \vec{dr}} = \int_{\theta_{1}}^{\theta_{2}} R \sqrt{1 + \sin^{2}\theta \dot{\phi}^{2}} d \theta.
    \end{align}
    Thus, we identify $x$ with $\theta$, $y(x)$ with $\phi(\theta)$, $f$ with $\sqrt{1+\sin^{2}\theta \dot{\phi}^{2}}$. Plugging the partials in the equation, we get
    \begin{align}
        \frac{R\sin^{2}\theta\dot{\phi}}{\sqrt{1+\sin^{2}\theta \dot{\phi}^{2}}} = C
    \end{align}
    This works for any $\theta$; plugging in $\theta_{1} = 0$ shows that $C = 0$. Thus, we must conclude that $\dot{\phi} \equiv 0$; the condition of a great circle. Notice that our choice of north pole and south pole could be arbitrary. Thus, the shortest distance between any two points on a sphere $\theta_{1} = 0$ and $\theta_{2}$ is a great circle.
\end{example}

\noindent \textit{August 18th.}

\begin{remark}
    The following may be shown via variational calculus.
    \begin{enumerate}
        \item A geodesic in $\R^{n}$ is a `straight' line.
        \item A geodesic on a sphere $S^{2}$ is a `great circle'.
        \item The minimum time path, with end points fixed, freely falling under $Mg$ is a `\eax{brachistochrone}' cycloid, starting under rest.
    \end{enumerate}
\end{remark}


Now suppose $f$ is a function as $f(y_{1}(x),\ldots,y_{n}(x),\dot{y}_{1}(x),\ldots,\dot{y}_{n}(x),x)$. With $J = \int_{x_{1}}^{x_{2}}f dx$ and $y_{k}(x_{1}), y_{k}(x_{2})$ fixed for all $k$, we again parameterize by $\alpha$ to show small changes as
\begin{align}
    y_{k}(x,\alpha) = y_{k}(x,0) + \alpha \eta_{k}(x)
\end{align}
with $\eta_{k}(x_{1}) = \eta_{k}(x_{2}) = 0$ for all $1 \leq k \leq n$. Thus,
\begin{align}
    \delta J = \frac{\partial J}{\partial \alpha} d \alpha = \int_{x_{1}}^{x_{2}} \left( \sum_{i=1}^{n} \left( \frac{\partial f}{\partial y_{i}} \frac{\partial y_{i}}{\partial \alpha} + \frac{\partial f}{\partial \dot{y}_{i}} \frac{\partial \dot{y}_{i}}{\partial \alpha} \right) dx\right).
\end{align}
For all $i$, the second term can be written as $T_{2} = \frac{\partial f}{\partial \dot{y}_{i}} \frac{\partial}{\partial \alpha} \left( \frac{dy_{i}}{dx} \right)$ giving us the second integral-sum as
\begin{align}
    \sum_{i=1}^{n} \int_{x_{1}}^{x_{2}} dx \frac{\partial f}{\partial \dot{y}_{i}} \frac{d}{dx} \left( \frac{\partial y_{i}}{\partial \alpha} \right) = \sum_{i=1}^{n} \left[ \frac{\partial f}{\partial y_{i}} \frac{\partial y_{i}}{\partial \alpha} \bigg|_{x_{1}}^{x_{2}} - \int_{x_{1}}^{x_{2}} \frac{d}{dx} \left( \frac{\partial f}{\partial \dot{y}_{i}} \right) \cdot \frac{\partial y_{i}}{\partial \alpha} dx \right].
\end{align}
Plugging this in, we have
\begin{align}
    \delta J = \frac{\partial J}{\partial \alpha} d\alpha = \int_{x_{1}}^{x_{2}} dx \sum_{i=1}^{n} \left( \frac{\partial f}{\partial y_{i}} - \frac{d}{dx} \left( \frac{\partial f}{\partial \dot{y}_{i}} \right) \right) \delta y_{i} = \sum_{i=1}^{n} \delta y_{i} \left[ \int_{x_{1}}^{x_{2}} dx \left[ \frac{\partial f}{\partial y_{i}} - \frac{d}{dx} \left( \frac{\partial f}{\partial \dot{y}_{i}} \right) \right] \right] = 0.
\end{align}
The $\delta y_{i}$'s are independent, giving the inner integral as zero for all $i$, resulting in
\begin{align}
    \frac{\partial f}{\partial y_{i}} - \frac{d}{dx} \left( \frac{\partial f}{\partial \dot{y}_{i}} \right) = 0.
\end{align}

\section{Lagrange Multipliers}

Finding the independent $q_{i}$'s is \textit{hard}; for a sphere, its simply the spherical coordinates without radial dependence. For a general surface, the solution is complicated and requires studying differential geometry and algebraic geometry. We have to introduce a way to introduce the constraints and eliminate the dependent coordinates. Thus introduces the concept of \eax{Lagrange multipliers}.



Suppose we have a nice hemispherical bowl of radius $R$ lying flat on the ground and a particle of mass $m$ rests atop the bowl. Without the loss of generality, we assume the particle only moves in the $xz$-plane. Thus, $y = 0$, $x = R \sin \theta$, $z = R \cos \theta$. By writing this, we have ensured a constraint $f(x,z) = x^{2}+z^{2}-R^{2} = 0$. Here the kinetic energy $T$ is given by
\begin{align}
    T = \frac{1}{2} m(\dot{x}^{2}+\dot{y}^{2}+\dot{z}^{2}) = \frac{1}{2}(R\dot{\theta} \cos \theta)^{2} + \frac{1}{2}(-R \dot{\theta} \sin \theta)^{2} = \frac{1}{2} m R^{2} \dot{\theta}^{2}.
\end{align}
The potential is
\begin{align}
    V = mgz = mgR \cos \theta.
\end{align}
The Lagrangian is
\begin{align}
    L = T - V = \frac{1}{2} m R^{2} \dot{\theta}^{2} - mgR \cos \theta.
\end{align}
The Euler-Lagrange equation is
\begin{align}
    \frac{d}{dt}\left( \frac{\partial L}{\partial \dot{\theta}} \right) - \frac{\partial L}{\partial \theta} = mR^{2}\ddot{\theta}-(mgR \sin \theta) = 0 \implies \ddot{\theta} = \frac{g}{R} \sin \theta.
\end{align}
Here, our constraint was fixed throughout; if we wanted to know the angle $\theta$ at which the particle leaves the surface, the constraint then would no longer be applicable.

In actuality, the particle presses a little on the bowl compressing it a tiny bit. Van der waals forces then react and push back with a normal force. Thus the distance $r$ of the particle from the centre of the bowl changes a little, and we model this $x = r \sin \theta$ and $z = r \cos \theta$, with
\begin{align}
    L = \frac{1}{2}m(\dot{r}^{2} + r^{2} \dot{\theta}^{2}) - mgR \cos \theta - V(r).
\end{align}
Here we get $\frac{d}{dt} \left( \frac{\partial L}{\partial \dot{r}} \right) = m\ddot{r}$ and $\frac{\partial L}{\partial r} = mr\dot{\theta}^{2} - mg\cos\theta - \frac{dV}{dr}$. The Euler-Lagrange equations are
\begin{align}
    m\ddot{r} - mr\dot{\theta}^{2} + mg\cos\theta + \frac{dV}{dr} &= 0,\\
    mr^{2}\ddot{\theta} + 2mr\dot{r}\dot{\theta} - mgr \sin \theta &= 0.
\end{align}
If we now put in the constraints $r(t) = R$ with $\dot{r} = 0$ and $\ddot{r} = 0$, we get
\begin{align}
    mR \dot{\theta}^{2} = mg \cos \theta + \frac{dV}{dr} \bigg|_{R}, \qquad mR^{2}\ddot{\theta} = mgR \sin \theta.
\end{align}
Plugging in the constraint later has given us an additional equation as $- F_{N} = \frac{dV}{dr} \big|_{R} = mR\dot{\theta}^{2} - mg \cos \theta$.\\ \\
\textit{August 20th.}

We now discuss how exactly one includes the constraints $\{f_{\alpha}(q_{i},t) = 0 \mid \alpha = 1,2,\ldots,k,\; k < N\}$ in Hamilton's principle. This is where the method of Lagrange multipliers comes in. Suppose we have $n$ generalized coordinates $\{q_{i}\}$ with $m$ constraint equations, for $n > m$. We can then form the augmented Lagragian as
\begin{align}
    S = \int_{1}^{2} dt \left( L + \sum_{j=1}^{m} \lambda_{j}f_{i}(\{q_{j}\},t) \right) \implies \delta S = 0
\end{align}
where $\{\lambda_{i}\}$, from $i = 1$ to $m$, are termed \eax{Lagrange multipliers}. This expands as
\begin{align}
    \delta S = \int_{1}^{2} dt \left( \sum_{i=1}^{n} \left( \frac{d}{dt} \left( \frac{\partial L}{\partial \dot{q}_{i}} \right) - \frac{\partial L}{\partial q_{i}} + \sum_{j=1}^{m} \lambda_{j} \frac{\partial f_{j}}{\partial q_{i}} \right) \delta q_{i} \right) = 0.
\end{align}
We choose the $m$ multipliers $\lambda_{j}$'s such that the first $m$ coefficients vanish, and we are left with $n-m$ coordinates that are now linearly independent. With $L$ still being $T-V$, we get
\begin{align}
    \frac{d}{dt}\left( \frac{\partial L}{\partial \dot{q}_{i}} \right) - \frac{\partial L}{\partial q_{i}} + \sum_{j=1}^{m} \lambda_{j} \frac{\partial f_{j}}{\partial q_{i}} = 0, \quad i = 1,2,\ldots,n.
\end{align}

\begin{example}
    Consider the prior example of a point mass atop a hemispherical bowl of radius $R$. Here, $L = \frac{1}{2}m(\dot{x}^{2}+\dot{z}^{2}) - mgz$. The easier of writing this is to switch to polar coordinates, so $L = \frac{1}{2}m(\dot{r}^{2}+r^{2}\dot{\theta}^{2}) - mgr \cos \theta$. The only constraint here is $f = r - R$. So the augmented Lagrangian here is
    \begin{align}
        \tilde{L} = L + \lambda(r-R)
    \end{align}
    The equations can be worked out to get
    \begin{align}
        mr\ddot{r}-mr\dot{\theta}^{2} + mg\cos\theta + \lambda_{1} &= 0,\\
        2mr\dot{r}\dot{\theta} + mr^{2}\ddot{\theta} - mgr\sin\theta &= 0.
    \end{align}
    One can then solve and plug in $r(t) = R$.
\end{example}

\begin{example}
    For a ball of radius $R$ rolling without slipping, the distance $x$ travelled across the slope is given as $R\dot{\theta} = \dot{x}$. Thus, this constraint is $R\dot{\theta} - \dot{x}$.
\end{example}

\section{Potential Dependent on Generalized Velocity}
\textit{September 1st.}

Essentially, a redefinition of the (generalized) momentum. Let us take the example of a point mass with charge moving in an electromagnetic filed. The force experienced by this particle is given by
\begin{align}
    \vec{F} = m\vec{a} = q(\vec{E} + \vec{v} \times \vec{B})
\end{align}
where $q$ is the charge on the particle, and $\vec{E}$ and $\vec{B}$ are the electric field and magnetic field respectively. Moreover, $\vec{E} \equiv \vec{E}(\vec{r},t)$ and $\vec{B} \equiv \vec{B}(\vec{r},t)$. One actually rewrites the magnetic field as
\begin{align}
    \vec{B}(\vec{r},t) = \vec{\nabla} \times \vec{A}(\vec{r},t)
\end{align}
where $\vec{A}(\vec{r},t)$ is termed the \eax{vector potential}. The magnetic field satisfies the constraint of $\vec{\nabla} \cdot \vec{B} = 0$ (In general, one can show that the curl of a divergence vector is always zero). When $\vec{\nabla} \times \vec{E} = 0$ is valid, the the vector field $\vec{E}$ can be written as $-\vec{\nabla} \varphi$. Also, for any vector field $\vec{B}$, if $\vec{\nabla} \times \vec{v}(\vec{r},t) = 0$, then the velocity can always be written as $\vec{v}(\vec{r},t) = \vec{\nabla}f(\vec{r},t)$. If the above is not valid, we instead have Faraday's law as $\vec{\nabla} \times \vec{E} = -\frac{\partial \vec{B}}{\partial t}$. In particular, it can be written as
\begin{align}
    \vec{E}(\vec{r},t) = -\vec{\nabla} \varphi(\vec{r},t) - \frac{\partial \vec{A}(\vec{r},t)}{\partial t}.
\end{align}
In this case, the velocity-dependent potential is
\begin{align}
    U = q \phi(\vec{r},t) - q\vec{A}(\vec{r},t) \cdot \vec{v}
\end{align}
which, in turn, gives the Lagrangian (in cartesian coordinates)
\begin{align}
    L = T - U = \frac{1}{2} m(\dot{x}^{2}+\dot{y}^{2}+\dot{z}^{2})-q\phi(\vec{r},t) + q\vec{A}(\vec{r},t) \cdot \vec{v}.
\end{align}
Here in our formal notation, $q_{1} = x$, $q_{2} = y$, $q_{3} = z$, and the corresponding time derivatives. Thus our equation of motion, for $q_{i} = q_{1} = x$, is
\begin{align}
    \frac{d}{dt} \left( \frac{\partial L}{\partial \dot{q}_{i}} \right) - \frac{\partial L}{\partial q_{i}} = 0.
\end{align}
Here,
\begin{align}
    \frac{\partial L}{\partial \dot{x}} = m\dot{x} + qA_{x} \implies \frac{d}{dt} \left( \frac{\partial L}{\partial \dot{x}} \right) = m\ddot{x} + q \frac{\partial A_{x}}{\partial t} + q \left( \frac{\partial A_{x}}{\partial x} \dot{x} + \frac{\partial A_{x}}{\partial y} \dot{y} + \frac{\partial A_{x}}{\partial z} \dot{z} \right).
\end{align}
One can define $\vec{\nabla}A_{x} = \frac{\partial A_{x}}{\partial x} \hat{e}_{x} + \frac{\partial A_{x}}{\partial y} \hat{e}_{y} + \frac{\partial A_{x}}{\partial z} \hat{e}_{z}$, and $\vec{v} = \dot{x}\hat{e}_{x} + \dot{y}\hat{e}_{y} + \dot{z}\hat{e}_{z}$, giving us
\begin{align}
    \frac{d}{dt} \left( \frac{\partial L}{\partial \dot{x}} \right) = m\ddot{x} + q \frac{\partial A_{x}}{\partial t} + q (\vec{v} \cdot \vec{\nabla}A_{x}).
\end{align}
For the second term in the Lagragian, we have
\begin{align}
    \frac{\partial L}{\partial x} = -q\frac{\partial \phi}{\partial x} + q \left( v_{x} \frac{\partial A_{x}}{\partial x} + v_{y} \frac{\partial A_{y}}{\partial y} + v_{z} \frac{\partial A_{z}}{\partial z} \right).
\end{align}
Plugging them into the Lagrangian equation of motion, and using the fact that $\vec{E} = -\vec{\nabla} \phi - \frac{\partial \vec{A}}{\partial t} \implies E_{x} = -\frac{\partial \phi}{\partial x} - \frac{\partial A_{x}}{\partial t}$, we get
\begin{align}
    m\ddot{x} + q\frac{\partial A_{x}}{\partial t} + q\frac{\partial \phi}{\partial x} + q \left( v_{x}\frac{\partial A_{x}}{\partial x} + v_{y}\frac{\partial A_{x}}{\partial y} + v_{z}\frac{\partial A_{x}}{\partial z} \right) - q \left( v_{x} \frac{\partial A_{x}}{\partial x} + v_{y} \frac{\partial A_{y}}{\partial x} + v_{z} \frac{\partial A_{z}}{\partial x} \right) = 0 \\
    \implies m\ddot{x} + q(-E_{x}) + q \left( v_{y} \left( \frac{\partial A_{x}}{\partial y} - \frac{\partial A_{y}}{\partial x} \right) + v_{z} \left( \frac{\partial A_{x}}{\partial z} - \frac{\partial A_{z}}{\partial x} \right) \right) = 0
\end{align}
Similarly, one can show that $B_{z} = (\vec{\nabla} \times \vec{A})_{z} = \partial_{x}A_{y} - \partial_{y}A_{x}$ and $B_{y} = (\vec{\nabla} \times \vec{A})_{y} = \partial_{z}A_{x} - \partial_{x}A_{z}$. Thus, the above transforms as
\begin{align}
    m\ddot{x} -q E_{x} + q (v_{y}(-B_{z}) + v_{z}B_{y}) = 0.
\end{align}
Using the cross product for the final term, we get
\begin{align}
    m\ddot{x} = \left( q\vec{E} + q(\vec{v} \times \vec{B}) \right)_{x}
\end{align}
This is the exact force required to describe the motion of a charged particle in an electromagnetic field.\\

\section{Cyclic Coordinates and Conservation Laws}
Recall the energies and how we write it:
\begin{align}
    T = \sum_{i=1}^{N} \frac{1}{2}m_{i}(\dot{x}_{i}^{2}+\dot{y}_{i}^{2}+\dot{z}_{i}^{2}),\quad U = U(\{x_{i},y_{i},z_{i}\}).
\end{align}
Here,
\begin{align}
    \frac{\partial L}{\partial \dot{x}_{i}} = m_{i}\dot{x}_{i} = p_{x,i},\quad \frac{\partial L}{\partial \dot{y}_{j}} = p_{y,j}.
\end{align}
In terms of the generalized coordinates, $L = L(\{q_{1},\ldots,q_{n},\dot{q}_{1},\ldots,\dot{q}_{n},t\})$, and $p_{i} = \frac{\partial L}{\partial \dot{q}_{i}}$. For $L$ independent of some $q_{i}$, we have $\frac{\partial L}{\partial q_{i}} = 0$. This $q_{i}$ is termed a \eax{cyclic coordinates}, and the equation of motion gives
\begin{align}
    \frac{d}{dt} \left( \frac{\partial L}{\partial \dot{q}_{i}} \right) = 0 \implies \frac{dp_{i}}{dt} = 0.
\end{align}
For a rigid body, let $q_{j}$ be the coordinate(s) of its center of mass. Then, the Lagrangian equation of motion (where $V \equiv V(\{q_{j}\}))$ is
\begin{align}
    \frac{d}{dt}\left( \frac{\partial T}{\partial \dot{q}_{j}} \right) + \frac{\partial V}{\partial q_{j}} = 0 \implies \frac{dp_{j}}{dt} = -\frac{\partial V}{\partial q_{j}} = Q_{j}.
\end{align}
In $Q_{j} = \sum_{i=1}^{N} \vec{F}_{i} \cdot \frac{\partial \vec{r}_{i}}{\partial q_{j}}$, one has
\begin{align}
    \frac{\partial \vec{r}_{i}}{\partial q_{j}} = \lim_{dq_{j} \to 0} \frac{\vec{r}_{i}(q_{j}+dq_{j})-\vec{r}_{i}(q_{j})}{dq_{j}} = \lim_{dq_{j} \to 0} \frac{dq_{j} \hat{e}_{n}}{dq_{j}} = \hat{e}_{n}.
\end{align}
Thus, $Q_{j} = \sum_{i=1}^{N} \vec{F}_{i} \cdot \hat{e}_{n} = \vec{F}_{\text{sys}} \cdot \hat{e}_{n}$. Using $T = \sum_{i=1}^{N} \frac{1}{2} m_{i} v_{i}^{2}$, we get
\begin{align}
    p_{j} = \frac{\partial T}{\partial \dot{q}_{j}} = \sum_{i=1}^{N} m_{r} \dot{\vec{r}}_{i} \cdot \hat{e}_{n} \implies \frac{dp_{j}}{dt} = \frac{d}{dt} \left( \sum_{i=1}^{n} m_{i} \vec{v}_{i} \cdot \hat{e}_{n} \right) = Q_{j} = \vec{F}_{\text{sys}} \cdot \hat{e}_{n}.
\end{align}
Thus the Newton's law statement has been translated into the generalized coordinates statement. Now suppose $q_{j}$ is cyclic and $dq_{j}$ is rotation about an axis. Here, we again take $\frac{\partial T}{\partial q_{j}} = 0$ and $\frac{\partial V}{\partial q_{j}} = 0$. Thus, the equation of motion gives
\begin{align}
    \dot{p}_{j} = \frac{d}{dt} \left( \frac{\partial L}{\partial \dot{q}_{j}} \right) = \frac{d}{dt} \left( \frac{\partial T}{\partial \dot{q}_{j}} \right) = \frac{\partial L}{\partial q_{j}} = -\frac{\partial V}{\partial q_{j}} = Q_{j}.
\end{align}
Working as above, $Q_{j} = \sum_{i=1}^{N} \vec{F}_{i} \cdot \frac{\partial \vec{r}_{i}}{\partial q_{j}}$, and
\begin{align}
    \frac{\partial \vec{r}_{i}}{\partial q_{j}} = \lim_{dq_{j} \to 0} \frac{\vec{r}_{i} (q_{j} + dq_{j}) - \vec{r}_{i}(q_{j})}{dq_{j}} = \lim_{dq_{j} \to 0} \frac{dq_{j} \hat{e}_{n}}{dq_{j}} = \hat{e}_{n}.
\end{align}
To proceed, one has
\begin{align}
    \abs{\partial \vec{r}_{i}} = \abs{\vec{r}_{i}} \sin \theta dq_{j} \implies \abs{\frac{\partial \vec{r}_{i}}{\partial q_{j}}} = \abs{\vec{r}_{i}} \abs{\hat{e}_{n}} \sin(\theta) \implies \frac{\partial \vec{r}_{i}}{\partial q_{j}} = \hat{e}_{n} \times \vec{r}_{i}.
\end{align}
where $\theta$ is the angle between $\vec{r}_{i}$ and $\hat{e}_{n}$. This tells us $Q_{j} = \sum_{i=1}^{N} \vec{F}_{i} \cdot (\hat{e}_{n} \times \vec{r}_{i}) = \sum_{i=1}^{N} (\vec{r}_{i} \times \vec{F}_{i}) \cdot \hat{e}_{n} = \vec{\tau}_{\text{tot,sys}} \cdot \hat{e}_{n}$.
\begin{align}
    p_{j} = \frac{\partial T}{\partial \dot{q}_{j}} = \sum_{i=1}^{N} m_{i} \vec{v}_{i} \cdot (\hat{e}_{n} \times \vec{r}_{i}) = \sum_{i=1}^{N} \hat{e}_{n} \cdot \vec{L}_{i} = \vec{L}_{\text{tot}} \cdot \hat{e}_{n} \implies \dot{p}_{j} = \frac{d}{dt} \left( \sum_{i=1}^{N} \vec{L}_{i} \cdot \hat{e}_{n} \right) = Q_{j} = \vec{\tau}_{\text{sys}} \cdot \hat{e}_{n}.
\end{align}
\\\textit{September 3rd.}

For $L(\{q_{i}\},\{\dot{q}_{i}\},t)$, we have
\begin{align}
    \frac{dL}{dt} = \sum_{i=1}^{n} \left(\frac{\partial L}{\partial q_{i}} \dot{q}_{i} + \frac{\partial L}{\partial \dot{q}_{i}} \ddot{q}_{i} \right) + \frac{\partial L}{\partial t} = \frac{d}{dt} \left( \sum_{i=1}^{n} \dot{q}_{i} \frac{\partial L}{\partial \dot{q}_{i}} \right) + \frac{\partial L}{\partial t} \implies \frac{d}{dt} \left( \sum_{i=1}^{n} \dot{q}_{i} \frac{\partial L}{\partial \dot{q}_{i}} - L\right) + \frac{\partial L}{\partial t} = 0.
\end{align}
Here, one then defines the \eax{energy function} as $h(\{q_{i}\},\{\dot{q}_{i}\},t) = \sum_{i=1}^{n} \dot{q}_{i} \frac{\partial L}{\partial \dot{q}_{i}} - L = \sum_{i=1}^{n} \dot{q}_{i} p_{i} - L$, resulting in
\begin{align}
    \frac{dh}{dt} = -\frac{\partial L}{\partial t}.
\end{align}
Thus, if the Lagrangian doesn't have an explicit time dependence, the energy function $h$ is conserved. Physically, a time-translation invariance results in the conservation of $h$. Recall the expression for the total kinetic energy $T = T_{0} + T_{1} + T_{2}$ where
\begin{align}
    T_{0} = \sum_{i=1}^{N} \frac{1}{n}m_{i}\abs{\frac{\partial \vec{r}_{i}}{\partial t}}^{2},\quad T_{1} = \sum_{j=1}^{n} \dot{q}_{j}M_{j}, \quad T_{2} = \sum_{j=1}^{n} \sum_{k=1}^{n} \dot{q}_{j} \dot{q}_{k} M_{jk}
\end{align}
and
\begin{align}
    M_{j} = \sum_{i=1}^{N} m_{i} \frac{\partial \vec{r}_{i}}{\partial t} \cdot \frac{\partial \vec{r}_{i}}{\partial q_{j}},\quad M_{jk} = \sum_{i=1}^{N} \frac{1}{2}m_{i} \frac{\partial \vec{r}_{i}}{\partial q_{j}} \cdot \frac{\partial \vec{r}_{i}}{\partial q_{k}}.
\end{align}
Now, if $\vec{r}_{i} = \vec{r}_{i}(\{q_{j}\})$ for all $i = 1,2,\ldots,N$, then $\frac{\partial \vec{r}_{i}}{\partial t} = 0$ tells us that $T = T_{2}$. Let us call this our first condition (i). For our second condition (ii), it may so happen that the Lagrangian $L = L_{0}(\{q\},t) + L_{1}(\{q\},\{\dot{q}\},t) + L_{2}(\{q\},\{\dot{q}\},t)$, where $L_{0}$ is independent of the generalized velocities, $L_{1}$ is linear in the generalized velocities, and $L_{2}$ is quadratic in the generalized velocities. One can show that if $f(\{x_{i}\})$ is homogenous in degree $n$, 
\begin{align}
    \sum_{i=1}^{n} x_{i} \frac{\partial f}{\partial x_{i}} = n f.
\end{align}
Applying this to the Lagrangian $L = L_{0} + L_{1} + L_{2}$, we find
\begin{align}
    h(\{q_{i}\},\{\dot{q}_{i}\},t) = \sum_{i=1}^{n} \dot{q}_{i} \frac{\partial L}{\partial \dot{q}_{i}} - L = (2L_{2}+L_{1})-(L_{0}+L_{1}+L_{2}) = L_{2} - L_{0}.
\end{align}
Our this third condition (iii) is that the potentail $V$ is independent of the velocities $\{\dot{q}_{k}\}$. With these three conditions, one has $L_{2} = T_{2} = T$ and $L_{0} = -V$, showing
\begin{align}
    h = L_{2} - L_{0} = T-(-V) = T+V = E,
\end{align}
the total energy. Note that $\dot{h} = -\frac{\partial L}{\partial t} = 0$ is \textit{not} equivalent to total energy conserved; the prior statements says that the energy function is conserved. Conditions (i), (ii), and (iii) must all be satisfied for the energy function to be the total energy.
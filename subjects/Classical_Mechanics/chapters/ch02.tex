\chapter{PLACEHOLDER}

\section{Integral Principle}
\textit{August 11th.}

Recall the assumptions we made prior to deriving the equations of motion:
\begin{enumerate}
    \item $\vec{F}_{i}$ is decomposed into $\vec{F}_{i}^{(a)}$ and $\vec{f}_{i}$.
    \item We have $\sum_{i=1}^{N_{p}} \vec{f}_{i} \cdot \delta \vec{r}_{i} = 0$.
    \item Each $\vec{r}_{i}$ is expressed as $\vec{r}_{i}(q_{1},q_{2},\ldots,q_{n},t)$, where the $\delta q_{k}$'s are independent.
\end{enumerate}

From our assumptions, there are no longer any constraint forces. Also recall the generalized forces where
\begin{align}
    Q_{j} = \sum_{i=1}^{N_{p}} \vec{F}_{i}^{(a)} \cdot \frac{\partial \vec{r}_{i}}{\partial q_{j}}, \quad j = 1,2,\ldots,n,
\end{align}
satisfying
\begin{align}
    \frac{d}{dt} \left( \frac{\partial T}{\partial \dot{q}_{i}} \right) - \frac{\partial T}{\partial q_{i}} - Q_{i} = 0, \quad i = 1,2,\ldots,n,
\end{align}
where $T = \sum_{i=1}^{N_{p}} \frac{1}{2} m_{i} \abs{\dot{\vec{r}}_{i}}^{2}$ is the kinetic energy. If a $V$ can be defined such that $\vec{F}_{i} = -\vec{\nabla}_{i} V$, then $L(q,\dot{q},t) = T - V$, and
\begin{align}
    \frac{d}{dt} \left( \frac{\partial L}{\partial \dot{q}_{j}} \right) - \frac{\partial L}{\partial q_{j}} = 0, \quad j = 1,2,\ldots,n,
\end{align}
where $-\frac{\partial V}{\partial q_{j}} = Q_{j}$ and $V$ does not depend on $\dot{q}_{j}$.

We now discuss the \eax{integral principle}, also known as \eax{Hamilton's principle}. As a precursor, we have $Q_{j} = U(\{q_{j}\},\{\dot{q}_{j}\},t)$ where $U$ is a function of the generalized coordinates, velocities, and time, and $L(\{q_{j}\},\{\dot{q}_{j}\},t) = T-U$. The \eax{action} is defined as
\begin{align}
    S = \int_{t_{i}}^{t_{f}} L(\{q_{j}\},\{\dot{q}_{j}\},t)dt.
\end{align}
The principle states that the physical trajectory taken by the system between two times $t_{i}$ and $t_{f}$ is the one for which the action $S$ is stationary, usually an extremum.

\subsection{Variational Calculus}

We set up a function $f(y,\frac{dy}{dx},x)$ where we have the correspondences $f \leftrightarrow x$, $q \leftrightarrow y$, and $L \leftrightarrow f$, which is analogous to writing $L(q,\dot{q},t)$. We also identify $\frac{dy}{dx}$ with $\dot{y}$. Then, correspondingly, we have
\begin{align}
    J \equiv \int_{x_{1}}^{x_{2}} f(y,\dot{y},x)dx.
\end{align}
Setting $y(x_{1}) = y_{1}$ and $y(x_{2}) = y_{2}$, we aim to minimize $J$ with respect to $y$. We ask how does one \textit{vary} $y(x)$, a function? To rectify this issue, we introduce a new parameter $\alpha$ in $y(x,\alpha)$ such that $y(x,0) = y(x)$. In other words,
\begin{align}
    y(x,\alpha) = y(x,0) + \alpha \eta(x),
\end{align}
where $\eta(x)$ is an arbitrary function that vanishes at the endpoints: $\eta(x_{1}) = \eta(x_{2}) = 0$. It is safe to assume here that every possible smooth path from $x_{1}$ to $x_{2}$ can be represented in this form; the neighbouring paths are parametrized by $\alpha$. Thus,
\begin{align}
    J[f] = \int_{x_{1}}^{x_{2}}dx \cdot f(y(x,\alpha),\dot{y}(x,\alpha),x).
\end{align}
We can now differentiate $J$ with respect to $\alpha$ and set it to $0$ at $\alpha = 0$.
\begin{align}
    \frac{dJ}{d\alpha} = \int_{x_{1}}^{x_{2}} \left( \frac{\partial f}{\partial y} \frac{\partial y}{\partial \alpha} + \frac{\partial f}{\partial \dot{y}} \frac{\partial \dot{y}}{\partial \alpha} \right) dx.
\end{align}
Using the fact that $\frac{\partial \dot{y}}{\partial \alpha} = \frac{d}{dx} \left( \frac{\partial y}{\partial \alpha} \right) = \frac{\partial^{2} y}{\partial x \partial \alpha}$ and noticing the $dx$ term, the integral screams out to use integration by parts. Thus, on the rightmost term of the integral
\begin{align}
    \int_{x_{1}}^{x_{2}} dx \cdot \frac{\partial f}{\partial \dot{y}} \frac{d}{dx} \left( \frac{\partial y}{\partial \alpha}\right) = \left[-\int_{x_{1}}^{x_{2}} \frac{\partial y}{\partial x} \frac{\partial}{\partial x} \left( \frac{\partial f}{\partial \dot{y}} \right) dx \right] + \frac{\partial f}{\partial \dot{y}} \frac{\partial y}{\partial \alpha} \bigg|_{x_{1}}^{x_{2}}.
\end{align}
We plug this back in the derivative of $J$, taking note that $\frac{\partial y}{\partial \alpha} = \eta(x)$.
\begin{align}
    \frac{dJ}{d\alpha} = \int_{x_{1}}^{x_{2}} \left( \frac{\partial f}{\partial y} - \frac{d}{dx} \left( \frac{\partial f}{\partial \dot{y}} \right) \right) \left( \frac{\partial y}{\partial \alpha} \right) dx = \int_{x_{1}}^{x_{2}} dx\left( \frac{\partial f}{\partial y} - \frac{d}{dx} \left( \frac{\partial f}{\partial \dot{y}} \right) \right) \eta(x) = 0.
\end{align}
Since this works for \textit{any} $\eta$ following our initial constraints, we must have
\begin{align}
    \frac{d}{dx} \left( \frac{\partial f}{\partial \dot{y}} \right) - \frac{\partial f}{\partial y} = 0.
\end{align}

\begin{example}
    We show that, without any external conditions, the shortest path between two points $(x_{1},y_{1})$ and $(x_{2},y_{2})$ is a straight line. Each path segment is characterized as $(ds)^{2} = (dx)^{2} + (dy)^{2}$. Thus,
    \begin{align}
        J = \int_{x_{1}}^{x_{2}} ds = \int_{x_{1}}^{x_{2}}\sqrt{(dx)^{2}+(dy)^{2}} = \int_{x_{1}}^{x_{2}} dx \sqrt{1 + \left( \frac{dy}{dx} \right)^{2}} = \int_{x_{1}}^{x_{2}} dx (1+\dot{y}^{2})^{1/2}
    \end{align}
    Identifying with our equation, we have $f = \sqrt{1+\dot{y}^{2}}$, and $\frac{\partial f}{\partial \dot{y}} = \frac{\dot{y}}{\sqrt{1+\dot{y}^{2}}}$ and $\frac{\partial f}{\partial y} = 0$. Thus, the equation tell us
    \begin{align}
        \frac{d}{dx} \left( \frac{\dot{y}}{\sqrt{1+\dot{y}^{2}}} \right) = 0 \implies \frac{\dot{y}}{\sqrt{1+\dot{y}^{2}}} = C \implies \dot{y} = \sqrt{\frac{C^{2}}{1-C^{2}}}.
    \end{align}
\end{example}


\begin{example}
    As another example, we show that the shortest path between the north pole $(\theta_{1} = 0)$ and south pole $(\theta_{2} = \pi)$ on a sphere is a great circle. Here, the movement in spherical coordinates is $\vec{dt} = dr \hat{e}_{r} + rd\theta \hat{e}_{\theta} + r \sin \theta d\phi \hat{e}_{\phi}$. For a sphere, $R$ is constant so really we have $\vec{dr} = Rd \theta \hat{e}_{\theta} + R \sin \theta d\phi \hat{e}_{\phi}$. Thus, the integral becomes
    \begin{align}
        J = \int_{\theta_{1}}^{\theta_{2}} \sqrt{\vec{dr} \cdot \vec{dr}} = \int_{\theta_{1}}^{\theta_{2}} R \sqrt{1 + \sin^{2}\theta \dot{\phi}^{2}} d \theta.
    \end{align}
    Thus, we identify $x$ with $\theta$, $y(x)$ with $\phi(\theta)$, $f$ with $\sqrt{1+\sin^{2}\theta \dot{\phi}^{2}}$. Plugging the partials in the equation, we get
    \begin{align}
        \frac{R\sin^{2}\theta\dot{\phi}}{\sqrt{1+\sin^{2}\theta \dot{\phi}^{2}}} = C
    \end{align}
    This works for any $\theta$; plugging in $\theta_{1} = 0$ shows that $C = 0$. Thus, we must conclude that $\dot{\phi} \equiv 0$; the condition of a great circle. Notice that our choice of north pole and south pole could be arbitrary. Thus, the shortest distance between any two points on a sphere $\theta_{1} = 0$ and $\theta_{2}$ is a great circle.
\end{example}

\noindent \textit{August 18th.}

\begin{remark}
    The following may be shown via variational calculus.
    \begin{enumerate}
        \item A geodesic in $\R^{n}$ is a `straight' line.
        \item A geodesic on a sphere $S^{2}$ is a `great circle'.
        \item The minimum time path, with end points fixed, freely falling under $Mg$ is a `\eax{brachistochrone}' cycloid, starting under rest.
    \end{enumerate}
\end{remark}


Now suppose $f$ is a function as $f(y_{1}(x),\ldots,y_{n}(x),\dot{y}_{1}(x),\ldots,\dot{y}_{n}(x),x)$. With $J = \int_{x_{1}}^{x_{2}}f dx$ and $y_{k}(x_{1}), y_{k}(x_{2})$ fixed for all $k$, we again parameterize by $\alpha$ to show small changes as
\begin{align}
    y_{k}(x,\alpha) = y_{k}(x,0) + \alpha \eta_{k}(x)
\end{align}
with $\eta_{k}(x_{1}) = \eta_{k}(x_{2}) = 0$ for all $1 \leq k \leq n$. Thus,
\begin{align}
    \delta J = \frac{\partial J}{\partial \alpha} d \alpha = \int_{x_{1}}^{x_{2}} \left( \sum_{i=1}^{n} \left( \frac{\partial f}{\partial y_{i}} \frac{\partial y_{i}}{\partial \alpha} + \frac{\partial f}{\partial \dot{y}_{i}} \frac{\partial \dot{y}_{i}}{\partial \alpha} \right) dx\right).
\end{align}
For all $i$, the second term can be written as $T_{2} = \frac{\partial f}{\partial \dot{y}_{i}} \frac{\partial}{\partial \alpha} \left( \frac{dy_{i}}{dx} \right)$ giving us the second integral-sum as
\begin{align}
    \sum_{i=1}^{n} \int_{x_{1}}^{x_{2}} dx \frac{\partial f}{\partial \dot{y}_{i}} \frac{d}{dx} \left( \frac{\partial y_{i}}{\partial \alpha} \right) = \sum_{i=1}^{n} \left[ \frac{\partial f}{\partial y_{i}} \frac{\partial y_{i}}{\partial \alpha} \bigg|_{x_{1}}^{x_{2}} - \int_{x_{1}}^{x_{2}} \frac{d}{dx} \left( \frac{\partial f}{\partial \dot{y}_{i}} \right) \cdot \frac{\partial y_{i}}{\partial \alpha} dx \right].
\end{align}
Plugging this in, we have
\begin{align}
    \delta J = \frac{\partial J}{\partial \alpha} d\alpha = \int_{x_{1}}^{x_{2}} dx \sum_{i=1}^{n} \left( \frac{\partial f}{\partial y_{i}} - \frac{d}{dx} \left( \frac{\partial f}{\partial \dot{y}_{i}} \right) \right) \delta y_{i} = \sum_{i=1}^{n} \delta y_{i} \left[ \int_{x_{1}}^{x_{2}} dx \left[ \frac{\partial f}{\partial y_{i}} - \frac{d}{dx} \left( \frac{\partial f}{\partial \dot{y}_{i}} \right) \right] \right] = 0.
\end{align}
The $\delta y_{i}$'s are independent, giving the inner integral as zero for all $i$, resulting in
\begin{align}
    \frac{\partial f}{\partial y_{i}} - \frac{d}{dx} \left( \frac{\partial f}{\partial \dot{y}_{i}} \right) = 0.
\end{align}

\section{Lagrange Multipliers}

Finding the independent $q_{i}$'s is \textit{hard}; for a sphere, its simply the spherical coordinates without radial dependence. For a general surface, the solution is complicated and requires studying differential geometry and algebraic geometry. We have to introduce a way to introduce the constraints and eliminate the dependent coordinates. Thus introduces the concept of \eax{Lagrange multipliers}.



Suppose we have a nice hemispherical bowl of radius $R$ lying flat on the ground and a particle of mass $m$ rests atop the bowl. Without the loss of generality, we assume the particle only moves in the $xz$-plane. Thus, $y = 0$, $x = R \sin \theta$, $z = R \cos \theta$. By writing this, we have ensured a constraint $f(x,z) = x^{2}+z^{2}-R^{2} = 0$. Here the kinetic energy $T$ is given by
\begin{align}
    T = \frac{1}{2} m(\dot{x}^{2}+\dot{y}^{2}+\dot{z}^{2}) = \frac{1}{2}(R\dot{\theta} \cos \theta)^{2} + \frac{1}{2}(-R \dot{\theta} \sin \theta)^{2} = \frac{1}{2} m R^{2} \dot{\theta}^{2}.
\end{align}
The potential is
\begin{align}
    V = mgz = mgR \cos \theta.
\end{align}
The Lagrangian is
\begin{align}
    L = T - V = \frac{1}{2} m R^{2} \dot{\theta}^{2} - mgR \cos \theta.
\end{align}
The Euler-Lagrange equation is
\begin{align}
    \frac{d}{dt}\left( \frac{\partial L}{\partial \dot{\theta}} \right) - \frac{\partial L}{\partial \theta} = mR^{2}\ddot{\theta}-(mgR \sin \theta) = 0 \implies \ddot{\theta} = \frac{g}{R} \sin \theta.
\end{align}
Here, our constraint was fixed throughout; if we wanted to know the angle $\theta$ at which the particle leaves the surface, the constraint then would no longer be applicable.

In actuality, the particle presses a little on the bowl compressing it a tiny bit. Van der waals forces then react and push back with a normal force. Thus the distance $r$ of the particle from the centre of the bowl changes a little, and we model this $x = r \sin \theta$ and $z = r \cos \theta$, with
\begin{align}
    L = \frac{1}{2}m(\dot{r}^{2} + r^{2} \dot{\theta}^{2}) - mgR \cos \theta - V(r).
\end{align}
Here we get $\frac{d}{dt} \left( \frac{\partial L}{\partial \dot{r}} \right) = m\ddot{r}$ and $\frac{\partial L}{\partial r} = mr\dot{\theta}^{2} - mg\cos\theta - \frac{dV}{dr}$. The Euler-Lagrange equations are
\begin{align}
    m\ddot{r} - mr\dot{\theta}^{2} + mg\cos\theta + \frac{dV}{dr} &= 0,\\
    mr^{2}\ddot{\theta} + 2mr\dot{r}\dot{\theta} - mgr \sin \theta &= 0.
\end{align}
If we now put in the constraints $r(t) = R$ with $\dot{r} = 0$ and $\ddot{r} = 0$, we get
\begin{align}
    mR \dot{\theta}^{2} = mg \cos \theta + \frac{dV}{dr} \bigg|_{R}, \qquad mR^{2}\ddot{\theta} = mgR \sin \theta.
\end{align}
Plugging in the constraint later has given us an additional equation as $- F_{N} = \frac{dV}{dr} \big|_{R} = mR\dot{\theta}^{2} - mg \cos \theta$.
\chapter{THE LAWS OF MOTION AND CONSERVATION}


\section{An Introduction}
\subsection{Standard Conventions}
\textit{July 21st.}

We will be using the convention of $\hat{e}_{x} \times \hat{e}_{y} = \hat{e}_{z}$, where $\times$ denotes the \eax{cross product}, and $\hat{e}_{w}$ is the unit vector along the $w$ axis. For two general vectors $\vec{A}$ and $\vec{B}$, the $i^{\text{th}}$ component of the cross product $\vec{C} = \vec{A} \times \vec{B}$ is given as
\begin{equation}
    [\vec{C}]_{i} = \sum_{j,k} \epsilon_{ijk} [\vec{A}]_{j} [\vec{B}]_{k},
\end{equation}
where $\epsilon_{ijk}$ is the \eax{Levi-Civita symbol}, which is simply the sign of the permutation of the indices $ijk$. A \eax{dot product} between the two vectors is defined as
\begin{equation}
    \vec{A} \cdot \vec{B} = \sum_{i} A_{i}B_{i}.
\end{equation}
Along with the Levi-Civita symbol, we will also be using the \eax{Kronecker delta} $\delta_{ij}$, which is defined as $\delta_{ij} = 1$ if $i = j$ and $0$ otherwise. It can be shown that
\begin{equation}
    \epsilon_{ijk}\epsilon_{ilm} = \delta_{jl}\delta_{km} - \delta_{jm} \delta_{kl}.
\end{equation}

\begin{example}
    The following are some examples of the above notation:
    \begin{itemize}
        \item \eax{Angular momentum}: $\vec{L} = \vec{r} \times \vec{p} \implies L_{i} = \epsilon_{ijk} r_{j}p_{k}$ where $\vec{p}$ is the momentum vector and $\vec{r}$ is the position vector.
        \item \eax{Kinetic energy}: $T = \frac{1}{2} m \vec{v} \cdot \vec{v} = \sum_{i} \frac{1}{2} m v_{i}v_{i} = \sum_{i} \frac{p_{i}p_{i}}{2m}$, where $\vec{v}$ is the velocity vector.
        \item \eax{Torque}: $\vec{\tau} = \vec{r} \times \vec{F} \implies \tau_{i} = \epsilon_{ijk} r_{j}F_{k}$, where $\vec{F}$ is the force vector.
    \end{itemize}
\end{example}

\subsubsection{Gradient, Divergence, and Curl}
\begin{definition}
    The \eax{gradient} of a scalar field $f$ is defined as
    \begin{equation}
        \vec{\nabla} f \defeq  \frac{\partial f}{\partial x} \hat{e}_{x} + \frac{\partial f}{\partial y} \hat{e}_{y} + \frac{\partial f}{\partial z} \hat{e}_{z}.
    \end{equation}
    The gradient points in the direction of the steepest ascent of the function $f$. The gradient operator $\vec{\nabla}$ is also defined as
    \begin{equation}
        \vec{\nabla} \defeq \frac{\partial}{\partial x} \hat{e}_{x} + \frac{\partial}{\partial y} \hat{e}_{y} + \frac{\partial}{\partial z} \hat{e}_{z}.
    \end{equation}
    This definition can be extended to higher dimensions as well.
\end{definition}
The above definition only works really on a \eax{scalar field}, a function from the 3 dimensional space to the real numbers. However, one encounters vector fields as well, which calls for another definition. A \eax{vector field}, in simple terms, is a function that assigns a vector to every point in space. 
\begin{definition}
    The \eax{divergence} of a vector field $\vec{v}(\vec{r}) = v_{x}(x,y,z) \hat{e}_{x} + v_{y}(x,y,z) \hat{e_{y}} + v_{z}(x,y,z) \hat{e}_{z}$, where $v_{i}$ is a scalar field, is defined as
    \begin{equation}
        \vec{\nabla} \cdot \vec{v} \defeq \frac{\partial v_{x}}{\partial x} + \frac{\partial v_{y}}{\partial y} + \frac{\partial v_{z}}{\partial z}.
    \end{equation}
    The operator carries over from the previous definition.
\end{definition}
In Einstein's notation, the components can be rewritten as $\frac{\partial v_{i}}{\partial x_{i}} = \partial_{i} v_{i}$. A cross product may also be performed on vector fields, leading to the following definition.
\begin{definition}
    The \eax{curl} of a vector field $\vec{v}(\vec{r})$ is defined as
    \begin{equation}
        \vec{\nabla} \times \vec{v} \defeq \left( \frac{\partial v_{z}}{\partial y} - \frac{\partial v_{y}}{\partial z} \right) \hat{e}_{x} + \left( \frac{\partial v_{x}}{\partial z} - \frac{\partial v_{z}}{\partial x} \right) \hat{e}_{y} + \left( \frac{\partial v_{y}}{\partial x} - \frac{\partial v_{x}}{\partial y} \right) \hat{e}_{z}.
    \end{equation}
    The operator carries over from the previous definitions.
\end{definition}
In Einstein's notation, the components can be rewritten as $[\vec{\nabla} \times \vec{v}]_{i} = \epsilon_{ijk} \partial_{j} v_{k}$, where $j$ and $k$ are cycled over.

\subsection{The Laws: Mechanics of a Particle}
We first discuss the third law of action-reaction pairs. Simply stated, every action has an equal and opposite reaction. It is important to note that the two action-reaction forces never act on the same body, and hence do not cancel each other out. If a human stands on a spring scale in an elevator, the human exerts a normal force $N_{h/s}$ on the spring, and the spring exerts a normal force $N_{s/h}$ on the human. The spring scale actually measures the force $N_{s/h}$. If the elevator were to accelrate upwards, then the force $N_{h/s}$ remains the same, yet the force $N_{s/h}$ increases, leading to a higher reading on the scale. One must also note that equal in mangnitude and opposite direction does not necessarily mean that they act along the same line. The second law of motion states $\vec{F} = m \vec{a}$, where $\vec{F}$ is the net force acting on a body, $m$ is the mass of the body, and $\vec{a}$ is the acceleration of the body.\\

To formalize this, we discuss the mechanics of a single particle. The particle is at position vector $\vec{r}$, and moves with the velocity vector $\vec{v} = \frac{d\vec{r}}{dt}$. The \eax{momentum} of the particle is defined as $\vec{p} \defeq m \vec{v}$. The net force acting on the particle is defined as $\vec{F} \defeq \frac{d\vec{p}}{dt}$. If the mass is constant, \textit{i.e.}, independent of time, then we can write
\begin{equation}
    \vec{F} = m \frac{d\vec{v}}{dt} = m \vec{a} = m \frac{d^{2}\vec{r}}{dt^{2}}.
\end{equation}
We call our reference frame an \eax{inertial frame} if $\vec{F} = \frac{d \vec{p}}{dt}$ holds true when independently measured. If the net force acting on the particle is zero, then $\frac{d\vec{p}}{dt} = 0$, implying that the momentum $\vec{p}$ is conserved. This is one of the conservation laws.

The \eax{angular momentum} of the particle is defined as $\vec{L} \defeq \vec{r} \times \vec{p}$. The net \eax{torque} acting on the particle is defined as $\vec{\tau} \defeq \vec{r} \times \vec{F} = \vec{r} \times \frac{d}{dt}(m\vec{v})$. If we again assume that the mass is constant, we then have
\begin{equation}
    \frac{d\vec{L}}{dt} = \frac{d}{dt}(\vec{r} \times \vec{p}) = \vec{r} \times \frac{d\vec{p}}{dt} = \vec{r} \times \vec{F} = \vec{\tau}.
\end{equation}
Thus, if $\vec{\tau} = 0$, then $\vec{L}$ is conserved. This is another conservation law. Note that in both the conservation laws, the mass is taken to be constant.\\

Now, suppose that the particle moves from position $i$, for initial, to position $f$, for final, taking $d\vec{s}$ steps, and gets acted upon by a force $\vec{F}$ for time $dt$. The \eax{work} done by the force is defined as $W \defeq \int_{i}^{f} \vec{F} \cdot d\vec{s}$. If the mass is constant, then we have
\begin{equation}
    W = \int_{i}^{f} m \vec{a} \cdot d\vec{s} = m \int_{i}^{f} \frac{d\vec{v}}{dt} \cdot \vec{v} dt = \frac{m}{2} \int_{i}^{f} dt \frac{d}{dt}(\vec{v} \cdot \vec{v}) = \frac{m}{2} \left( v_{f}^{2} - v_{i}^{2} \right) = T_{f} - T_{i},
\end{equation}
where $T$ denotes the \eax{kinetic energy} of the particle. This is, roughly stated, the \eax{work-energy theorem}. Again, note that the mass is taken to be constant. If the exeternal work done $W$ is zero, then $T$ is conserved. This is our third conservation law. If the work done between two positions is independent of the path taken, then the force is termed a \eax{conservative force}. If the path starts and ends at the same point, then the work done is zero in case of a conservative force. Thus, $\oint_{\cC}\vec{F} \cdot d\vec{s} = 0$, where $\cC$ is a closed path. The opposite is also true; a necessary and sufficient condition for a force to be conservative is $\vec{F} = -\vec{\nabla} V$, where $V$ is a scalar field called the \eax{potential energy}. In this case,
\begin{equation}
    W = \oint_{\cC} \vec{F} \cdot d\vec{s} = -\oint_{\cC} \vec{\nabla} V \cdot d\vec{s} = -\oint_{\cC} dV = 0.
\end{equation}
By the work-energy theorem, for an open path, we have
\begin{equation}
    \int_{i}^{f} \vec{F} \cdot d\vec{s} = -\int_{i}^{f} \vec{\nabla} V \cdot d\vec{s} = -(V_{f}-V_{i}) = T_{f} - T_{i} \implies T_{f} + V_{f} = T_{i} + V_{i}.
\end{equation}
Thus, the quantity $T+V$ is always conserved in a conservative field; this quantity is termed the \eax{total energy}, $E = T+V$.
\subsubsection{Galilean Transformation}
\textit{July 23rd.}

Suppose one observer sits at a frame of reference $O$ and another obsever sits at a frame of reference $O'$, which is moving away from $O$ with a constant velocity $\vec{v}$. In this case, the positions of the two observers can be related as $\vec{r'} = \vec{r} - \vec{v}t$. However, this is based on the assumption that $t' = t$; time is experienced the same by both the observers. Newton made this assumption without any justification and it seemed right at the time. It is still important to note that Newton's form of $\vec{F} = m \frac{d^{2} \vec{r}}{dt^{2}}$ holds true in both cases--- it is independent of the frame of reference. Unless the velocity $\vec{v}$ is much, much higher (comparable to the speed of light), we can make confidently make the assumption $t = t'$ and work with Newton's equations. A transformation of coordinates between two frames of reference which differ only by constant relative motion is termed a \eax{Galilean transformation}.\\

All in all, \textit{this} is where mechanics ends of a single particle mass. Note that we have not introduced the notion of internal or external forces. For much more complicated bodies made up of many more particles, more theory has to be introduced. We now discuss a system of particles.

\subsection{The Laws: System of Particles}
The first step is to choose a frame of reference and a coordinate system, and stick to it. Every particle is then assigned a position in time $\vec{r}_{i}$, a velocity $\dot{\vec{r}}_{i}$, and its acceleration $\ddot{\vec{r}}_{i}$. Along with this, each particle experiences an external force $\vec{F}_{i}^{\text{ext}}$ and forces by the other particles $\vec{F}_{ji}$ for $j \neq i$; we make the assumption that no particle exerts a force on itself. By Newton's second law, we obtain
\begin{equation}
    \vec{F}_{i}^{\text{tot}} = \vec{F}_{i}^{\text{ext}} + \sum_{j \neq i} \vec{F}_{ji} = \frac{d\vec{p}_{i}}{dt}.
\end{equation}
If $N_{p}$ dentes the number of particles, and we assume that $m_{i}(t) \equiv m_{i}$, summing over all particles leads to
\begin{equation}
    \sum_{i=1}^{N_{p}} \sum_{j=1,j\neq i}^{N_{p}} \vec{F}_{ji} + \sum_{i=1}^{N_{p}} \vec{F}_{i}^{\text{ext}} = \sum_{i=1}^{N_{p}} \frac{d}{dt} (m_{i}\vec{v}_{i}) = \frac{d^{2}}{dt^{2}} \left( \sum_{i=1}^{N_{p}} m_{i}\vec{r}_{i} \right).
\end{equation}
We invoke the weak form of Newton's third law, which states $\vec{F}_{ji} + \vec{F}_{ij} = 0$ for all pairs $i \neq j$. With this, it is easy to see that the first summation term of the above equations drops out. Let us denote the total mass of the system by $M_{\text{tot}}$. Thus, we then have
\begin{align}
    \implies \sum_{i=1}^{N_{p}} \vec{F}_{i}^{\text{ext}} &= \frac{d^{2}}{dt^{2}} \left( \frac{\sum_{i=1}^{N_{p}} m_{i}\vec{r}_{i}}{\sum_{i=1}^{N_{p}} m_{i}} \cdot M_{\text{tot}} \right) \\
    \implies \vec{F}_{\text{tot}}^{\text{ext}} &= M_{\text{tot}} \cdot \frac{d^{2}}{dt^{2}} \left( \frac{\sum_{i=1}^{N_{p}} m_{i}\vec{r}_{i}}{\sum_{i=1}^{N_{p}} m_{i}} \right) = M_{\text{tot}} \ddot{\vec{R}}_{CM}
\end{align}
where the quantity $\vec{R}_{CM}$ is termed the \eax{center of mass} of the system. For the total momentum of the system, we work roughly the same---
\begin{align}
    \vec{P}_{\text{tot}} = \sum_{i=1}^{N_{p}} \vec{p}_{i} = M_{\text{tot}} \cdot \frac{d}{dt} \left( \frac{\sum_{i=1}^{N_{p}}m_{i}\vec{r}_{i}}{M_{\text{tot}}} \right) = M_{\text{tot}} \dot{\vec{R}}_{CM}.
\end{align}
Differentiating the above with respect to time gives us Newton's second law of motion for a system of particles.
\begin{align}
    \frac{d\vec{P}_{\text{tot}}}{dt} = M_{\text{tot}} \frac{d^{2}\vec{R}_{CM}}{dt^{2}} = \vec{F}_{\text{tot}}^{\text{ext}}.
\end{align}
Thus, $\vec{P}_{\text{tot}}$ is conserved if the total external force is zero. This is our conservation of momentum law for a system of particles. Similarly, we can derive the law of conservation of angular momentum by defining $\vec{L}_{i} = \vec{r}_{i} \times \vec{p}_{i}$, and $\vec{L}_{\text{tot}} = \sum_{i=1}^{N_{p}} \vec{L}_{i}$. Noting that $\frac{d\vec{r}}{dt} \times \vec{p} = \vec{v} \times m\vec{v} = 0$, we get
\begin{align}
    \frac{d\vec{L}_{\text{tot}}}{dt} &= \frac{d}{dt} \left( \sum_{i=1}^{N_{p}} \vec{r}_{i} \times \vec{p}_{i} \right) = \sum_{i=1}^{N_{p}} \vec{r}_{i} \times \frac{d\vec{p}_{i}}{dt} = \sum_{i=1}^{N_{p}} \vec{r}_{i} \times \vec{F}_{i}^{\text{ext}} + \sum_{i,j=1, j \neq i}^{N_{p}} \vec{r}_{i} \times \vec{F}_{ji} \\
    \implies \frac{d\vec{L}_{\text{tot}}}{dt} &= \sum_{i=1}^{N_{p}} \vec{r}_{i} \times \vec{F}_{i}^{\text{ext}} = \sum_{i=1}^{N_{p}} = \vec{\tau}_{i}^{\text{ext}} = \vec{\tau}_{\text{tot}}^{\text{ext}}.
\end{align}
In the above, we have made use of the strong form of Newton's third law, which states that the opposite and equal forces also act on the same line joining them. If the total external torque vanishes, then the total angular momentum is conserved.\\ \\
\textit{July 28th.}

For the $i^{\text{th}}$ particle, its position vector and the positive vector of the center of mass are related as $\vec{r}_{i} = \vec{R}_{CM} + \vec{r}'_{i}$, where $\vec{r}'_{i}$ is the relative position between them. Taking the time derivative gives us $\vec{v}_{i} = \vec{v}_{CM} + \vec{v}'_{i}$, giving the instantaneous relative velocity between them. Using this, we rewrite as
\begin{align}
    \vec{L}_{\text{tot}} = \sum_{i=1}^{N_{p}} m_{i}(\vec{r}'_{i} + \vec{R}_{CM}) \times (\vec{v}'_{i} + \vec{v}_{CM}).
\end{align}
Looking at the center of mass terms ($\vec{R}_{CM}$ and $\vec{v}_{CM}$), we have
\begin{align}
    T_{1} = \left( \sum_{i=1}^{N_{p}} m_{i} \right) \vec{R}_{CM} \times \vec{v}_{CM} = \vec{R}_{CM} \times (M_{\text{tot}}\vec{v}_{CM}) = \vec{R}_{CM \times \vec{P}_{\text{tot}}} = \vec{L}_{CM,\text{O}},
\end{align}
where $\vec{L}_{CM,\text{O}}$ denotes the angular momentum of the center of mass about the chosen origin. Looking at the relative terms $(\vec{r}'_{i}$ and $\vec{v}'_{i}$), we have
\begin{align}
    T_{2} = \sum_{i=1}^{N_{p}} m_{i} \vec{r}'_{i} \times \vec{v}'_{i} = \sum_{i=1}^{N_{p}} \vec{r}'_{i} \times \vec{p}'_{i} = \sum_{i=1}^{N_{p}} \vec{L}_{i,CM},
\end{align}
where $\vec{L}_{i,CM}$ denotes the angular momentum of the $i^{\text{th}}$ particle about the center of mass. Looking at the terms $\vec{r}'_{i}$ and $\vec{v}_{CM}$ leads to an evaluation of zero. Similarly, looking at the terms $\vec{R}_{CM}$ and $\vec{v}'_{i}$ also results to an evaluation of zero. Thus, with all terms taken care of, we have
\begin{align}
    \vec{L}_{\text{tot}} = \vec{L}_{CM,\text{O}} + \sum_{i=1}^{N_{p}} \vec{L}_{i,CM}.
\end{align}

Recall that $W_{if} = T_{f} - T_{i}$ holds regardless of the nature of the force. If the force is conservative, then potential is introduced and $T_{f} + V_{f} = T_{i} + V_{i}$ holds true even when $W_{if}$ is non-zero. We translate this for a system of particles.
\begin{align}
    W_{i \mapsto f} = \int_{i}^{f} \sum_{i=1}^{N_{p}} \vec{F}_{i} \cdot d\vec{S}_{i},
\end{align}
where $\vec{F}_{i} = \vec{F}_{i}^{\text{ext}} + \sum_{j=1, j \neq i}^{N_{p}} \vec{F}_{ji}$ (\textit{Note that the $i$ in the integral bound represents the initial state and has no relation to the indexing $i$.}). Thus, we have
\begin{align}
    W_{i \mapsto f} = \int_{i}^{f} \sum_{i=1}^{N_{p}} \vec{F}_{i}^{\text{ext}} d\vec{S}_{i} + \int_{i}^{f} \sum_{i=1}^{N_{p}} \left( \sum_{j=1, i \neq j}^{N_{p}} \vec{F}_{ji} \cdot d\vec{S}_{i} \right).
\end{align}
Let us focus on the first term for now.
\begin{align}
    \int_{i}^{f} \sum_{i=1}^{N_{p}} \vec{F}_{i}^{\text{ext}} d\vec{S}_{i} &= \int_{i}^{f} \sum_{i=1}^{N_{p}} m_{i} \dot{\vec{v}}_{i} \cdot \vec{v}_{i} dt = \sum_{i=1}^{N_{p}} \int_{i}^{f} d \left( \frac{1}{2} m_{i}v_{i}^{2} \right) = \sum_{i=1}^{N_{p}} \int_{i}^{f} dT_{i}.
\end{align}
If we define $T_{\text{tot}} = \sum_{i=1}^{N_{p}} T_{i}$, we get
\begin{align}
    W_{i \mapsto f} = T_{\text{tot},f} - T_{\text{tot},i}.
\end{align}
This equation is the work-energy theorem for a system of particles. Before dealing with the second term, let us try to alter the kinetic energies to relate to the center of mass instead. We have
\begin{align}
    T_{\text{tot}} = \frac{1}{2} \sum_{i=1}^{N_{p}} m_{i}(\vec{v}_{CM} + \vec{v}'_{i}) \cdot (\vec{v}_{CM} + \vec{v}'_{i}) = \frac{1}{2}M_{\text{tot}}\abs{\vec{v}_{CM}}^{2} + \frac{1}{2} \sum_{i=1}^{M_{p}} m_{i} \abs{\vec{v}'_{i}}^{2}.
\end{align}
We now make our first assumption: \textit{the external forces are conservative}. Thus,
\begin{align}
    \int_{i}^{f} \sum_{i=1}^{N_{p}} \vec{F}_{i}^{\text{ext}} \cdot d\vec{S}_{i} = - \int_{i}^{f} \vec{\nabla}_{i} V_{i}^{\text{ext}} \cdot d\vec{S}_{i} = \sum_{i=1}^{N_{p}} \left( - \int_{i}^{f} \vec{\nabla}_{i} V_{i}^{\text{ext}} \cdot d\vec{S}_{i} \right)
\end{align}
Our next assumption is that the force $\vec{F}_{ij}$, for all $i \neq j$, is conservative. This implies that for all pairs $i \neq j$, there exists $V_{ij}$ such that $\vec{F}_{ij} = -\vec{\nabla}_{j}V_{ij}$. Another assumption we make is the fact that $\vec{F}_{ij}$ is central; that the potential $V_{ij}$ between the $i^{\text{th}}$ and $j^{\text{th}}$ particles must only depend on the absolute distance between them. That is, $V_{ij} = V_{ij}(\abs{\vec{r}_{i}-\vec{r}_{j}}) = V_{ji}(\abs{\vec{r}_{i}-\vec{r}_{j}})$. Thus, from both these assumptions, we get $\vec{F}_{ij} = -\vec{\nabla}_{j}V_{ij} \Leftrightarrow \vec{F}_{ji} = -\vec{\nabla}_{i}V_{ij}$. This then shows that $\vec{F}_{ji} = -\vec{F}_{ij}$, so our assumptions are valid. Also, it can be shown that the gradient can be rewritten as
\begin{align}
    \vec{\nabla}_{j} V_{ij}(\abs{\vec{r}_{i}-\vec{r}_{j}}) = (\vec{r}_{i}-\vec{r}_{j})f(\vec{r}_{ij}).
\end{align}
Let $\vec{r} = \vec{r}_{i} = \vec{r}_{j}$. Then, the $j^{\text{th}}$ component is
\begin{align}
    \left[\vec{\nabla}_{j} V_{ij}(\abs{\vec{r}})\right]_{j} = \frac{\partial V_{ij}}{\partial r} \left[ \vec{\nabla}_{j} r \right]_{j} = \frac{\partial}{\partial x_{j}} \left( (x_{i}-x_{j})^{2} + (y_{i}-y_{j})^{2} + (z_{i}-z_{j})^{2} \right)^{1/2} = \frac{-(x_{i}-x_{j})}{r}.
\end{align}
Therefore,
\begin{align}
    \vec{\nabla}_{j} V_{ij}(\abs{\vec{r}}) = \left( -\frac{1}{r}\frac{\partial V_{ij}}{\partial r} \right) \left( (x_{i}-x_{j})\hat{e}_{x} + (y_{i}-y_{j})\hat{e}_{y} + (z_{i}-z_{j})\hat{e}_{z} \right) = f(\vec{r}_{ij}) (\vec{r}_{i}-\vec{r}_{j})
\end{align}
as desired. Coming back to the second term of the work equation, we now have
\begin{align}
    \sum_{i,j=1;i \neq j}^{N_{p}} \int_{i}^{f} \vec{F}_{ji} \cdot d\vec{S}_{i} = \sum_{1 \leq i < j \leq N_{p}} \int_{i}^{f} \left( F_{ij} \cdot d\vec{S}_{j} + \vec{F}_{ji} \cdot d\vec{S}_{i} \right).
\end{align}
Focusing on just the integral for now gives us
\begin{align}
    \int_{i}^{f} \left( F_{ij} \cdot d\vec{S}_{j} + \vec{F}_{ji} \cdot d\vec{S}_{i} \right) = -\int_{i}^{f} \left( (\vec{\nabla}_{j}V_{ij}) \cdot d\vec{S}_{j} + (\vec{\nabla}_{i} V_{ij}) \cdot d\vec{S}_{i} \right) = -\int_{i}^{f} \vec{\nabla}_{ij} V_{ij} \cdot d\vec{S}_{ij}
\end{align}
where $d\vec{S}_{ij} = d\vec{S}_{i} - d\vec{S}_{j}$. Thus, the second term is finally
\begin{align}
    -\sum_{i,j=1;i \neq j}^{N_{p}} \frac{1}{2} \int_{i}^{f} \vec{\nabla}_{ij} V_{ij} \cdot d\vec{S}_{ij}
\end{align}
which gives us
\begin{align}
    T_{\text{tot},f} - T_{\text{tot},i} = -\sum_{i=1}^{N_{p}} \left( V_{i,\text{final}}^{\text{ext}} - V_{i,\text{init}}^{\text{ext}} \right) + (-\frac{1}{2}) \sum_{i,j=1;i \neq j}^{N_{p}} (V_{ij}^{\text{final}}-V_{ij}^{\text{init}}) = -(V_{\text{tot}},\text{init}^{\text{sys}} - V_{\text{tot},\text{final}}^{\text{sys}}).
\end{align}
We conclude that the total energy, $E_{\text{sys}}^{\text{tot}} = T_{\text{tot}}^{\text{sys}} + V_{\text{tot}}^{\text{sys}}$ is conserved.

\section{The Laws: Rigid Bodies}
\textit{July 30th.}

We now extensively study the mechanics of rigid bodies. By a \eax{rigid body}, we mean one where the distance between any pair of particles remains constant; for all $i,j$, we have
\begin{align}
    \abs{\vec{r}_{i}(t)-\vec{r}_{j}(t)}^{2} = c_{ij}^{2}.
\end{align}
Now comes the question of the coordinate system. For a rigid body, the cartesian coordinate system may not always be the nicest to work with. For example, suppose we have a particle always travelling on the surface of a sphere of radius $R$. Then its coordinates are always related as $x^{2}(t)+y^{2}(t)+z^{2}(t) = R^{2}$. To come up with a more suitable coordinate system, we look at 2 dimensions first.


\subsection{Coordinate Systems}
If a particle is at $(x,y)$, then we can rewrite its coordinates as $(r,\phi)$, where $r$ is the distance from the origin, and $\phi$ is the counter-clockwise angle made with the positive $x$-axis. Here, they are related as $x = r \cos \phi$ and $y = r \sin \phi$. If the particle moves around in a circle, then choosing the latter coordinate system, known as the \eax{polar coordinate system}, proves to be useful since $\dot{r}(t) = 0$ and only the analysis on $\phi$ is to be done.

Extending this idea into three dimensions, we first assign a new unit vector $\hat{e}_{r}$ that points in the same direction as $\vec{r}$. $\theta$ is an angle measured from the positive $z$-axis, and takes the values $[0,\pi]$, with $\theta = 0$ at the north pole and $\theta = \pi$ at the south pole. The unit vector associated with $\theta$ is $\hat{e}_{\theta}$ which points tangentially along the longitude from the north to the south. Next, $\phi$ is termed the angle the positive $x$-axis makes with the projection of $\vec{r}$ onto the $xy$-plane in a counter-clockwise manner. Thus, $\phi \in [0,2\pi]$. The unit vector associated with $\phi$ is $\vec{e}_{\phi}$ in such a way that $\hat{e}_{r} \times \hat{e}_{\theta} = \hat{e}_{\phi}$. This is the \eax{spherical coordinate system}. It is important to note that the unit vectors are \textit{not} fixed and are, in fact, functions of $r$, $\theta$, and $\phi$.

\begin{remark}
    The cartesian coordinates and the spherical coordinates are related as
    \begin{align}
        x(t) &= r(t) \sin\theta(t) \cos\phi(t),\\
        y(t) &= r(t) \sin\theta(t) \sin\phi(t),\\
        z(t) &= r(t) \cos\theta(t).
    \end{align}
    If $r(t) = R$ is fixed, then $x^{2}+y^{2}+z^{2} = R^{2}$ and, thus,
    \begin{align}
        \dot{x}(t) &= R \dot{\theta}(t) \cos\theta(t) \cos\phi(t) - R\dot{\phi}(t) \sin\theta(t) \sin\phi(t),\\
        \dot{y}(t) &= R \dot{\theta}(t) \cos\theta(t) \sin\phi(t) + R\dot{\phi}(t) \sin\theta(t) \cos\phi(t),\\
        \dot{z}(t) &= -R\dot{\theta}(t) \sin\theta(t).
    \end{align}
\end{remark}

Our next goal is to derive expressions for gradient, divergence, and curl in the coordinates $(r,\theta,\phi)$. For any function $f$ of $(x,y,z)$, it can also be a function $f$ of $(r,\theta,\phi)$. One might be tempted to think that the gradient, or divergence or curl, can simply be found by replacing $x,y,z,\hat{e}_{x},\hat{e}_{y},\hat{e}_{z}$ with $r,\theta,\phi,\hat{e}_{r},\hat{e}_{\theta},\hat{e}_{\phi}$ in the equation(s), but then one has the problem of dimensions. Figuring out the exact equations is left as an exercise the reader.\\

In the \eax{cyclindrical coordinate system}, one has $(\rho,\phi,z)$, where $\phi$ and $z$ retain their usual meaning from the spherical and cartesian coordinates respectively, and $\rho$ is the distance of the coordinate from the $z$-axis. In this case, the unit vector associated with $\rho$ satisfies $\hat{e}_{\rho} \times \hat{e}_{\phi} = \hat{e}_{z}$. The reader is urged to formulate the relationships between the three coordinate systems.

\subsection{Generalized Coordinates}

Coming back to the rigid body mechanics, the inter-particle distance equation can be rewritten as
\begin{align}
    (x-x_{C}(t))^{2} + (y-y_{C}(t))^{2} + (z-z_{C}(t))^{2} - R^{2} = 0.
\end{align}

Such an equation which can be written in the form
\begin{align}
    f(\vec{r}_{1},\vec{r}_{2},\ldots,\vec{r}_{N_{p}},t) = 0
\end{align}
is termed a \eax{holonomic constraint}. These are again of two types; a mechanical system is \eax{rheonomous} if its equations of constraints contain time as an explicit variable. These are hard to deal with. On the other hand, a mechanical system is \eax{scleronomous} if the equations of constraints do not contain the time as an explicit variable.

We now discuss the notion of \eax{generalized coordinates}, starting with $N_{p}$ particles in $d$ dimensions. Thus, we will have $dN_{p}$ independent coordinates to deal with. For now, we deal with $d = 3$. Suppose we also have $k$ equations of constraints which are holonomic in nature. Combining, we than have
\begin{align}
    (3N_{p}-k) \text{ independent `generalized' coordinates}
\end{align}
which are termed $\{q_{j}\}$.

\subsubsection{Equilibrium}
\textit{August 4th.}

The case of \eax{equilibrium} simply means $\vec{F}_{i} = 0$ for $i = 1,2,\ldots,N_{p}$. Trivially, the vector sum $\sum_{i=1}^{N_{p}} \vec{F}_{i}$ is also zero. More importantly, $\vec{F}_{i} \cdot \delta \vec{r}_{i} = 0$ and no work is being done by the system. Here, $\delta \vec{r}_{i}$ is the \eax{virtual displacement} of each particle and are not independent of each other. The virtual displacements are consistent with the constraints of the system.
\begin{align}
    \sum_{i=1}^{N_{p}} (\vec{F}_{i}^{(a)} + \vec{f}_{i}) \cdot \delta \vec{r}_{i} = 0
\end{align}
where $\vec{F}_{i}^{(a)}$ is the applied force, and $\vec{f}_{i}$ is the constraint force. We now make our first assumption: that the constraint forces do no work under virtual displacements. That is, $\vec{f}_{i} \cdot \delta \vec{r}_{i} = 0$. Of course, this implies that
\begin{align}
    \sum_{i=1}^{N_{p}} \vec{F}_{i}^{(a)} \cdot \delta \vec{r}_{i} = 0
\end{align}
This is the \eax{principle of virtual work}: the total virtual work done by the applied forces vanishes. Note that this does not imply that each applied force is zero. We now extend this to the case of dynamics. In this case, Newton's second law gives us $\vec{F}_{i} = \dot{\vec{p}}_{i}$, where $\vec{p}_{i}$ is the linear momentum of the $i$th particle. Then we have:
\begin{align}
    \vec{F}_{i} - \dot{\vec{p}}_{i} = 0 \Rightarrow \sum_{i=1}^{N_{p}} (\vec{F}_{i}^{(a)} + \vec{f}_{i} - \dot{\vec{p}}_{i}) \cdot \delta \vec{r}_{i} = 0
\end{align}
Again invoking the assumption that the constraint forces do no virtual work, we find:
\begin{align}
    \sum_{i=1}^{N_{p}} (\vec{F}_{i}^{(a)} - \dot{\vec{p}}_{i}) \cdot \delta \vec{r}_{i} = 0
\end{align}
This is known as \eax{d'Alembert's principle}. It states that the difference between the applied forces and the time derivative of the momentum (i.e., the inertial forces) does no virtual work.\\

Suppose $\vec{r}_{i} = \vec{r}_{i}(q_{1},\ldots,q_{n},t)$ for all $i = 1,2,\ldots,N_{p}$, with $n$ generalized coordinates. We transfer many definitions to the idea of generalized coordinates

The first step is the definition of the velocities. We have
\begin{align}
    \vec{v}_{i} = \frac{d\vec{r}_{i}}{dt} = \sum_{j=1}^{n} \frac{\partial \vec{r}_{i}}{\partial q_{j}} \frac{d q_{j}}{dt} + \frac{\partial \vec{r}_{i}}{\partial t}
\end{align}
The virtual displacement is again a function of the generalized coordinates, so we can write this as
\begin{align}
    \delta \vec{r}_{i} = \sum_{j=1}^{n} \frac{\partial \vec{r}_{i}}{\partial q_{j}} \delta q_{j}.
\end{align}
Moving forward,
\begin{align}
    \sum_{i=1}^{N_{p}} \vec{F}_{i}^{(a)} \cdot \delta \vec{r}_{i} = \sum_{i=1}^{N_{p}} \vec{F}_{i}^{(a)} \cdot \left( \sum_{j=1}^{n} \frac{\partial \vec{r}_{i}}{\partial q_{j}} \delta q_{j} \right) = \sum_{j=1}^{n} \left( \sum_{i=1}^{N_{p}} \vec{F}_{i}^{(a)} \cdot \frac{\partial \vec{r}_{i}}{\partial q_{j}}\right) \delta q_{j} = \sum_{j=1}^{N} Q_{j} \delta q_{j}
\end{align}
where $Q_{j}$ is the scalar $\sum_{i=1}^{N_{p}} \vec{F}_{i}^{(a)} \cdot \frac{\partial \vec{r}_{i}}{\partial q_{j}}$. Next, using Newtons' law,
\begin{align}
    \dot{\vec{p}}_{i} \cdot \delta\vec{r}_{i} = m_{i}\ddot{\vec{r}}_{i} \cdot \delta\vec{r}_{i} \Rightarrow \sum_{i=1}^{N_{p}} m_{i} \ddot{\vec{r}_{i}} \cdot \delta\vec{r}_{i} = \sum_{j=1}^{n} \left( \sum_{i=1}^{N_{p}} m_{i} \ddot{\vec{r}}_{i} \cdot \frac{\partial \vec{r}_{i}}{\partial q_{j}} \right) \delta q_{j}.
\end{align}
A bit of algebra shows us
\begin{align}
    m_{i} \ddot{\vec{r}}_{i} \cdot \frac{\partial \vec{r}_{i}}{\partial q_{j}} = \frac{d}{dt} \left( m_{i} \dot{\vec{r}}_{i} \cdot \frac{\partial \vec{r}_{i}}{\partial q_{j}} \right) - m_{i} \dot{\vec{r}}_{i} \cdot \frac{d}{dt} \left( \frac{\partial \vec{r}_{i}}{\partial q_{j}} \right)
\end{align}
and
\begin{align}
    \frac{d}{dt} \left( \frac{\partial \vec{r}_{i}}{\partial q_{j}} \right) = \frac{\partial}{\partial q_{j}} \left( \frac{d\vec{r}_{i}}{dt} \right) = \frac{\partial \vec{v}_{i}}{\partial q_{j}}.
\end{align}
Denoting the time derivative of $q_{j}$ as $\dot{q_{j}}$, we have
\begin{align}
    \frac{\partial \vec{v}_{i}}{\partial \dot{q}_{j}} = \frac{\partial}{\partial \dot{q}_{j}} \left( \frac{d \vec{r}_{i}}{dt} \right) = \frac{\partial}{\partial \dot{q}_{j}} \left( \sum_{k=1}^{n} \frac{\partial \vec{r}_{i}}{\partial q_{k}} \dot{q}_{k} + \frac{\partial \vec{r}_{i}}{\partial t} \right) = \frac{\partial \vec{r}_{i}}{\partial q_{j}}.
\end{align}
Thus,
\begin{align}
    \frac{d}{dt} \left( m_{i} \ddot{\vec{r}}_{i} \cdot \frac{\partial \vec{r}_{i}}{\partial q_{j}} \right) = \frac{d}{dt} \left( m_{i} \vec{v}_{i} \cdot \frac{\partial \vec{v}_{i}}{\partial \dot{q}_{j}}\right) \text{ and } m_{i} \dot{\vec{r}}_{i} \cdot \frac{d}{dt} \left( \frac{\partial \vec{r}_{i}}{\partial q_{j}} \right) = m_{i} \vec{v}_{i} \cdot \frac{\partial \vec{v}_{i}}{\partial q_{j}}.
\end{align}
Note that both these terms are distinct and are summed over $i$ and $j$. With $T = (\sum_{i=1}^{N_{p}} m_{i}v_{i}^{2})/2$, we turn to
\begin{align}
    \sum_{i=1}^{N_{p}} \dot{\vec{p}}_{i} \cdot \delta \vec{r}_{i} &= \sum_{i=1}^{N_{p}} \left( \sum_{j=1}^{N_{p}} \left( \frac{d}{dt} \left( m_{i} \vec{v}_{i} \cdot \frac{\partial \vec{v}_{i}}{\partial \dot{q}_{j}} \right) - m_{i}\vec{v}_{i} \cdot \frac{\partial \vec{v}_{i}}{\partial q_{j}} \right) \delta q_{j} \right) \\
    &= \sum_{j=1}^{n} \left( \frac{d}{dt} \left( \frac{\partial T}{\partial \dot{q}_{j}} \right) - \frac{\partial T}{\partial q_{j}} \right) \delta q_{j}.
\end{align}
This is the term that appears from the above. Remember that there was also a term $\vec{F}_{i}^{(a)} \cdot \delta \vec{r}_{i} = \sum_{j=1}^{n} Q_{j} \delta q_{j}$. Therefore,
\begin{align}
    \sum_{i=1}^{N_{p}} (\vec{F}_{i}^{(a)} - \dot{\vec{p}}_{i}) \cdot \delta \vec{r}_{i} = 0 \implies \sum_{j=1}^{n} \left( \frac{d}{dt} \left( \frac{\partial T}{\partial \dot{q}_{j}} \right) - \frac{\partial T}{\partial q_{j}} -Q_{j}\right) \delta q_{j} = 0.
\end{align}
What we have essentially done is transform the equations depending on the non-independent virtual displacements into a linear sum of $\delta q_{j}$'s which are independent of each other. We get the desired result
\begin{align}
    \frac{d}{dt} \left( \frac{\partial T}{\partial \dot{q}_{j}} \right) - \frac{\partial T}{\partial q_{j}} - Q_{j} = 0 \text{ for all } j = 1,2,\ldots,n.
\end{align}

Let us use the above result to derive the equations for three dimensional motion. Here, $q_{1} = x$, $q_{2} = y$, and $q_{3} = z$. Also, $\frac{\partial T}{\partial q_{j}} = 0$ for all $j$, where $T = \frac{1}{2}m(\dot{x}^{2}+\dot{y}^{2}+\dot{z}^{2})$. Moreover,
\begin{align}
    \frac{\partial T}{\partial \dot{q}_{1}} = \frac{\partial T}{\partial \dot{x}} = m\dot{x} \text{ and } \frac{d}{dt} \left( \frac{\partial T}{\partial \dot{q}_{1}} \right) = \frac{d}{dt}(m\dot{x}) = m\ddot{x} = 0 
\end{align}
which matches the standard laws of motions. For $n$ particles, we have
\begin{align}
    \frac{d}{dt} \left( \frac{\partial T}{\partial q_{j}} \right) - \frac{\partial T}{\partial q_{j}} = Q_{j} \text{ for all } j = 1,2,\ldots,n.
\end{align}
Since $\vec{F}_{i} = -\vec{\nabla}_{i} V$,
\begin{align}
    Q_{j} = \sum_{i=1}^{N_{p}} \vec{F}_{i} \cdot \frac{\partial \vec{r}_{i}}{\partial q_{j}} = -\sum_{i=1}^{N_{p}} \vec{\nabla}_{i} V \cdot \frac{\partial \vec{r}_{i}}{\partial q_{j}} = -\frac{\partial V}{\partial q_{j}}
\end{align}
$V$ generally does not depend on $\dot{q}_{j}$. So,
\begin{align}
    \frac{d}{dt} \left( \frac{\partial T}{\partial \dot{q}_{j}} \right) - \frac{\partial(T-V)}{\partial q_{j}} = 0 \implies \frac{d}{dt} \left( \frac{\partial(T-V)}{\partial \dot{q}_{j}} \right) - \frac{\partial(T-V)}{\partial q_{j}} = 0.
\end{align}
This aove equation holds true for any $j$. The quantity $T-V$ is termed the \eax{Lagrangian} denoted as $L = L(\{q_{j}\},\{\dot{q}_{j}\},t)$.

\begin{example}
    Let us look at a particle moving in 2 dimensions via polar coordinates. The force acting on it is $\vec{F} = F_{x}\hat{e}_{x} + F_{y}\hat{e}_{y}$. Here, $q_{1} = r$ and $q_{2} = \theta$. The kinetic energy is $\frac{1}{2}m(\dot{x}^{2}+\dot{y}^{2})$. The coordinates are changed to polar as $x = r\cos\theta = x(r,\theta)$ and $y = r \sin \theta = y(r,\theta)$, giving us
    \begin{align}
        \dot{x} = \dot{r} \cos \theta - r \dot{\theta} \sin \theta \text{ and } \dot{y} = \dot{r} \sin \theta + r \dot{\theta} \cos \theta.
    \end{align}
    The kinetic energy then becomes
    \begin{align}
        T &= \frac{1}{2} m(\dot{x}^{2}+\dot{y}^{2}) = \frac{1}{2}m\dot{r}^{2} + \frac{1}{2} mr^{2}\dot{\theta}^{2}
    \end{align}
    giving us
    \begin{align}
        \frac{\partial T}{\partial \dot{r}} = m\dot{r} \implies \frac{d}{dt} \left( \frac{\partial T}{\partial \dot{r}} \right) = \frac{d}{dt} (m\dot{r}) = m\ddot{r}.
    \end{align}
    Along with $\frac{\partial U}{\partial \dot{r}} = 0$, we get
    \begin{align}
        \frac{d}{dt} \left( \frac{\partial L}{\partial \dot{r}} \right) = m\ddot{r}.
    \end{align}
    Moreover, $\frac{\partial T}{\partial r} = mr \dot{\theta}^{2}$ and $\frac{\partial U}{\partial r} = -F_{r}$. Thus, we have
    \begin{align}
        \frac{d}{dt}\left( \frac{\partial L}{\partial \dot{r}} \right) - \frac{\partial L}{\partial r} = 0 \implies m\ddot{r} - mr\dot{\theta}^{2} - F_{r} = 0.
    \end{align}
    Moving on to $q_{2} = \theta$, we first have
    \begin{align}
        \frac{\partial T}{\partial \dot{\theta}} = mr^{2}\dot{\theta} \implies \frac{d}{dt} \left( \frac{\partial T}{\partial \dot{\theta}} \right) = 2mr\dot{r}\dot{\theta} + mr^{2}\ddot{\theta} = \frac{d}{dt} \left( \frac{\partial L}{\partial \dot{\theta}} \right)
    \end{align}
    since $\frac{\partial U}{\partial \dot{\theta}} = 0$. Moreover, since $\frac{\partial T}{\partial \theta} = 0$, we get $\frac{\partial L}{\partial \theta} = -\frac{\partial U}{\partial \theta} = rF_{\theta} = r \vec{F} \cdot \hat{e}_{\theta}$. Thus, we get
    \begin{align}
        \frac{d}{dt}\left( \frac{\partial L}{\partial \dot{\theta}} \right) - \frac{\partial L}{\partial \theta} = 0 \implies 2mr\dot{r}\dot{\theta} + mr^{2}\ddot{\theta} = rF_{\theta}.
    \end{align}
\end{example}

The final differential equations we derived are known as the \eax{Euler-Lagrangian equations of motion}.
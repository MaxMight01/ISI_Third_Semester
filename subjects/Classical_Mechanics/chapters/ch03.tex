\chapter{CENTRAL FORCE MOTION AND ROTATIONAL PHYSICS}

\section{The Central Force Problem}
\subsection{One Body Problem}
\textit{September 15th.}

Consider two particles of masses lying $m_{1}$ and $m_{2}$ lying at positions $\vec{r}_{1}$ and $\vec{r}_{2}$ respectively. If $\vec{r} = \vec{r}_{2}-\vec{r}_{1}$ is the relative position vector, and $\vec{R}$ is the position vector of the centre of mass, then the kinetic energy may be given as
\begin{align}
    T = \frac{1}{2}m_{1}\dot{\vec{r}}_{1}^{2} + \frac{1}{2}m_{2}\dot{\vec{r}}_{2}^{2} = \frac{1}{2}(m_{1}+m_{2})\dot{\vec{R}}^{2} + \frac{1}{2}\left( \frac{m_{1}m_{2}}{m_{1}+m_{2}} \right) \dot{\vec{r}}^{2}.
\end{align}
Here the quantity $\mu = \frac{m_{1}m_{2}}{m_{1}+m_{2}}$ is called the \eax{reduced mass}. If the potential energy of the system is $U(\vec{r})$, only dependent on the relative position, then the Lagrangian of the system may be written as
\begin{align}
    L = T-U = \frac{1}{2}M\dot{R}^{2} + \frac{1}{2}\mu \dot{r}^{2} - U(\vec{r})
\end{align}
Note that $R$ is a cyclic coordinate here. So
\begin{align}
    \frac{\partial L}{\partial \dot{\vec{R}}} = M\dot{\vec{R}} = \vec{P}_{CM} \implies \frac{d}{dt} \vec{P}_{CM} = 0.
\end{align}
The momentum of the center of mass is conserved. For the most popular of cases, we consider $U(\vec{r}) \equiv U(r)$, i.e., the potential energy depends only on the magnitude of the relative position vector. This is called a \eax{conservative central force}. If a particle of mass $m_{1}$ is fixed at the origin, then the other particle of mass $m_{2}$ moves in the potential $U(r)$ with reduced mass $\mu$. We shall focus our analysis only on the reduced mass since the center of mass positive is a cyclic coordinate. The Lagrangian of the system is
\begin{align}
    L = \frac{1}{2}\mu(\dot{r}^{2} + r^{2}\dot{\theta}^{2} + r^{2}\sin^{2}\theta \dot{\phi}^{2}) - U(r).
\end{align}
Since the problem is spherically symmetric, the angular momentum $\vec{L} = \vec{r} \times \mu \dot{\vec{r}}$ is a conserved quantity. We can infer that the motion is planar in central force (first property). Choose the axis of $\vec{L}$ to be the $z$-axis. Then $\theta = \frac{\pi}{2}$. The Lagrangian reduces to
\begin{align}
    L = T-U = \frac{1}{2}\mu(\dot{r}^{2} + r^{2}\dot{\phi}^{2}) - U(r).
\end{align}
Here, $\phi$ is a cyclic coordinate and hence the angular momentum $\ell = \mu r^{2}\dot{\phi}$ is conserved, giving us $\dot{\phi} = \frac{\ell}{\mu r^{2}}$. From this conservation of $\ell$, it also follows that $\frac{d}{dt}(\frac{1}{2}\mu r^{2}) = 0 \implies \frac{1}{2}r^{2}\dot{\phi}$ is constant. We can interpret this as $\frac{1}{2}r(r\dot{\phi}) = dA$, the infinitesimal area swept out by the radius vector in time $dt$. The \eax{areal velocity} $\frac{dA}{dt}$ is constant; equal area is swept out in equal time (second property). The energy function is
\begin{align}
    h = \frac{1}{2}\mu\dot{r}^{2} + \frac{\ell^{2}}{2\mu r^{2}} + U(r) \implies \frac{dh}{dt} = \mu\ddot{r} - \frac{\ell^{2}}{\mu r^{3}} + \frac{dU}{dr} = 0.
\end{align}
Rewriting gives us
\begin{align}
    \mu\ddot{r} = -\frac{d}{dr}\left( U + \frac{\ell^{2}}{2\mu r^{2}} \right) = -\frac{dU_{\text{eff}}(r)}{dr}
\end{align}
where the new quantity is termed the \eax{effective potential energy}. The quantity $T_{\text{eff}} = \frac{1}{2}\mu\dot{r}^{2}$ is termed the effective kinetic energy. The total energy is $h = T_{\text{eff}} + U_{\text{eff}}$. Summarising all the properties, the formal solution for the problem is
\begin{align}
    \ell = \mu r^{2}\dot{\phi},\; E=\frac{1}{2}\mu \dot{r}^{2} + \frac{\ell^{2}}{2\mu r^{2}} + U,\; \dot{r} = \sqrt{\frac{2}{\mu}\left( E-U-\frac{\ell^{2}}{2 \mu r^{2}} \right)}.
\end{align}
Thus, $r$ can be found as a function of $t$ by integrating
\begin{align}
    \int_{r_{0}}^{r} \frac{dr}{\sqrt{\frac{2}{\mu}\left( E-U-\frac{\ell^{2}}{2 \mu r^{2}} \right)}} = \int_{0}^{t} dt = t
\end{align}
and inverting the solution to get a function $r(t)$. The angle $\phi$ can be found as a function of $t$ by integrating
\begin{align}
    \frac{d\phi}{dt} = \frac{\ell}{\mu r^{2}} \implies \phi - \phi_{0} = \frac{\ell}{\mu}\int_{0}^{t} \frac{dt}{r^{2}(t)}.
\end{align}
The energy-distance diagrams using $U_{\text{eff}} = \frac{\ell^{2}}{2\mu r^{2}} + U(r)$ are very useful in visualising the motion of the particle. The points where $E = U_{\text{eff}}$ are the turning points of the motion. If $E < 0$, the motion is bounded, and if $E > 0$, the motion is unbounded. The point where $U_{\text{eff}}$ is minimum is a point of stable equilibrium. For a potential of the form $U(r) = -\frac{k}{r}$, this `orbit' of stable equilbrium can be found by setting $\frac{dU_{\text{eff}}}{dr} = 0$ and $\frac{d^{2}U_{\text{eff}}}{dr^{2}} > 0$. This gives us $r_{0} = \frac{\ell^{2}}{\mu k}$. But setting $E = U_{\text{eff}}(r_{0})$ gives us the effective force $f_{\text{eff}}$ to be zero, or $\ddot{r} = 0$. This means that the radius is constant, and the motion is circular.\\

\noindent\textit{September 17th.}

With $u = \frac{1}{r}$, and $\frac{du}{d\phi} = -\frac{1}{r^{2}} \frac{dr}{d\phi}$, the differential equation can be written as differential operators; using this, we shall rewrite our equation of motion.
\begin{align}
    \frac{d}{dt} = \frac{\ell}{\mu r^{2}} \frac{d}{d\phi} \implies \mu \ddot{r} = \mu \frac{d}{dt}\left( \frac{dr}{dt} \right) = \frac{\ell}{r^{2}} \frac{d}{d\phi}\left( \frac{\ell}{\mu r^{2}} \frac{dr}{d\phi} \right) = -\frac{\ell^{2}}{\mu} u^{2} \frac{d^{2}u}{d\phi^{2}}.
\end{align}
The second term is simply $\frac{\ell^{2}}{\mu r^{3}} = \frac{\ell^{2}}{\mu} u^{3}$. The final term is $\frac{dU}{dr} = \frac{dU}{du} \frac{du}{dr} = -u^{2} \frac{dU}{du}$. Thus, the equation of motion becomes
\begin{align}
    \frac{d^{2}u}{d\phi^{2}} + u = -\frac{\mu}{\ell^{2}} \frac{dU}{du}.
\end{align}
Thus, given $(u_{0},\phi_{0},\ell,E)$, we can solve for $u = u(\phi)$ and hence $r = r(\phi)$. Recall the relation between the differentials $dt$ and $dr$. Rewriting $dt$ in terms of $d\phi$, we have
\begin{align}
    d\phi = \frac{\ell dr}{\mu r^{2} \sqrt{\frac{2}{\mu} \left( E-U(r) - \frac{\ell^{2}}{2\mu r^{2}} \right)}}.
\end{align}
Integrating gives us
\begin{align}
    \phi - \phi_{0} = -\int_{u_{0}}^{u} \frac{du}{\sqrt{\frac{2\mu E}{\ell^{2}} - \frac{2\mu U(u)}{\ell^{2}} -u^{2}}}.
\end{align}

\subsection{The Kepler Problem}
We are simply considering $U = -\frac{k}{r} = -ku$ for some $k > 0$. Replacing this in the above equation gives
\begin{align}
    \phi - \phi_{0} = -\int_{u_{0}}^{u} \frac{du}{\sqrt{\frac{2\mu E}{\ell^{2}} + \frac{2\mu k}{\ell^{2}}u - u^{2}}}.
\end{align}
Recalling that
\begin{align}
    \int \frac{dx}{\sqrt{\alpha + \beta x + \gamma x^{2}}} = \frac{1}{\sqrt{-\gamma}} \cos^{-1} \left( -\frac{(\beta + 2\gamma x)}{\sqrt{\beta^{2}} - 4\alpha \gamma^{2}a} \right) + C,
\end{align}
we obtain
\begin{align}
    \phi - \phi_{0} = \cos^{-1}\left( \frac{\frac{\ell^{2} u}{\mu k} - 1}{\sqrt{1 + \frac{2 E \ell^{2}}{\mu k^{2}}}} \right).
\end{align}
Inverting the equation gives
\begin{align}
    \frac{1}{r} = u = \frac{\mu k}{\ell^{2}} \left( 1 + \sqrt{1 + \frac{2E \ell^{2}}{\mu k^{2}}} \cos(\phi - \phi_{0}) \right)
\end{align}
or just simply
\begin{align}
    \frac{1}{r} = C\left( 1 + e \cos(\phi - \phi_{0}) \right),\quad C = \frac{\mu k}{\ell^{2}},\; e = \sqrt{1 + \frac{2E \ell^{2}}{\mu k^{2}}}.
\end{align}
The above is simply the equation of a general conic section with one focus at the origin. Note that if $E > 0$, then $e > 1$ resulting in the shape of a hyperbola. If $E = 0$, then $e = 1$ resulting in the shape of a parabola. If $E < 0$ such that $0 < e < 1$ then the orbit is elliptical. To get a circle, the eccentricity must be zero, which means that the energy is \textit{fixed}, as $E = -\frac{\mu k^{2}}{2\ell^{2}}$. Coming in with this energy results in a circular orbit.

\section{Rotational Motion}

\textit{September 24th.}

In two dimensional motion, rotational motion can simply be thought of as the group action of the unitary group of matrices $U(1)$ on $\R^{2}$. This is an abelian group, and is simple to work with. In three dimensions, however, one must consider the group action of special orthogonal matrices $SO(3)$ on $\R^{3}$. This is a non-abelian group, and hence the order of rotations matters.

For rotation on a plane, the direction of $\vec{L}$ is fixed, and we simply have
\begin{align}
    \vec{L} = \abs{\vec{L}} \hat{e}_{L},\quad \frac{d\vec{L}}{dt} = \frac{d\abs{\vec{L}}}{dt} \hat{e}_{L}.
\end{align}
Note that since $\vec{L} = \vec{r} \times \vec{p}$, the angular momentum depends on the choice of the origin. Taking the time derivative of the total angular momentum of a system of particles gives
\begin{align}
    \frac{d\vec{L}_{\text{tot}}}{dt} = \frac{d}{dt} \sum_{i=1}^{N} \vec{r}_{i} \times \vec{p}_{i} = \sum_{i=1}^{N} \left(\frac{d \vec{r}_{i}}{dt} \times \vec{p}_{i}+ \vec{r}_{i} \times \frac{d \vec{p}_{i}}{dt} \right) = \sum_{i=1}^{N} \vec{r}_{i} \times \vec{F}_{i} = \vec{\tau}_{\text{tot}}.
\end{align}

Let us consider rotation only about a fixes axis, say about $\hat{e}_{z}$. The angular velocity may be denoted by $\vec{\omega} = \omega \hat{e}_{z}$. The velocity of a particle at position $\vec{r}_{i}$ is given by $\vec{v}_{i} = \vec{\omega} \times \vec{r}_{i}$. In the simplest case, there is only one particle of mass $m$ at position $\vec{r}$ and velocity $\vec{v}$; its angular momentum will simply be $\vec{L} = \vec{r} \times m \vec{v}$. For a continuous body $S$, with $\vec{p} = (dm) \vec{v}$, the angular momentum is given by
\begin{align}
    \vec{L} = \int_{S} \vec{r} \times \vec{p} = \int_{S} (\vec{r} \times \vec{v}) dm = \int_{S} (\vec{r} \times (\vec{\omega} \times \vec{r})) dm.
\end{align}
Recalling the identity $\vec{a} \times (\vec{b} \times \vec{c}) = (\vec{a} \cdot \vec{c}) \vec{b} - \vec{c} (\vec{a} \cdot \vec{b})$, we obtain
\begin{align}
    \vec{L} = \int_{S} (\vec{r} \cdot \vec{r}) \vec{\omega} dm = \left( \int_{S} r^{2} dm \right) \omega \hat{e}_{z} = I_{zz} \omega \hat{e}_{z}
\end{align}
where the quantity $I_{zz} = \int_{S} r^{2} dm$ is termed the \eax{moment of inertia} about the $z$-axis. The subscript $zz$ is used is lieu of future generalizations. Thus, the relation gives $L = I \omega$ for rotation about a fixes axis. For a discrete case, $I_{zz} = \sum_{i=1}^{N} m_{i} r_{i}^{2}$. The kinetic energy of the rotating body, using $dT = \frac{1}{2}v^{2}(dm)$, is given by
\begin{align}
    T = \int_{S} dT = \int_{S} \frac{1}{2} (\omega r)^{2} dm = \frac{\omega^{2}}{2} \int_{S} r^{2} dm = \frac{1}{2} I_{zz} \omega^{2}.
\end{align}
The continuous version of the center of mass formula is
\begin{align}
    \left( \frac{\sum_{i} m_{i} \vec{r}_{i}}{\sum_{i} m_{i}} = \right)\; \vec{r}_{CM} = \frac{\int_{S} \vec{r} dm}{\int_{S} dm}.
\end{align}
We, earlier, obtained the equations $\vec{r} = \vec{r}_{CM} + \vec{r}'$ and $\vec{v} = \vec{v}_{CM} + \vec{v}'$, where the primed quantities are relative to the center of mass. Working as we did before in chapter 1, and using $\vec{p}_{CM} = M\vec{v}_{CM}$ and $\vec{v}' = \vec{\omega}' \times \vec{r}'$, we obtain
\begin{align}
    \vec{L} = \vec{r}_{CM} \times \vec{p}_{CM} + \left( \int_{S} r'^{2} dm \right) \omega' \hat{e}_{z}\; \text{ or }\; \vec{L} = \vec{L}_{CM} + I_{CM}^{zz} \omega' \hat{e}_{Z}.
\end{align}
The kinetic energy, when the center of mass transates and the rotations is $\vec{\omega}' = \omega' \hat{e}_{z}$, is given by
\begin{align}
    T = \frac{1}{2} M v_{CM}^{2} + \frac{1}{2} I_{CM}^{zz} \omega'^{2}.
\end{align}

If we consider rotation about an origin at another point $O$ with axis parallel to the rotation axis through the center of mass, then the kinetic energy gives
\begin{align}
    T = \frac{1}{2} M r_{CM}^{2} \omega^{2} + \frac{1}{2} I_{CM}^{zz} \omega^{2} = \frac{1}{2} (I_{CM}^{zz} + M r_{CM}^{2}) \omega^{2} = \frac{1}{2} I_{O}^{zz} \omega^{2}.
\end{align}
This is called the \eax{parallel axis theorem}, giving $I_{O}^{zz} = I_{CM}^{zz} + M r_{CM}^{2}$. The torque about $O$ is given by
\begin{align}
    \vec{L}_{O} = \sum_{i=1}^{N} m_{i}(\vec{r}_{i} - \vec{r}_{O}) \times (\dot{\vec{r}}_{i} - \dot{\vec{r}}_{O}) \implies \frac{d \vec{L}_{O}}{dt} = \sum_{i=1}^{N} (\vec{r}_{i} - \vec{r}_{O}) \times (\vec{F}_{i}^{\text{ext}} + \vec{F}_{i}^{\text{int}}).
\end{align}
One can show that the internal force vanishes for forces acting along $\vec{r}_{i} - \vec{r}_{j}$. Thus, we have
\begin{align}
    \frac{d \vec{L}_{O}}{dt} = \vec{\tau}_{\text{tot},O}^{\text{ext}} = \sum_{i=1}^{N} m_{i}(\vec{r}_{i}-\vec{r}_{O}) \times (-\ddot{\vec{r}}_{O}) = \vec{\tau}_{\text{tot},O}^{\text{ext}} + (\vec{r}_{CM} - \vec{r}_{O}) \times (-M \ddot{\vec{r}}_{O}).
\end{align}
This last term, $-M\ddot{\vec{r}}_{O}$, is called the \eax{fictitious force} or the \eax{pseudoforce}.

\subsection{Rotation in Three Dimensions}
\textit{September 29th.}

We will restrict ourselves to rigid bodies, where the distance between any two particles is a constant. We start by discussing the \eax{Mozzi-Chasles theorem}, which states that the most general motion of a rigid body can be represented as a rotation about an axis and a translation along the same axis. So, if we consider a point $P$ of the body, the entire motion can be inferred from the translation of $P$ and a single rotation about an axis through $P$. Thus the required quantities are $\vec{r}_{P}$, $\vec{v}_{P}$, and $\vec{\omega}$. For any other point $Q$ of the body, we have
\begin{align}
    \vec{r}_{Q} = \vec{r}_{P} + \vec{r}_{Q/P}, \quad \dot{\vec{r}}_{Q} = \dot{\vec{r}}_{P} + \dot{\vec{r}}_{Q/P}, \quad \ddot{\vec{r}}_{Q} = \ddot{\vec{r}}_{P} + \ddot{\vec{r}}_{Q/P}.
\end{align}
If $P$ is fixed and we perform an infinitesimal motion of any other point with no turning back, the \eax{Borsuk-Ulam theorem} states that there must be two antipodal points which do not move at all under this infinitesimal motion.

The quantities above are related as $\vec{v}_{p} = \vec{\omega} \times \vec{r}_{P}(t)$, with $\abs{\vec{v}_{P}} = \abs{\vec{\omega}} \abs{\vec{r}_{P}} \sin \theta$, where $\theta$ is the angle between $\vec{\omega}$ and $\vec{r}_{P}$. Suppose we have three bodies $S_{1}$, $S_{2}$, and $S_{3}$, with the relative angular velocity of $S_{1}$ with respect to $S_{2}$ being $\vec{\omega}_{12}$, and so on. Then the angular velocity of $S_{1}$ with respect to $S_{3}$ is given by
\begin{align}
    \vec{\omega}_{13} = \vec{\omega}_{12} + \vec{\omega}_{23}.
\end{align}

Now suppose we have a rigid body $S$ rotating about an axis, with $\vec{\omega} = \omega_{1} \hat{e}_{x} + \omega_{2} \hat{e}_{y} + \omega_{3} \hat{e}_{z}$. For an infinitesimal mass element of a point on the body, $dm = \rho(x,y,z) dx dy dz$, the angular momentum of the entire body is given as
\begin{align}
    \vec{L} = \int_{S} \vec{r} \times \vec{v} dm = \int_{S} dm(\vec{r} \times (\vec{\omega} \times \vec{r})) = \int_{S} dm \left( r^{2} \vec{\omega} - (\vec{r} \cdot \vec{\omega}) \vec{r} \right).
\end{align}
We now introduce the \eax{inertia tensor} $\tensor{I}$ as the linear transformation such that $\vec{L} = \tensor{I} \vec{\omega}$. We have
\begin{align}
    \vec{L} = \int_{S} dm \left( (x^{2}+y^{2}+z^{2})(\omega_{1} \hat{e}_{x} + \omega_{2} \hat{e}_{y} + \omega_{3} \hat{e}_{z}) - (x\omega_{1} + y\omega_{2} + z\omega_{3})(x\hat{e}_{x} + y\hat{e}_{y} + z\hat{e}_{z}) \right).
\end{align}
The $x$ component of $\vec{L}$ is
\begin{align}
    L_{x} = \vec{L} \cdot \hat{e}_{x} = \int_{S} dm \left( (y^{2}+z^{2}) \omega_{1} - yx \omega_{2} - xz \omega_{3} \right).
\end{align}
Similarly, we obtain $L_{y}$ and $L_{z}$. Rewriting this in matrix form, we see that
\begin{align}
    \begin{pmatrix}
        L_{x} \\ L_{y} \\ L_{z}
    \end{pmatrix} = \begin{pmatrix}
        \int dm (y^{2}+z^{2}) & - \int dm xy & - \int dm zx \\
        -\int dm xy & \int dm (z^{2}+x^{2}) & - \int dm yz \\
        -\int dm zx & - \int dm yz & \int dm (x^{2}+y^{2})
    \end{pmatrix} \begin{pmatrix}
        \omega_{1} \\ \omega_{2} \\ \omega_{3}
    \end{pmatrix}
\end{align}
where the matrix thus formed is called the inertia tensor $\tensor{I}$. The matrix here is symmetric, and is more often written as
\begin{align}
    I = \begin{pmatrix}
        I_{xx} & I_{xy} & I_{xz} \\
        I_{yx} & I_{yy} & I_{yz} \\
        I_{zx} & I_{zy} & I_{zz}
    \end{pmatrix}
\end{align}
where $I_{\mu \nu} = I_{\nu \mu}$.

\begin{example}
    Consider a uniform cube of side $a$ and mass $M$. The density of the cube is $\rho = \frac{M}{a^{3}}$. The moment of inertia about the $x$-axis is given by
    \begin{align}
        I_{xx} = \int_{S} (y^{2}+z^{2}) dm = \int_{0}^{a} \int_{0}^{a} \int_{0}^{a} (y^{2}+z^{2}) \rho dx dy dz = \frac{2}{3} Ma^{2}.
    \end{align}
    By symmetry, we have $I_{yy} = I_{zz} = \frac{2}{3} Ma^{2}$. The off-diagonal terms are given by
    \begin{align}
        I_{xy} = I_{yx} = -\int_{S} xy dm = -\int_{0}^{a} \int_{0}^{a} \int_{0}^{a} xy \rho dx dy dz = -\frac{1}{4}Ma^{2}.
    \end{align}
    By symmetry, we have $I_{xz} = I_{zx} = I_{yz} = I_{zy} = -\frac{1}{4}Ma^{2}$. Thus, the inertia tensor is given by
    \begin{align}
        \tensor{I} = Ma^{2} \begin{pmatrix}
            2/3 & -1/4 & -1/4 \\
            -1/4 & 2/3 & -1/4 \\
            -1/4 & -1/4 & 2/3
        \end{pmatrix}.
    \end{align}
    Let us look at the $x$-component of the moment of inertia. From the tensor relation above, we obtain
    \begin{align}
        L_{x} = \frac{2}{3}Ma^{2}\omega_{1} - \frac{1}{4} ML^{2} \omega_{2} - \frac{1}{4} Ma^{2} \omega_{3}.
    \end{align}
    Note that even if we set $\omega_{1} = \omega_{2} = 0$, that is, the cube only rotates about the $z$-axis, we still have $L_{x} = -\frac{1}{4} Ma^{2} \omega_{3} \neq 0$. Thus, $\vec{L}$ and $\vec{\omega}$ are not parallel in general.
\end{example}

We now move to further generalized motion. Here, the best idea is to choose $P$ to be the centre of mass. In such a case, we know that the angular momentum is
\begin{align}
    \vec{L} = M(\vec{r}_{CM} \times \vec{v}_{CM}) + \int_{S} dm (\vec{r}' \times (\vec{\omega}' \times \vec{r}'))
\end{align}
For any arbitrary point $Q$, note that $\vec{\omega}_{CM/O} = \vec{\omega}' = \vec{\omega}_{Q/CM}$. Thus we have
\begin{align}
    M \vec{r}_{CM} \times (\vec{\omega}' \times \vec{r}_{CM}) = M\left( (r_{CM}^{2}) \vec{\omega}' - (\vec{r}_{CM} \cdot \vec{\omega}') \vec{r}_{CM} \right)
\end{align}
We can then work as before to get relations such as $I_{xx}^{CM/O} = M(y_{CM}^{2}+z_{CM}^{2})$ and so on. We then obtain the \eax{generalized parallel axis theorem} as
\begin{align}
    \vec{L} = (\tensor{I}_{CM/O} + \tensor{I}_{S/CM}) \vec{\omega}' = \tensor{I}_{S/O} \vec{\omega}'.
\end{align}
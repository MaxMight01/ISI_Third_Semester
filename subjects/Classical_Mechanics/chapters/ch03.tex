\chapter{CENTRAL FORCE MOTION AND ROTATIONAL PHYSICS}

\section{The Central Force Problem}
\subsection{One Body Problem}
\textit{September 15th.}

Consider two particles of masses lying $m_{1}$ and $m_{2}$ lying at positions $\vec{r}_{1}$ and $\vec{r}_{2}$ respectively. If $\vec{r} = \vec{r}_{2}-\vec{r}_{1}$ is the relative position vector, and $\vec{R}$ is the position vector of the centre of mass, then the kinetic energy may be given as
\begin{align}
    T = \frac{1}{2}m_{1}\dot{\vec{r}}_{1}^{2} + \frac{1}{2}m_{2}\dot{\vec{r}}_{2}^{2} = \frac{1}{2}(m_{1}+m_{2})\dot{\vec{R}}^{2} + \frac{1}{2}\left( \frac{m_{1}m_{2}}{m_{1}+m_{2}} \right) \dot{\vec{r}}^{2}.
\end{align}
Here the quantity $\mu = \frac{m_{1}m_{2}}{m_{1}+m_{2}}$ is called the \eax{reduced mass}. If the potential energy of the system is $U(\vec{r})$, only dependent on the relative position, then the Lagrangian of the system may be written as
\begin{align}
    L = T-U = \frac{1}{2}M\dot{R}^{2} + \frac{1}{2}\mu \dot{r}^{2} - U(\vec{r})
\end{align}
Note that $R$ is a cyclic coordinate here. So
\begin{align}
    \frac{\partial L}{\partial \dot{\vec{R}}} = M\dot{\vec{R}} = \vec{P}_{CM} \implies \frac{d}{dt} \vec{P}_{CM} = 0.
\end{align}
The momentum of the center of mass is conserved. For the most popular of cases, we consider $U(\vec{r}) \equiv U(r)$, i.e., the potential energy depends only on the magnitude of the relative position vector. This is called a \eax{conservative central force}. If a particle of mass $m_{1}$ is fixed at the origin, then the other particle of mass $m_{2}$ moves in the potential $U(r)$ with reduced mass $\mu$. We shall focus our analysis only on the reduced mass since the center of mass positive is a cyclic coordinate. The Lagrangian of the system is
\begin{align}
    L = \frac{1}{2}\mu(\dot{r}^{2} + r^{2}\dot{\theta}^{2} + r^{2}\sin^{2}\theta \dot{\phi}^{2}) - U(r).
\end{align}
Since the problem is spherically symmetric, the angular momentum $\vec{L} = \vec{r} \times \mu \dot{\vec{r}}$ is a conserved quantity. We can infer that the motion is planar in central force (first property). Choose the axis of $\vec{L}$ to be the $z$-axis. Then $\theta = \frac{\pi}{2}$. The Lagrangian reduces to
\begin{align}
    L = T-U = \frac{1}{2}\mu(\dot{r}^{2} + r^{2}\dot{\phi}^{2}) - U(r).
\end{align}
Here, $\phi$ is a cyclic coordinate and hence the angular momentum $\ell = \mu r^{2}\dot{\phi}$ is conserved, giving us $\dot{\phi} = \frac{\ell}{\mu r^{2}}$. From this conservation of $\ell$, it also follows that $\frac{d}{dt}(\frac{1}{2}\mu r^{2}) = 0 \implies \frac{1}{2}r^{2}\dot{\phi}$ is constant. We can interpret this as $\frac{1}{2}r(r\dot{\phi}) = dA$, the infinitesimal area swept out by the radius vector in time $dt$. The \eax{areal velocity} $\frac{dA}{dt}$ is constant; equal area is swept out in equal time (second property). The energy function is
\begin{align}
    h = \frac{1}{2}\mu\dot{r}^{2} + \frac{\ell^{2}}{2\mu r^{2}} + U(r) \implies \frac{dh}{dt} = \mu\ddot{r} - \frac{\ell^{2}}{\mu r^{3}} + \frac{dU}{dr} = 0.
\end{align}
Rewriting gives us
\begin{align}
    \mu\ddot{r} = -\frac{d}{dr}\left( U + \frac{\ell^{2}}{2\mu r^{2}} \right) = -\frac{dU_{\text{eff}}(r)}{dr}
\end{align}
where the new quantity is termed the \eax{effective potential energy}. The quantity $T_{\text{eff}} = \frac{1}{2}\mu\dot{r}^{2}$ is termed the effective kinetic energy. The total energy is $h = T_{\text{eff}} + U_{\text{eff}}$. Summarising all the properties, the formal solution for the problem is
\begin{align}
    \ell = \mu r^{2}\dot{\phi},\; E=\frac{1}{2}\mu \dot{r}^{2} + \frac{\ell^{2}}{2\mu r^{2}} + U,\; \dot{r} = \sqrt{\frac{2}{\mu}\left( E-U-\frac{\ell^{2}}{2 \mu r^{2}} \right)}.
\end{align}
Thus, $r$ can be found as a function of $t$ by integrating
\begin{align}
    \int_{r_{0}}^{r} \frac{dr}{\sqrt{\frac{2}{\mu}\left( E-U-\frac{\ell^{2}}{2 \mu r^{2}} \right)}} = \int_{0}^{t} dt = t
\end{align}
and inverting the solution to get a function $r(t)$. The angle $\phi$ can be found as a function of $t$ by integrating
\begin{align}
    \frac{d\phi}{dt} = \frac{\ell}{\mu r^{2}} \implies \phi - \phi_{0} = \frac{\ell}{\mu}\int_{0}^{t} \frac{dt}{r^{2}(t)}.
\end{align}
The energy-distance diagrams using $U_{\text{eff}} = \frac{\ell^{2}}{2\mu r^{2}} + U(r)$ are very useful in visualising the motion of the particle. The points where $E = U_{\text{eff}}$ are the turning points of the motion. If $E < 0$, the motion is bounded, and if $E > 0$, the motion is unbounded. The point where $U_{\text{eff}}$ is minimum is a point of stable equilibrium. For a potential of the form $U(r) = -\frac{k}{r}$, this `orbit' of stable equilbrium can be found by setting $\frac{dU_{\text{eff}}}{dr} = 0$ and $\frac{d^{2}U_{\text{eff}}}{dr^{2}} > 0$. This gives us $r_{0} = \frac{\ell^{2}}{\mu k}$. But setting $E = U_{\text{eff}}(r_{0})$ gives us the effective force $f_{\text{eff}}$ to be zero, or $\ddot{r} = 0$. This means that the radius is constant, and the motion is circular.\\

\noindent\textit{September 17th.}

With $u = \frac{1}{r}$, and $\frac{du}{d\phi} = -\frac{1}{r^{2}} \frac{dr}{d\phi}$, the differential equation can be written as differential operators; using this, we shall rewrite our equation of motion.
\begin{align}
    \frac{d}{dt} = \frac{\ell}{\mu r^{2}} \frac{d}{d\phi} \implies \mu \ddot{r} = \mu \frac{d}{dt}\left( \frac{dr}{dt} \right) = \frac{\ell}{r^{2}} \frac{d}{d\phi}\left( \frac{\ell}{\mu r^{2}} \frac{dr}{d\phi} \right) = -\frac{\ell^{2}}{\mu} u^{2} \frac{d^{2}u}{d\phi^{2}}.
\end{align}
The second term is simply $\frac{\ell^{2}}{\mu r^{3}} = \frac{\ell^{2}}{\mu} u^{3}$. The final term is $\frac{dU}{dr} = \frac{dU}{du} \frac{du}{dr} = -u^{2} \frac{dU}{du}$. Thus, the equation of motion becomes
\begin{align}
    \frac{d^{2}u}{d\phi^{2}} + u = -\frac{\mu}{\ell^{2}} \frac{dU}{du}.
\end{align}
Thus, given $(u_{0},\phi_{0},\ell,E)$, we can solve for $u = u(\phi)$ and hence $r = r(\phi)$. Recall the relation between the differentials $dt$ and $dr$. Rewriting $dt$ in terms of $d\phi$, we have
\begin{align}
    d\phi = \frac{\ell dr}{\mu r^{2} \sqrt{\frac{2}{\mu} \left( E-U(r) - \frac{\ell^{2}}{2\mu r^{2}} \right)}}.
\end{align}
Integrating gives us
\begin{align}
    \phi - \phi_{0} = -\int_{u_{0}}^{u} \frac{du}{\sqrt{\frac{2\mu E}{\ell^{2}} - \frac{2\mu U(u)}{\ell^{2}} -u^{2}}}.
\end{align}

\subsection{The Kepler Problem}
We are simply considering $U = -\frac{k}{r} = -ku$ for some $k > 0$. Replacing this in the above equation gives
\begin{align}
    \phi - \phi_{0} = -\int_{u_{0}}^{u} \frac{du}{\sqrt{\frac{2\mu E}{\ell^{2}} + \frac{2\mu k}{\ell^{2}}u - u^{2}}}.
\end{align}
Recalling that
\begin{align}
    \int \frac{dx}{\sqrt{\alpha + \beta x + \gamma x^{2}}} = \frac{1}{\sqrt{-\gamma}} \cos^{-1} \left( -\frac{(\beta + 2\gamma x)}{\sqrt{\beta^{2}} - 4\alpha \gamma^{2}a} \right) + C,
\end{align}
we obtain
\begin{align}
    \phi - \phi_{0} = \cos^{-1}\left( \frac{\frac{\ell^{2} u}{\mu k} - 1}{\sqrt{1 + \frac{2 E \ell^{2}}{\mu k^{2}}}} \right).
\end{align}
Inverting the equation gives
\begin{align}
    \frac{1}{r} = u = \frac{\mu k}{\ell^{2}} \left( 1 + \sqrt{1 + \frac{2E \ell^{2}}{\mu k^{2}}} \cos(\phi - \phi_{0}) \right)
\end{align}
or just simply
\begin{align}
    \frac{1}{r} = C\left( 1 + e \cos(\phi - \phi_{0}) \right),\quad C = \frac{\mu k}{\ell^{2}},\; e = \sqrt{1 + \frac{2E \ell^{2}}{\mu k^{2}}}.
\end{align}
The above is simply the equation of a general conic section with one focus at the origin. Note that if $E > 0$, then $e > 1$ resulting in the shape of a hyperbola. If $E = 0$, then $e = 1$ resulting in the shape of a parabola. If $E < 0$ such that $0 < e < 1$ then the orbit is elliptical. To get a circle, the eccentricity must be zero, which means that the energy is \textit{fixed}, as $E = -\frac{\mu k^{2}}{2\ell^{2}}$. Coming in with this energy results in a circular orbit.

\section{Rotational Motion}

\textit{September 24th.}

In two dimensional motion, rotational motion can simply be thought of as the group action of the unitary group of matrices $U(1)$ on $\R^{2}$. This is an abelian group, and is simple to work with. In three dimensions, however, one must consider the group action of special orthogonal matrices $SO(3)$ on $\R^{3}$. This is a non-abelian group, and hence the order of rotations matters.

For rotation on a plane, the direction of $\vec{L}$ is fixed, and we simply have
\begin{align}
    \vec{L} = \abs{\vec{L}} \hat{e}_{L},\quad \frac{d\vec{L}}{dt} = \frac{d\abs{\vec{L}}}{dt} \hat{e}_{L}.
\end{align}
Note that since $\vec{L} = \vec{r} \times \vec{p}$, the angular momentum depends on the choice of the origin. Taking the time derivative of the total angular momentum of a system of particles gives
\begin{align}
    \frac{d\vec{L}_{\text{tot}}}{dt} = \frac{d}{dt} \sum_{i=1}^{N} \vec{r}_{i} \times \vec{p}_{i} = \sum_{i=1}^{N} \left(\frac{d \vec{r}_{i}}{dt} \times \vec{p}_{i}+ \vec{r}_{i} \times \frac{d \vec{p}_{i}}{dt} \right) = \sum_{i=1}^{N} \vec{r}_{i} \times \vec{F}_{i} = \vec{\tau}_{\text{tot}}.
\end{align}

Let us consider rotation only about a fixes axis, say about $\hat{e}_{z}$. The angular velocity may be denoted by $\vec{\omega} = \omega \hat{e}_{z}$. The velocity of a particle at position $\vec{r}_{i}$ is given by $\vec{v}_{i} = \vec{\omega} \times \vec{r}_{i}$. In the simplest case, there is only one particle of mass $m$ at position $\vec{r}$ and velocity $\vec{v}$; its angular momentum will simply be $\vec{L} = \vec{r} \times m \vec{v}$. For a continuous body $S$, with $\vec{p} = (dm) \vec{v}$, the angular momentum is given by
\begin{align}
    \vec{L} = \int_{S} \vec{r} \times \vec{p} = \int_{S} (\vec{r} \times \vec{v}) dm = \int_{S} (\vec{r} \times (\vec{\omega} \times \vec{r})) dm.
\end{align}
Recalling the identity $\vec{a} \times (\vec{b} \times \vec{c}) = (\vec{a} \cdot \vec{c}) \vec{b} - \vec{c} (\vec{a} \cdot \vec{b})$, we obtain
\begin{align}
    \vec{L} = \int_{S} (\vec{r} \cdot \vec{r}) \vec{\omega} dm = \left( \int_{S} r^{2} dm \right) \omega \hat{e}_{z} = I_{zz} \omega \hat{e}_{z}
\end{align}
where the quantity $I_{zz} = \int_{S} r^{2} dm$ is termed the \eax{moment of inertia} about the $z$-axis. The subscript $zz$ is used is lieu of future generalizations. Thus, the relation gives $L = I \omega$ for rotation about a fixes axis. For a discrete case, $I_{zz} = \sum_{i=1}^{N} m_{i} r_{i}^{2}$. The kinetic energy of the rotating body, using $dT = \frac{1}{2}v^{2}(dm)$, is given by
\begin{align}
    T = \int_{S} dT = \int_{S} \frac{1}{2} (\omega r)^{2} dm = \frac{\omega^{2}}{2} \int_{S} r^{2} dm = \frac{1}{2} I_{zz} \omega^{2}.
\end{align}
The continuous version of the center of mass formula is
\begin{align}
    \left( \frac{\sum_{i} m_{i} \vec{r}_{i}}{\sum_{i} m_{i}} = \right)\; \vec{r}_{CM} = \frac{\int_{S} \vec{r} dm}{\int_{S} dm}.
\end{align}
We, earlier, obtained the equations $\vec{r} = \vec{r}_{CM} + \vec{r}'$ and $\vec{v} = \vec{v}_{CM} + \vec{v}'$, where the primed quantities are relative to the center of mass. Working as we did before in chapter 1, and using $\vec{p}_{CM} = M\vec{v}_{CM}$ and $\vec{v}' = \vec{\omega}' \times \vec{r}'$, we obtain
\begin{align}
    \vec{L} = \vec{r}_{CM} \times \vec{p}_{CM} + \left( \int_{S} r'^{2} dm \right) \omega' \hat{e}_{z}\; \text{ or }\; \vec{L} = \vec{L}_{CM} + I_{CM}^{zz} \omega' \hat{e}_{Z}.
\end{align}
The kinetic energy, when the center of mass transates and the rotations is $\vec{\omega}' = \omega' \hat{e}_{z}$, is given by
\begin{align}
    T = \frac{1}{2} M v_{CM}^{2} + \frac{1}{2} I_{CM}^{zz} \omega'^{2}.
\end{align}

If we consider rotation about an origin at another point $O$ with axis parallel to the rotation axis through the center of mass, then the kinetic energy gives
\begin{align}
    T = \frac{1}{2} M r_{CM}^{2} \omega^{2} + \frac{1}{2} I_{CM}^{zz} \omega^{2} = \frac{1}{2} (I_{CM}^{zz} + M r_{CM}^{2}) \omega^{2} = \frac{1}{2} I_{O}^{zz} \omega^{2}.
\end{align}
This is called the \eax{parallel axis theorem}, giving $I_{O}^{zz} = I_{CM}^{zz} + M r_{CM}^{2}$. The torque about $O$ is given by
\begin{align}
    \vec{L}_{O} = \sum_{i=1}^{N} m_{i}(\vec{r}_{i} - \vec{r}_{O}) \times (\dot{\vec{r}}_{i} - \dot{\vec{r}}_{O}) \implies \frac{d \vec{L}_{O}}{dt} = \sum_{i=1}^{N} (\vec{r}_{i} - \vec{r}_{O}) \times (\vec{F}_{i}^{\text{ext}} + \vec{F}_{i}^{\text{int}}).
\end{align}
One can show that the internal force vanishes for forces acting along $\vec{r}_{i} - \vec{r}_{j}$. Thus, we have
\begin{align}
    \frac{d \vec{L}_{O}}{dt} = \vec{\tau}_{\text{tot},O}^{\text{ext}} = \sum_{i=1}^{N} m_{i}(\vec{r}_{i}-\vec{r}_{O}) \times (-\ddot{\vec{r}}_{O}) = \vec{\tau}_{\text{tot},O}^{\text{ext}} + (\vec{r}_{CM} - \vec{r}_{O}) \times (-M \ddot{\vec{r}}_{O}).
\end{align}
This last term, $-M\ddot{\vec{r}}_{O}$, is called the \eax{fictitious force} or the \eax{pseudoforce}.

\subsection{Rotation in Three Dimensions}
\textit{September 29th.}

We will restrict ourselves to rigid bodies, where the distance between any two particles is a constant. We start by discussing the \eax{Mozzi-Chasles theorem}, which states that the most general motion of a rigid body can be represented as a rotation about an axis and a translation along the same axis. So, if we consider a point $P$ of the body, the entire motion can be inferred from the translation of $P$ and a single rotation about an axis through $P$. Thus the required quantities are $\vec{r}_{P}$, $\vec{v}_{P}$, and $\vec{\omega}$. For any other point $Q$ of the body, we have
\begin{align}
    \vec{r}_{Q} = \vec{r}_{P} + \vec{r}_{Q/P}, \quad \dot{\vec{r}}_{Q} = \dot{\vec{r}}_{P} + \dot{\vec{r}}_{Q/P}, \quad \ddot{\vec{r}}_{Q} = \ddot{\vec{r}}_{P} + \ddot{\vec{r}}_{Q/P}.
\end{align}
If $P$ is fixed and we perform an infinitesimal motion of any other point with no turning back, the \eax{Borsuk-Ulam theorem} states that there must be two antipodal points which do not move at all under this infinitesimal motion.

The quantities above are related as $\vec{v}_{p} = \vec{\omega} \times \vec{r}_{P}(t)$, with $\abs{\vec{v}_{P}} = \abs{\vec{\omega}} \abs{\vec{r}_{P}} \sin \theta$, where $\theta$ is the angle between $\vec{\omega}$ and $\vec{r}_{P}$. Suppose we have three bodies $S_{1}$, $S_{2}$, and $S_{3}$, with the relative angular velocity of $S_{1}$ with respect to $S_{2}$ being $\vec{\omega}_{12}$, and so on. Then the angular velocity of $S_{1}$ with respect to $S_{3}$ is given by
\begin{align}
    \vec{\omega}_{13} = \vec{\omega}_{12} + \vec{\omega}_{23}.
\end{align}

Now suppose we have a rigid body $S$ rotating about an axis, with $\vec{\omega} = \omega_{1} \hat{e}_{x} + \omega_{2} \hat{e}_{y} + \omega_{3} \hat{e}_{z}$. For an infinitesimal mass element of a point on the body, $dm = \rho(x,y,z) dx dy dz$, the angular momentum of the entire body is given as
\begin{align}
    \vec{L} = \int_{S} \vec{r} \times \vec{v} dm = \int_{S} dm(\vec{r} \times (\vec{\omega} \times \vec{r})) = \int_{S} dm \left( r^{2} \vec{\omega} - (\vec{r} \cdot \vec{\omega}) \vec{r} \right).
\end{align}
We now introduce the \eax{inertia tensor} $\tensor{I}$ as the linear transformation such that $\vec{L} = \tensor{I} \vec{\omega}$. We have
\begin{align}
    \vec{L} = \int_{S} dm \left( (x^{2}+y^{2}+z^{2})(\omega_{1} \hat{e}_{x} + \omega_{2} \hat{e}_{y} + \omega_{3} \hat{e}_{z}) - (x\omega_{1} + y\omega_{2} + z\omega_{3})(x\hat{e}_{x} + y\hat{e}_{y} + z\hat{e}_{z}) \right).
\end{align}
The $x$ component of $\vec{L}$ is
\begin{align}
    L_{x} = \vec{L} \cdot \hat{e}_{x} = \int_{S} dm \left( (y^{2}+z^{2}) \omega_{1} - yx \omega_{2} - xz \omega_{3} \right).
\end{align}
Similarly, we obtain $L_{y}$ and $L_{z}$. Rewriting this in matrix form, we see that
\begin{align}
    \begin{pmatrix}
        L_{x} \\ L_{y} \\ L_{z}
    \end{pmatrix} = \begin{pmatrix}
        \int dm (y^{2}+z^{2}) & - \int dm xy & - \int dm zx \\
        -\int dm xy & \int dm (z^{2}+x^{2}) & - \int dm yz \\
        -\int dm zx & - \int dm yz & \int dm (x^{2}+y^{2})
    \end{pmatrix} \begin{pmatrix}
        \omega_{1} \\ \omega_{2} \\ \omega_{3}
    \end{pmatrix}
\end{align}
where the matrix thus formed is called the inertia tensor $\tensor{I}$. The matrix here is symmetric, and is more often written as
\begin{align}
    I = \begin{pmatrix}
        I_{xx} & I_{xy} & I_{xz} \\
        I_{yx} & I_{yy} & I_{yz} \\
        I_{zx} & I_{zy} & I_{zz}
    \end{pmatrix}
\end{align}
where $I_{\mu \nu} = I_{\nu \mu}$.

\begin{example}
    Consider a uniform cube of side $a$ and mass $M$. The density of the cube is $\rho = \frac{M}{a^{3}}$. The moment of inertia about the $x$-axis is given by
    \begin{align}
        I_{xx} = \int_{S} (y^{2}+z^{2}) dm = \int_{0}^{a} \int_{0}^{a} \int_{0}^{a} (y^{2}+z^{2}) \rho dx dy dz = \frac{2}{3} Ma^{2}.
    \end{align}
    By symmetry, we have $I_{yy} = I_{zz} = \frac{2}{3} Ma^{2}$. The off-diagonal terms are given by
    \begin{align}
        I_{xy} = I_{yx} = -\int_{S} xy dm = -\int_{0}^{a} \int_{0}^{a} \int_{0}^{a} xy \rho dx dy dz = -\frac{1}{4}Ma^{2}.
    \end{align}
    By symmetry, we have $I_{xz} = I_{zx} = I_{yz} = I_{zy} = -\frac{1}{4}Ma^{2}$. Thus, the inertia tensor is given by
    \begin{align}
        \tensor{I} = Ma^{2} \begin{pmatrix}
            2/3 & -1/4 & -1/4 \\
            -1/4 & 2/3 & -1/4 \\
            -1/4 & -1/4 & 2/3
        \end{pmatrix}.
    \end{align}
    Let us look at the $x$-component of the moment of inertia. From the tensor relation above, we obtain
    \begin{align}
        L_{x} = \frac{2}{3}Ma^{2}\omega_{1} - \frac{1}{4} ML^{2} \omega_{2} - \frac{1}{4} Ma^{2} \omega_{3}.
    \end{align}
    Note that even if we set $\omega_{1} = \omega_{2} = 0$, that is, the cube only rotates about the $z$-axis, we still have $L_{x} = -\frac{1}{4} Ma^{2} \omega_{3} \neq 0$. Thus, $\vec{L}$ and $\vec{\omega}$ are not parallel in general.
\end{example}

We now move to further generalized motion. Here, the best idea is to choose $P$ to be the centre of mass. In such a case, we know that the angular momentum is
\begin{align}
    \vec{L} = M(\vec{r}_{CM} \times \vec{v}_{CM}) + \int_{S} dm (\vec{r}' \times (\vec{\omega}' \times \vec{r}'))
\end{align}
For any arbitrary point $Q$, note that $\vec{\omega}_{CM/O} = \vec{\omega}' = \vec{\omega}_{Q/CM}$. Thus we have
\begin{align}
    M \vec{r}_{CM} \times (\vec{\omega}' \times \vec{r}_{CM}) = M\left( (r_{CM}^{2}) \vec{\omega}' - (\vec{r}_{CM} \cdot \vec{\omega}') \vec{r}_{CM} \right)
\end{align}
We can then work as before to get relations such as $I_{xx}^{CM/O} = M(y_{CM}^{2}+z_{CM}^{2})$ and so on. We then obtain the \eax{generalized parallel axis theorem} as
\begin{align}
    \vec{L} = (\tensor{I}_{CM/O} + \tensor{I}_{S/CM}) \vec{\omega}' = \tensor{I}_{S/O} \vec{\omega}'.
\end{align}

\section{Rigid Body Rotation}

\textit{October 8th.}

When we start with a rigid body comprising of $N$ particles, we are dealing with $3N$ degrees of freedom. However, a rigid body is defined by constraints of the form $\abs{\vec{r}_{i}-\vec{r}_{j}} = r_{ij} = c_{ij}$ and still these constraints are not independent. The general idea is to choose three non-collinear points $P_{1},P_{2},P_{3}$ in the rigid body, which gives us $9$ degrees of freedom, and the distances between then give $3$ constraints, giving the final number as $6$ independent degrees of freedom at any given moment.

Thus first establish position of $P_{1}$ via $x_{P_{1}},y_{P_{1}},z_{P_{1}}$ as a function of time, and then specify the position of $P_{2}$ via $\theta,\phi$ as functions of time on the sphere centred at $P_{1}$ with radius $r = c_{1}$. Finally, specify the position of $P_{3}$ via $\lambda$ as a function of time on the intersecting curve. Of course, this is just a general idea, and we usually use any 6 independent coordinates which are the most convenient to the problem at hand.

Suppose we have a rigid body $S$ with which we have associated a coordinate system $O'$ and $x',y',z'$ which is fixed with respect to the body. We wish to relate this coordinate system with the standard one (an inertial frame) and derive our 6 generalized coordinates as so. We will location the position of $O'(t)$ and then the orientation of $(x',y',z')(t)$ with respect to $(x,y,z)$.

If we denote $\cos \theta_{ij} = \hat{e}_{x_{i}'} \cdot \hat{e}_{x_{j}}$, (these are functions of time and are not fixed), we obtain
\begin{align}
    \hat{e}_{x'} &= (\hat{e}_{x'} \cdot \hat{e}_{x}) \hat{e}_{x} + (\hat{e}_{x'} \cdot \hat{e}_{y}) \hat{e}_{y} + (\hat{e}_{x'} \cdot \hat{e}_{z}) \hat{e}_{z} = \cos \theta_{11} \hat{e}_{x} + \cos \theta_{12} \hat{e}_{y} + \cos \theta_{13} \hat{e}_{z}, \\
    \hat{e}_{y'} &= \cos \theta_{21} \hat{e}_{x} + \cos \theta_{22} \hat{e}_{y} + \cos \theta_{23} \hat{e}_{z}, \\
    \hat{e}_{z'} &= \cos \theta_{31} \hat{e}_{x} + \cos \theta_{32} \hat{e}_{y} + \cos \theta_{33} \hat{e}_{z}.
\end{align}
If we write $\vec{r} = x \hat{e}_{x} + y \hat{e}_{y} + z \hat{e}_{z} = x' \hat{e}_{x'} + y' \hat{e}_{y'} + z' \hat{e}_{z'}$, utilising the above relations gives us
\begin{align}
    x' &= x \cos \theta_{11} + y \cos \theta_{12} + z \cos \theta_{13},\\
    y' &= x \cos \theta_{21} + y \cos \theta_{22} + z \cos \theta_{23},\\
    z' &= x \cos \theta_{31} + y \cos \theta_{32} + z \cos \theta_{33}.
\end{align}
Using the fact that $\hat{e}_{x_{i}} \cdot \hat{e}_{x_{j}} = \delta_{ij}$, we obtain the relation
\begin{align}
    \sum_{l=1}^{3} \cos \theta_{lm} \cos \theta_{ln} = \delta_{mn}.
\end{align}
Note that we have written $x_{i}' = \sum_{j} a_{ij} x_{j}$, where $a_{ij} = \cos \theta_{ij}$. It is crucial to note that what we have written down is a linear combination. The relation $\sum_{i} a_{ij} a_{ik} = \delta_{jk}$ implies that the matrix $A = (a_{ij})_{3 \times 3}$ is an orthogonal matrix, i.e., $A^{T}A = AA^{T} = I_{3}$. Let us consider rotation in a plane; in such a case, we have $\hat{e}_{x'} = \cos \phi \hat{e}_{x} + \sin \phi \hat{e}_{y}$, $\hat{e}_{y'} = -\sin \phi \hat{e}_{x} + \cos \phi \hat{e}_{y}$, and $\hat{e}_{z'} = \hat{e}_{z}$. The matrix $A$, in the case of rotation in the $xy$-plane, is given by
\begin{align}
    A_{xy} = \begin{bmatrix}
        \cos \phi & \sin \phi & 0 \\
        -\sin \phi & \cos \phi & 0 \\
        0 & 0 & 1
    \end{bmatrix}.
\end{align}
About the other orthogonal axes, we find $A$ to be of the form
\begin{align}
    A_{yz} = \begin{bmatrix}
        1 & 0 & 0 \\
        0 & \cos \phi & \sin \phi \\
        0 & -\sin \phi & \cos \phi
    \end{bmatrix},\quad A_{zx} = \begin{bmatrix}
        \cos \phi & 0 & -\sin \phi \\
        0 & 1 & 0 \\
        \sin \phi & 0 & \cos \phi
    \end{bmatrix}.
\end{align}
Of course, one may verify that if $A$ and $B$ are orthogonal, then so is $AB$. Moreover, $A^{-1} = A^{T}$ and the identity matrix $I$ is also orthogonal. Thus, the set of all orthogonal matrices (of order $n$) forms a group $O(n)$ under matrix multiplication. Note that $\det(A) = \pm 1$. The orthogonal matrices with determinant $-1$ do not represent pure rotations, but rather a combination of a rotation and a reflection. The set of orthogonal matrices with determinant $+1$ form a subgroup termed the special orthogonal group $SO(n)$, and we deal with $SO(3)$ for three dimensional rotations. A special property of $SO(3)$ is that any $A \in SO(3)$ can be built up as a tranformation from infinitesimal rotations. This is what makes $SO(3)$ something known as a Lie group.

\subsection{Euler Angles}

We now deal with the standard convention of the choice of 3 angles to build up a rotation. In this convention, we use 3 subsequent rotations to build up a general (final) rotation.\\

\textit{October 13th.}
\begin{theorem}[\eax{Euler's theorem of rigid body motion}]
    If a rigid body moves with one point fixed, the motion is a rotation.
\end{theorem}

Our goal, then becomes, is given $A(t)$, we wish to find $\hat{n}$ and $\Phi$ such that $A(t)$ is a rotation about $\hat{n}$ by an angle $\Phi$. We start with the fact that $A \in SO(3)$, and hence $A^{T}A = I$ and $\det(A) = 1$. Note that if $\vec{R}$ is the rotation axis, then $A\vec{R} = \vec{R}$ tells us that $\vec{R}$ is an eigenvector of $A$ with eigenvalue $1$. For three eigenvalues $\lambda_{1},\lambda_{2},\lambda_{3}$, we have $\lambda_{1}\lambda_{2}\lambda_{3} = \det(A) = 1$. Also,
\begin{align}
    (A-I)A^{T} = I - A^{T} \implies \det(A-I) = \det(I-A^{T}) = \det(I-A) \implies \det(A-I) = 0.
\end{align}
Thus, one of the eigenvalues is $1$, and such a $\vec{R}$ exists. The other two eigenvalues $\lambda_{1}$ and $\lambda_{2}$ are multiplicative inverses of each other. Note that $A$ is real; if $\lambda$ is an eigenvalue, then so is $\lambda^{*}$. Thus, the other two eigenvalues are complex conjugates of each other, and hence of the form $e^{i\Phi}$ and $e^{-i\Phi}$.

In the case of all eigenvalues begin unity, the rotation is simply the identity. In the case of $\lambda_{1} = \lambda_{2} = -1 = e^{-i\Phi}$, we have $\Phi = \pi$, and the rotation is a rotation by $\pi$ about some axis. In the general case where $A_{ij} = a_{ij}$, and $\vec{R} = (x,y,z)^{t}$, we have
\begin{align}
    \begin{pmatrix}
        a_{11}-1 & a_{12} & a_{13} \\
        a_{21} & a_{22}-1 & a_{23} \\
        a_{31} & a_{32} & a_{33}-1
    \end{pmatrix} \begin{pmatrix}
        x \\ y \\ z
    \end{pmatrix} = 0.
\end{align}
This gives us two independent equations, and we can add the constraint $x^{2}+y^{2}+z^{2} = 1$ to obtain $\vec{R}$. There exists a similarity transformation $A' = X^{-1}AX$ such that
\begin{align}
    A' = \begin{pmatrix}
        \cos \Phi & \sin \Phi & 0 \\
        -\sin \Phi & \cos \Phi & 0 \\
        0 & 0 & 1
    \end{pmatrix}.
\end{align}
The advantage of a similarity transformation is that it preserves eigenvalues, and hence the trace. Thus, $\tr A = 1 + 2\cos \Phi$, giving
\begin{align}
    \Phi = \cos^{-1} \left( \frac{1}{2}(\tr A - 1) \right), \quad 0 \leq \Phi \leq 2\pi.
\end{align}
Thus, given $A$, we can find $\Phi$. Finding $\vec{R}$, however, is not as straightforward. The matrix equation above subject to the unit vector constraint gives us $\vec{R}$. There is, of course, an ambiguity in the sense that $-\vec{R}$ and $-\Phi$ are also valid solutions. 

There exist two representations of $A$, one in terms of $\hat{e}_{n}$ and $\Phi$, and the other in terms of Euler angles $R_{\psi},R_{\theta},R_{\phi}$. For $\vec{r} \mapsto \vec{r}' = A\vec{r}$ we wish to write $\vec{r}'$ in terms of $\vec{r},\hat{e}_{n}$, and $\Phi$. To this end, let $\vec{r}' = \vec{OQ}$ and $\vec{R} = \vec{OP}$. Let $N$ be the point on the rotation axis such that $ON \perp NP,NQ$, and let $V$ be the projection of $Q$ onto the line $NP$. Then
\begin{align}
    \vec{r}' = \vec{ON} + \vec{NV} + \vec{VQ}.
\end{align}
By definition of $N$, $\vec{ON} = (\vec{r} \cdot \hat{e}_{n}) \hat{e}_{n}$. Also,
\begin{align}
    \abs{\vec{NV}} = \abs{\vec{NQ}} \cos \Phi = \abs{\vec{NP}} \cos \Phi = \abs{\vec{r} - (\vec{r} \cdot \hat{e}_{n}) \hat{e}_{n}} \cos \Phi \implies \vec{NV} = (\vec{r} - \hat{e}_{n}(\hat{e}_{n} \cdot \vec{r})) \cos \Phi.
\end{align}
Moreover, $\abs{\vec{VQ}} = \abs{\vec{NQ}} \sin \Phi = \abs{\vec{NP}} \sin \Phi = \abs{\vec{r}} \sin \psi \sin \Phi$, where $\psi$ is the angle between $\vec{r}$ and $\hat{e}_{n}$. But this is just $\abs{\vec{r} \times \hat{e}_{n}} \sin \Phi$, and hence 
\begin{align}
    \vec{VQ} = (\vec{r} \times \hat{e}_{n}) \sin \Phi.
\end{align}
Thus, we substitute into $\vec{r}'$ to get
\begin{align}
    \vec{r}' = (\vec{r} \cdot \hat{e}_{n}) \hat{e}_{n} + (\vec{r} - (\vec{r} \cdot \hat{e}_{n}) \hat{e}_{n}) \cos \Phi + (\vec{r} \times \hat{e}_{n}) \sin \Phi = \vec{r} \cos \Phi + (\vec{r} \cdot \hat{e}_{n}) \hat{e}_{n} ( 1-\cos\Phi) + (\vec{r} \times \hat{e}_{n}) \sin \Phi.
\end{align}
Note that the vector $\vec{r}$ isn't special, and can be replaced by any vector undergoing the same rotational transformation for a finite angle. It is left as an exercise to the reader to find a relationship between $(\hat{e}_{n},\Phi)$ and the Euler angle rotations $(R_{\psi},R_{\theta},R_{\phi})$. Thus a `rotation formula' has been obtained.

\subsection{Composition of Rotations}

For an infinitesimal rotation, we have $x_{i}' = x_{i} + \sum_{j \neq i}\epsilon_{ij} x_{j}$, where $(\epsilon_{ij})$ is an infinitesimal matrix (a tensor). Writing in vector form,
\begin{align}
    \vec{r}' = \vec{r} + \tensor{\epsilon} \vec{r}.
\end{align}
We will drop the $\tensor{\epsilon}$ and use $\epsilon$ to denote the infinitesimal rotation tensor. For two such infinitesimal rotations, we have
\begin{align}
    (I+\epsilon_{1})(I+\epsilon_{2}) = I + \epsilon_{1} + \epsilon_{2} + \epsilon_{1}\epsilon_{2}, \quad (I+\epsilon_{2})(I+\epsilon_{1}) = I + \epsilon_{1} + \epsilon_{2} + \epsilon_{2}\epsilon_{1}.
\end{align}
Since a rotation is orthogonal, we have $(I+\epsilon)^{T} = (I+\epsilon)^{-1}$. Thus, up to first order, we have
\begin{align}
    (I+\epsilon)^{-1} = I - \epsilon, \quad (I+\epsilon)^{T} = I + \epsilon^{T} \implies \epsilon^{T} = -\epsilon.
\end{align}
Thus, $\epsilon$ is an antisymmetric tensor, and hence has only 3 independent components. We can write
\begin{align}
    \epsilon = \begin{pmatrix}
        0 & d\Omega_{3} & -d\Omega_{2} \\
        -d\Omega_{3} & 0 & d\Omega_{1} \\
        d\Omega_{2} & -d\Omega_{1} & 0
    \end{pmatrix}.
\end{align}

The infinitesimal change in position is
\begin{align}
    d\vec{r} = \vec{r}' - \vec{r} = \epsilon \vec{r} = \begin{pmatrix}
        0 & d\Omega_{3} & -d\Omega_{2} \\
        -d\Omega_{3} & 0 & d\Omega_{1} \\
        d\Omega_{2} & -d\Omega_{1} & 0
    \end{pmatrix} \begin{pmatrix}
        x_{1} \\ x_{2} \\ x_{3}
    \end{pmatrix}.
\end{align}
Thus we obtain $dx_{1} = x_{2}(d\Omega_{3}) - x_{3}(d\Omega_{2})$. Similarly, we obtain $dx_{2}$ and $dx_{3}$. Note that this is just the cross product of $\vec{r}$ with $\vec{d\Omega} = (d\Omega_{1},d\Omega_{2},d\Omega_{3})^{T}$ (note that $\vec{d\Omega} \neq d\vec{\Omega}$; it is not the infinitesimal change in a vector, but rather just notation). Thus, we have
\begin{align}
    d \vec{r} = \vec{r} \times \vec{d\Omega}, \quad \text{ or } \quad \vec{r}' = \vec{r} + \vec{r} \times \vec{d\Omega}.
\end{align}
This formula is only true for infinitesimal rotations. Let us verify this using the rotation formula. Recall
\begin{align}
    \vec{r}' = \vec{r} \cos \Phi + \hat{e}_{n}(\hat{e}_{n} \cdot \vec{r}) (1-\cos \Phi) + (\vec{r} \times \hat{e}_{n}) \sin \Phi.
\end{align}
For an infinitesimal rotation, we have $\Phi = d\Phi$, $\cos d\Phi = 1$ and $\sin d\Phi = d\Phi$. Thus, we have
\begin{align}
    \vec{r}' = \vec{r} + (\vec{r} \times \hat{e}_{n}) d \Phi = \vec{r} + \vec{r} \times \vec{d\Omega},
\end{align}
where we have taken $\vec{d\Omega} = \hat{e}_{n} d\Phi$. Thus, the two formulae agree. From this relation if we write $\hat{e}_{n} = (n_{1},n_{2},n_{3})^{T}$, we obtain
\begin{align}
    \epsilon = \begin{pmatrix}
        0 & n_{3} & -n_{2} \\
        -n_{3} & 0 & n_{1} \\
        n_{2} & -n_{1} & 0
    \end{pmatrix}d\Phi.
\end{align}

\section{Rate of Change}
\textit{October 22nd.}

COnsider the infinitesimal change $\vec{dG}$ of any vector $\vec{G}$ under an infinitesimal rotation $\vec{d\Omega}$. We have
\begin{align}
    (\vec{dG})_{\text{space}} = (\vec{dG})_{\text{body}} + \vec{d\Omega} \times \vec{G}.
\end{align}
Taking the time derivative, we obtain
\begin{align}
    \frac{d\vec{G}}{dt}\Big|_{\text{space}} = \frac{d\vec{G}}{dt}\Big|_{\text{body}} + \vec{\omega} \times \vec{G}
\end{align}
where $\vec{\omega}$ is the instantaneous angular velocity vector. This is a sort of hand-wavy derivation, but it can be made more rigorous. Suppose the body-set is rotated with respect to the space-set by an orthogonal matrix $A(t)$. Let $[A(t)]_{ij} = a_{ij}$. Then without the loss of generality, let at time $t$ ($G_{i}'=G_{i}$), we have
\begin{align}
    G_{i} = \sum_{j} (A^{-1})_{ij} G_{j}' = \sum_{j} a_{ji} G_{j}' \implies dG_{i} = \sum_{j} a_{ji} dG_{j}' + \sum_{j} G_{j}' da_{ji} = dG_{i}' + \sum_{j} G_{j}' da_{ji}.
\end{align}
With $-\epsilon = \begin{pmatrix}
    0 & -d\Omega_{3} & d\Omega_{2} \\
    d\Omega_{3} & 0 & -d\Omega_{1} \\
    -d\Omega_{2} & d\Omega_{1} & 0
\end{pmatrix}$, we have
\begin{align}
    -\epsilon_{ij} = -\epsilon_{ijk} d\Omega_{k} = \epsilon_{ikj} d\Omega_{k} \implies da_{ji} = \epsilon_{ikj} d\Omega_{k}
\end{align}
since $da_{ji} = (\epsilon^{T})_{ij} = (-\epsilon)_{ij}$. Note that $\epsilon_{ij}$ denotes the components of the antisymmetric tensor, while $\epsilon_{ijk}$ is the Levi-Civita symbol. Thus,
\begin{align}
    dG_{i} = dG_{i}' + \sum_{j,k} G_{j}' \epsilon_{ikj} d\Omega_{k} = dG_{i}' + (\vec{d\Omega} \times \vec{G}')_{i}.
\end{align}
Note that the vector $\vec{G}$ did not really play a role; any vector would have worked. Taking the time derivative, we obtain the desired result. For example, putting in the position vector $\vec{r}$, we have
\begin{align}
    \vec{v}_{s} = \vec{v}_{b} + \vec{\omega} \times \vec{r}.
\end{align}
We take it a step further to derive the accelration---
\begin{align}
    \vec{a}_{s} = \frac{d\vec{v}_{s}}{dt}\Big|_{s} = \frac{d\vec{v}_{s}}{dt}\Big|_{b} + \vec{\omega} \times \vec{v}_{s} = \frac{d\vec{v}_{b}}{dt}\Big|_{b} + \vec{\omega} \times (\vec{v}_{b} + \vec{\omega} \times \vec{r}) = \vec{a}_{b} + 2\vec{\omega} \times \vec{v}_{b} + \vec{\omega} \times (\vec{\omega} \times \vec{r}).
\end{align}
Define $\vec{F}_{s} = m\vec{a}_{s}$ and $\vec{F}_{b} = m\vec{a}_{b}$ to get
\begin{align}
    \vec{F}_{b} = \vec{F}_{s} - m(\vec{\omega} \times (\vec{\omega} \times \vec{r})) - 2m(\vec{\omega} \times \vec{v}_{b}).
\end{align}
The first of these addition terms is called the \eax{centrifugal force}, while the second is called the \eax{Coriolis force}. These are the only two pseudoforces that arise in a rotating frame with constant angular velocity. 

Recall the angular momentum equation, $\vec{L} = \tensor{I} \vec{\omega} \implies \sum_{\beta} I_{\alpha \beta} \omega_{\beta} = L_{\alpha}$. The rotational kinetic energy takes the form $T = \frac{1}{2} \vec{\omega} \cdot (\tensor{I} \vec{\omega}) = \sum_{\alpha,\beta} \frac{1}{2} I_{\alpha \beta} \omega_{\alpha} \omega_{\beta}$. But note that the inertia tensor is symmetric, and hence can be diagonalized. Thus, there exists a coordinate system in which $\tensor{I}$ is diagonal, i.e., $I_{\alpha \beta} = 0$ for $\alpha \neq \beta$. The corresponding axes are called the \eax{principal axes of inertia}. Diagonalizing $\tensor{I}$ gives
\begin{align}
    S^{-1} \tensor{I} S = \begin{pmatrix}
        I_{1} & 0 & 0 \\
        0 & I_{2} & 0 \\
        0 & 0 & I_{3}
    \end{pmatrix}
\end{align}
where $I_{i}$'s are called the \eax{principal moments of inertia}. Corresponding to these principal axes are unit vectors $\hat{e}_{1},\hat{e}_{2},\hat{e}_{3}$. In this coordinate system, we simply have $L_{1} = I_{1}\omega_{1}$, and so on. Note that the principal axes are orthogonal, since $S$ is an orthogonal matrix. Here we have taken $\vec{\omega} = \omega_{1} \hat{e}_{1} + \omega_{2} \hat{e}_{2} + \omega_{3} \hat{e}_{3}$ and $\vec{L} = L_{1} \hat{e}_{1} + L_{2} \hat{e}_{2} + L_{3} \hat{e}_{3}$. Moving forward, if $\vec{L}$ plays the role of $\vec{G}$, we get
\begin{align}
    \vec{\tau}_{\text{ext}} = \frac{d\vec{L}}{dt}\Big|_{s} = \frac{d\vec{L}}{dt}\Big|_{b} + \vec{\omega} \times \vec{L} \implies \frac{dL_{i}}{dt}\Big|_{b} + \epsilon_{ijk} \omega_{j} L_{k} = \tau_{\text{ext},i}.
\end{align}
With $L_{i} = I_{i} \omega_{i}$, we have
\begin{align}
    \tau_{\text{ext},i} = I_{i} \frac{d\omega_{i}}{dt} + \epsilon_{ijk} (I_{k} \omega_{k}) \implies \tau_{\text{ext},1} = I_{1}\dot{\omega}_{1} + \omega_{2}\omega_{3}(I_{3}-I_{2}) \quad \text{ and cyclic permutations.}
\end{align}

\noindent\textit{October 27th.}

For a generalized coordinate $\theta$, oscillations are entirely described by the small angle approximation $\cos \theta \approx 1$ and $\sin \theta \approx \theta$, $\tan \theta \approx \theta$. One then relates the equation of motion as $\ddot{\theta} \approx - \alpha \theta$, where $\alpha > 0$ is a constant. The solution to this equation is $\theta(t) = \theta_{0} \cos(\sqrt{\alpha} t + \phi)$, where $\theta_{0}$ and $\phi$ are constants determined by initial conditions.

\section{Heavy Spinning Top}

Let $\phi,\theta,\psi$ be the Euler angles describing the orientation of the top. Then $\dot{\psi}$ represents the rotation of top about its own body $z_{b}$-axis (the instantaneous spin), $\dot{\phi}$ represents the precession of the body $z_{b}$-axis about the space $z_{s}$-axis, and $\dot{\theta}$ represents the nutation (the oscillation of the angle between the body $z_{b}$-axis and the space $z$-axis). In the body axes, clearly $I_{3} \neq I_{1} = I_{2} = I$ (say). We have assumed $\dot{\psi} \gg \dot{\phi}, \dot{\theta}$. The equations of motion are then
\begin{align}
    I_{1} \dot{\omega}_{1} - \omega_{2}\omega_{3} (I-I_{3}) &= \tau_{1}, \\
    I_{2} \dot{\omega}_{2} - \omega_{3}\omega_{1} (I_{3}-I) &= \tau_{2}, \\
    I_{3} \dot{\omega}_{3} &= \tau_{3}.
\end{align}
The kinetic energy is given by
\begin{align}
    T = \frac{1}{2}I(\omega_{1}^{2} + \omega_{2}^{2}) + \frac{1}{2} I_{3} \omega_{3}^{2}.
\end{align}
We proceed to relate these with the Euler angles. Note that $\vec{\omega}_{\phi}$ pointing along $z_{s}$ is given by the projection of $\dot{\phi} \hat{e}_{z_{s}}$ in the $x_{b}y_{b}$-plane, which is $\dot{\phi} \sin \theta$. So, $(\vec{\omega}_{\phi})_{x_{b}} = \dot{\phi} \sin \theta \sin \psi$, $(\vec{\omega}_{\phi})_{y_{b}} = \dot{\phi} \sin \theta \cos \psi$, and $(\vec{\omega}_{\phi})_{z_{b}} = \dot{\phi} \cos \theta$. A similar analysis for $\psi$ shows that $(\vec{\omega}_{\psi})_{x_{b}} = 0 = (\vec{\omega}_{\psi})_{y_{b}}$ and $(\vec{\omega}_{\psi})_{z_{b}} = \dot{\psi}$. Finally, for $\theta$, $\vec{\omega}_{\theta}$ points along $\xi$ (the line of nodes), so $(\vec{\omega}_{\theta})_{x_{b}} = \dot{\theta} \cos \psi$, $(\vec{\omega}_{\theta})_{y_{b}} = -\dot{\theta} \sin \psi$, and $(\vec{\omega}_{\theta})_{z_{b}} = 0$. Adding these up, we obtain

\begin{align}
    \omega_{x_{b}} &= \dot{\phi} \sin \theta \sin \psi + \dot{\theta} \cos \psi,\\
    \omega_{y_{b}} &= \dot{\phi} \sin \theta \cos \psi - \dot{\theta} \sin \psi,\\
    \omega_{z_{b}} &= \dot{\phi} \cos \theta + \dot{\psi}.
\end{align}
$z_{b}$ is an axis of symmetry, so it must be a principal axis. Since we are free to choose the other two axes, we choose them to be $y_{b}$ and $z_{b}$. Thus, we have $I_{1} = I_{2} = I$ and $I_{3}$ as before, with the indices corresponding to $x_{b},y_{b},z_{b}$. The kinetic energy is then
\begin{align}
    T = \frac{1}{2}I(\omega_{x_{b}}^{2} + \omega_{y_{b}}^{2}) + \frac{1}{2} I_{3} \omega_{z_{b}}^{2} = \frac{I}{2}(\dot{\theta}^{2} + \dot{\phi}^{2} \sin^{2} \theta) + \frac{I_{3}}{2} (\dot{\psi} + \dot{\phi} \cos \theta)^{2}.
\end{align}
The potential energy is given by $U = Mgl \cos \theta$, where $l$ is the distance of the center of mass from the fixed point $O$. Since $\phi$ and $\psi$ do not appear in $L = T-U$, they are cyclic coordinates, and their generalized momenta $p_{\phi}$ and $p_{\psi}$ are conserved. Thus,
\begin{align}
    p_{\psi} = \frac{\partial L}{\partial \dot{\psi}} = I_{3}(\dot{\psi} + \dot{\phi} \cos \theta) = I_{3} \omega_{z_{b}} = I_{1}a
\end{align}
where $a$ is a constant (this quantity $p_{\psi}$ is conserved, so is $\omega_{z_{b}}$; hence such an $a$ is possible). Similarly,
\begin{align}
    p_{\phi} = \frac{\partial L}{\partial \dot{\phi}} = (I \sin^{2}\theta + I_{3} \cos^{2} \theta) \dot{\phi} + (I_{3} \cos \theta) \dot{\psi} = I b.
\end{align}
where $b$ is a constant. Also, $E = T+U$ is also conserved, which is
\begin{align}
    E = \frac{I}{2}(\dot{\theta}^{2} + \dot{\phi}^{2} \sin^{2} \theta) + \frac{1}{2} I_{3} \omega_{z_{b}}^{2} + Mgl \cos \theta.
\end{align}
We eliminate the other two coordinates to get an equation in $\theta$ alone. To eliminate $\dot{\psi}$, using $I_{3}\omega_{z_{b}} = I_{1} a$, we have $\dot{\psi} = \frac{I}{I_{3}} a - \dot{\phi} \cos \theta$. Substituting this into the expression for $p_{\phi}$, we get $\dot{\phi} = \frac{b-a \cos \theta}{\sin^{2} \theta}$. $E' = E - \frac{1}{2} I_{3} \omega_{z_{b}}^{2}$ is also conserved, and is given by
\begin{align}
    E' = E - \frac{1}{2} I_{3} \omega_{z_{b}}^{2} = \frac{1}{2} I \dot{\theta}^{2} + \frac{1}{2} I \frac{(b-a\cos\theta)^{2}}{\sin^{2}\theta} + Mgl \cos \theta = T_{\theta} + V_{\theta}
\end{align}
where $T_{\theta} = \frac{1}{2} I \dot{\theta}^{2}$ and $V_{\theta} = \frac{1}{2} I \frac{(b-a\cos\theta)^{2}}{\sin^{2}\theta} + Mgl \cos \theta$. Thus,
\begin{align}
    \alpha &= \frac{2E'}{I} = \dot{\theta}^{2} + \frac{(b-a\cos\theta)^{2}}{\sin^{2}\theta} + \frac{2Mgl}{I} \cos \theta \\
    \implies \alpha \sin^{2} \theta = \dot{\theta}^{2} \sin^{2} \theta + (b-a\cos\theta)^{2} + \beta \cos \theta \sin^{2} \theta,
\end{align}
where $\beta = \frac{2Mgl}{I}$. Setting $u = \cos \theta$ and $\dot{u}^{2} = \sin^{2} \theta \dot{\theta}^{2}$, we have
\begin{align}
    \alpha (1-u^{2}) = \dot{u}^{2} + (b-au)^{2} + \beta u (1-u^{2}) \implies \dot{u}^{2} = (1-u^{2})(\alpha-\beta u) - (b-au)^{2}.
\end{align}
The solution to this equation is given by the inverse function of the following:
\begin{align}
    t = \int_{0}^{t} dt = \int_{u(0)}^{u(t)} \frac{du}{((1-u^{2})(\alpha - \beta u) - (b-au)^{2})^{1/2}}.
\end{align}
This is the required equation, an elliptic integral which is unsolvable in elementary functions. Let us perform some analysis on this integral. Let $f(u) = \dot{u}^{2}$. Note that $-1 \leq u \leq 1$ since $u = \cos \theta$. In the limiting case, when $u = \pm 1$, we have $\dot{u}^{2} = f(u) = -(b \mp a)^{2}$ (we have assumed $\beta > 0$). But $f(u)$ must be a non-negative quantity, so this limiting case does not occur. We note down that $f(\pm 1) < 0$. The turning points of the motion occur when $\dot{u} = 0$, i.e., $f(u) = 0$. Thus, $f(u)$ has a non-physical root beyond $1$ but has two physical roots in $(-1,1)$, say $u_{1} < u_{2}$ (let us assume $\theta < \pi/2$ so that $u > 0$). This implies that the associated $\theta$s are $\theta_{2} < \theta_{1}$, and the motion is between these two angles. Note that at these turning points, $\dot{\theta} = 0$, so the top is momentarily stationary in the nutation angle. Since we require $f(u)$ to be non-negative, we have that $u$ must be between $u_{1}$ and $u_{2}$. Thus, the motion is oscillatory between these two angles, which is nutation.
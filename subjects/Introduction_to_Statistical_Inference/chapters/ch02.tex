\chapter{POINT ESTIMATION}
\textit{August 8th.}

We begin with a definition.

\begin{definition}
    A \eax{point estimator} is a function $W$ of the sample, mapping into the parameter space $\Theta$ of the parameter $\theta$ of interest. The value $W(X)$ is called a \eax{point estimate} of $\theta$.
\end{definition}

Various methods of point estimation are discussed in this chapter, including the method of moments, maximum likelihood estimation, and Bayes estimation.

\section{Estimators}
\subsection{Method of Moments}
Let $X_{1},\ldots,X_{n}$ be a sample from a population with a probability distribution function and probability density function. The \eax{method of moments} estimators are found by equating the first $k$ sample moments to the corresponding $k$ population moments and solving the resulting system of simultaneous equations. Here, the $k^{\text{th}}$ sample moment and the $k^{\text{th}}$ population moment are given as
\begin{align}
    m_{k} = \frac{1}{n} \sum_{j=1}^{n} X_{j}^{k} \qquad \text{and} \qquad \mu_{k}' = E[X^{k}]
\end{align}
respectively. For $\mu_{j}' = \mu_{j}'(\theta_{1},\ldots,\theta_{i})$, with $1 \leq j \leq k$, the estimators are obtained by solving the equations
\begin{align}
    m_{j} = \mu_{j}'(\theta_{1},\ldots,\theta_{i}) \qquad \text{for } 1 \leq j \leq k.
\end{align}

\subsection{Maximum Likelihood Estimators}

Let $X_{1},\ldots,X_{n}$ be an independent and identically distributed sample from a population with a probability distribution/mass function $f(\underline{x} \mid \theta_{1},\ldots,\theta_{k})$. The \eax{likelihood function} is defin as
\begin{align}
    L(\underline{\theta} \mid \underline{x}) = L(\theta_{1},\ldots,\theta_{k} \mid x_{1},\ldots,x_{n}) \defeq \prod_{j=1}^{n} f(x_{j} \mid \theta_{1},\ldots,\theta_{k}).
\end{align}

\begin{definition}
    For each sample points, let $\overline{\theta}(\underline{x})$ be a parameter value at which $L(\underline{\theta} \mid \underline{x})$ attains its maxima as a function of $\underline{\theta}$, with $\underline{x}$ held fixed. A \eax{maximum likelihood estimator} (MLE) of the parameter $\theta$ based on a sample $\underline{X}$ is $\hat{\theta}(\underline{X})$.
\end{definition}

\begin{remark}
    \begin{enumerate}
        \item By its contraction, the ranges of the MLE concludes with the range of the parameter.
        \item The MLE is the parameter points for which the observation sample is most likely.
    \end{enumerate}
\end{remark}

If the likelihood function is differentiable in $\theta_{i}$, then possible candidates for the MLE are the values of $(\theta_{1},\ldots,\theta_{k})$ that solve $\frac{\partial}{\partial \theta_{i}}L(\underline{\theta} \mid \underline{x}) = 0$ for $1 \leq i \leq k$. Moreover, we must have $\frac{\partial^{2}}{\partial^{2}\theta_{i}}L(\underline{\theta} \mid \underline{x}) < 0$ for the solution to be a maximum.

\begin{remark}
    \begin{enumerate}
        \item The solutions to the above equation are the only possible candidates for the MLE since the first derivative being zero is only a necessary condition for the maximum and not sufficient.
        \item The zeroes of the first derivatives only locate extreme points in the interior of the domain of the function.
        \item If the extrema occurs on te boundary of the domain, then the first derivative may not be zero. Thus the boundary must be checked separately.
        \item The points where the first derivatives are zero may be local/global maxima or minima. 
    \end{enumerate}
\end{remark}


\begin{example}
    Let us take the example of the binomial distribution, $X \sim \Bin (n,p)$. The likelihood function is given by $L(p \mid x) = \binom{n}{x} p^{x} (1-p)^{n-x}$, where $0 < p < 1$. We have
    \begin{align}
        \frac{\partial}{\partial p} L(p \mid x) = \frac{dL}{dp}(p \mid x) = 0.
    \end{align}
    This derivative is hard to compute directly. So we take the logarithm (an increasing function) of the likelihood function, and then differentiate.
    \begin{align}
        \implies \frac{d}{dp} \log L(p \mid x) = \frac{x}{p} - \frac{n-x}{1-p} = 0.
    \end{align}
    This gives us $p = \frac{x}{n}$, with the second derivative less than zero, confirming that this is a maximum. Thus, the MLE of $p$ is $\hat{p} = \frac{X}{n}$.
\end{example}

The method in this example is known as \eax{log likelihood estimation}. It is often easier to work with the log likelihood function, especially when dealing with products.

\begin{remark}
    \begin{enumerate}
        \item The MLE may not exist at all or may not be unique.
        \item If $\hat{\theta}$ is the MLE of $\theta$, then $\rho(\hat{\theta})$ is the MLE of $\rho(\theta)$ for any function $\rho$.
    \end{enumerate}
\end{remark}

The following is a vital and important result.
\begin{theorem}
    The MLE depends on $\underline{x}$ only through the sufficient statistic $T(\underline{x})$.
\end{theorem}
\begin{proof}
    We have $L(\theta \mid \underline{x}) = f(\underline{x} \mid \theta) = g(T(\underline{x}),\theta) h(\underline{x})$. Therefore, we have
    \begin{align}
        L(\hat{\theta}(\underline{x}) \mid \underline{x}) = \max_{\theta} g(T(\underline{x}),\theta) h(\underline{x}).
    \end{align}
    Since $h(\underline{x}) > 0$, and does not depend on $\theta$, we must have
    \begin{align}
        L(\hat{\theta}(\underline{x}) \mid \underline{x}) = h(\underline{x}) \max_{\theta} g(T(\underline{x}),\theta)
    \end{align}
    where the maximization is on the part that involves $\underline{x}$ through $T(\underline{x})$ only.
\end{proof}

\subsection{One Parameter Exponential Family in Natural Form}
Recall that the usual density form of the exponential family was given as
\begin{align}
    f(x \mid \theta) = \exp(c(\theta)T(x) + d(\theta) + s(x))\1_{A}(x).
\end{align}
Define $\eta = c(\theta)$ for $\theta \in \Theta$, and let $\Gamma = \{\eta \mid \eta = c(\theta), \theta \in \Theta\}$. We then have
\begin{align}
    f^{\ast}(x \mid \eta) = \exp(\eta T(x) + d_{0}(\eta) + s(x))\1_{A}(x)
\end{align}
where $d_{0}(\eta) = d(c^{-1}(\eta))$ if $c$ is one-one. Moreover,
\begin{align}
    1 &= \int_{A}f^{\ast}(x \mid \eta)dx = \int_{A} \exp(\eta T(x) + d_{0}(\eta) + s(x))dx = \exp(d_{0}(\eta)) \int_{A} \exp(\eta T(x)+s(x))dx \\
    \implies d_{0}(\eta) &= -\log\left( \int_{A} \exp(\eta T(x) + s(x))dx\right). 
\end{align}

\begin{theorem}
    If $X$ has density of the form $f(x \mid \eta) = \exp(\eta T(x) + d_{0}(\eta) + s(x))\1_{A}(x)$ and $\eta$ is an interior points of $H \defeq \{\eta \mid \abs{d_{0}(\eta)} < \infty\}$, then the moment generating function of $T(X)$ exists and is given by
    \begin{align}
        \varphi(s) = E[\exp(sT(X))].
    \end{align}
    Also,
    \begin{align}
        E[T(X)] = -\frac{d}{d\eta} d_{0}(\eta), \qquad \text{ and } \qquad \Var(T(X)) = -\frac{d^{2}}{d\eta^{2}} d_{0}(\eta).
    \end{align}
\end{theorem}

\begin{theorem}
    Let $\{P_{\theta} \mid \theta \in \Theta\}$ be a one parameter exponential family with density $f(x \mid \theta) = \exp(c(\theta)T(x) + d(\theta) + s(x))\1_{A}(x)$ and let $c$ be the interior of $C = \{c(\theta) \mid \theta \in \Theta\}$. Also suppose $\theta \mapsto c(\theta)$ is one-one. If the equation
    \begin{align}
        E_{\theta}[T(X)] = T(x)
    \end{align}
    has a solution $\hat{\theta}(x)$ for which $c(\hat{\theta}(x)) \in C$, then $\hat{\theta}(x)$ is the unique MLE of $\theta$.
\end{theorem}
\begin{proof}
    Since $\theta \mapsto c(\theta)$ is one-one, maximizing the likelihood over $\theta$ is maximizing over $\eta = c(\theta)$. Hence, consider the natural parametrization
    \begin{align}
        f(x \mid \eta) = \exp(\eta T(x) + d_{0}(\eta) + s(x))\1_{A}(x) \text{ for } \eta \in H.
    \end{align}
    $L(\eta \mid x) = \eta T(x) + d_{0}(\eta) + s(x)$, if $x \in A$, is the log likelihood function. Also,
    \begin{align}
        \frac{\partial}{\partial \eta} L(\eta \mid x) = T(x) + d_{0}'(\eta) \text{ and } \frac{\partial^{2}}{\partial \eta^{2}} L(n \mid x) = d_{0}''(\eta).
    \end{align}
    Therefore, we get $\frac{\partial}{\partial \eta} L(\eta \mid x) = T(x) - E_{\eta}[T(x)] = 0$ implying that $E_{\eta}[T(X)] = T(x)$. Now, $\frac{\partial^{2}}{\partial \eta^{2}} L(\eta \mid x) < 0$ so that $L$ is strictly concave. Then we get a unique maxima at $\hat{\eta}(x)$ for which $E_{\eta}[T(X)]_{\eta = \hat{\eta}(x)} = T(x)$.
\end{proof}
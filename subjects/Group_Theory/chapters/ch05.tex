\chapter{MATRIX GROUPS}

We begin with $GL_{n}(\R) \subseteq M_{n}(\R)$, the set of all invertible $n \times n$ matrices with real entries. If we instead look at $M_{n}(\R)$ as $\R^{n^{2}}$ instead, then $GL_{n}(\R)$ is the set of all matrices such that the polynomial $\det(A) \neq 0$. Thus $GL_{n}(\R)$ is the complement of the zero set of a polynomial, and hence is open in $\R^{n^{2}}$. Note that $GL_{n}$ acts on $M_{n}$ via matrix conjugation as
\begin{align}
    GL_{n} \times M_{n} \to M_{n}, \quad (P,A) \mapsto PAP^{t}.
\end{align}
Let us find the stabilizer of the identity. We have
\begin{align}
    \Stab (I_{n}) = \{P \in GL_{n} \mid PP^{t} = I_{n}\} = O_{n},
\end{align}
the \eax{orthogonal group}. Sylvester's law of inertia states that $GL_{n}(\R)$ orbits of real symmetric matrices consist of a unique matrix of the form $\diag(I_{p},-I_{m},0_{z})$ where $p+m+z = n$. One may also look at $O_{n-1,1}(\R)$ is defined as $\Stab (I_{n-1,1})$ where $I_{n-1,1} = \diag(I_{n-1},-1)$. The group $O_{3,1}(\R)$ is also called the \eax{Lorentz group}.\\ \\
\textit{October 9th.}

\begin{definition}
    A (complex) matrix $A$ is termed a \eax{normal matrix} if $AA^{\ast} = A^{\ast}A$. It is termed a \eax{Hermitian matrix} if $A = A^{\ast}$, and a \eax{unitary matrix} if $AA^{\ast} = I$.
\end{definition}

\begin{theorem}[The \eax{spectral theorem}]
    If $A$ is a normal matrix, then there exists a unitary matrix $P$ such that $PAP^{\ast}$ is diagonal.
\end{theorem}

Let $A$ be a $n \times n$ real matrix. Then $A$ defines a bilinear form on $\R^{n}$, that is, a map $\R^{n} \times \R^{n} \to \R$ is given as $(x,y) \mapsto x^{t}Ay$. The map is typically denoted by $\ip{\cdot,\cdot}_{A}$. If $A$ is symmetric, then $\ip{\cdot,\cdot}_{A}$ is a symmetric bilinear form. If $A$ is skew-symmetric, then $\ip{\cdot,\cdot}_{A}$ is a skew-symmetric bilinear form. Moreover if $A$ is non-singular, then $\ip{\cdot,\cdot}_{A}$ is non-degenerate. 

Just like before, $GL_{n}(\C)$ acts on $M_{n}(\C)$ via $PAP^{\ast}$, the conjugation action. In this case, we have
\begin{align}
    \Stab(I_{n}) = \{P \in GL_{n}(\C) \mid PP^{\ast} = I_{n}\} = U_{n},
\end{align}
the \eax{unitary group}. We also have $SL_{n} = \{A \mid \det A = 1\}$, the special linear group.

\begin{definition}
    A subset $X \subseteq \R^{N}$ is called a $n$-dimensional \eax{manifold} if for every $x \in X$, there exists an open neighbourhood $U \subseteq \R^{N}$ of $x$, which is homeomorphic to $\R^{n}$. That is, there exists a map $\varphi : U \to \R^{n}$ such that $\varphi$ is continuous, bijective, and $\varphi^{-1}$ is also continuous.
\end{definition}

The group $SO_{n} = SL_{n} \cap O_{n}$ is termed the special orthogonal group. The group $SU_{n} = SL_{n} \cap U_{n}$ is termed the special unitary group.

\begin{example}
    We have $GL_{1}(\R) = \R^{\ast}$, $GL_{1}(\C) = \C^{\ast} \cong \R^{2} \setminus \{0\}$. One can also infer
    \begin{align}
        SL_{1}(\R) = \{1\} = SL_{1}(\C), \quad O_{1}(\R) = \{1,-1\} = O_{1}(\C),\quad U_{1}(\C) = S_{1},\quad SO_{1}(\R) = SU_{1}(\C) = \{1\}.
    \end{align}
\end{example}

\begin{example}
    $GL_{2}(\R)$ is a $4$-dimensional manifold since it is an open subset of $\R^{4}$. $SL_{2}(\R)$ is a $3$-dimensional manifold since it is the zero set of the polynomial $\det(A) - 1$. $SO_{2}(\R) = SL_{2}(\R) \cap O_{2}(\R)$ is the set of matrices of the form
    \begin{align}
        SO_{2}(\R) = \left\{ \begin{pmatrix}
            a & b \\ c & d
        \end{pmatrix} \Big| AA^{t} = A^{t}A = I_{2},\; \det A = 1 \right\} = \left\{ \begin{pmatrix}
            x_{1} & -x_{2} \\ x_{2} & x_{1}
        \end{pmatrix} \Big| (x_{1},x_{2}) \in S^{1}\right\}
    \end{align}
\end{example}

\begin{proposition}
    There is an isomorphism $U_{1}(\C) \to SO_{2}(\R)$, which is also a homeomorphism.
\end{proposition}
\begin{proof}
    Simply send $z = x_{1}+ix_{2} \in U_{1}(\C)$ to the matrix $\begin{pmatrix}
        x_{1} & -x_{2} \\ x_{2} & x_{1}
    \end{pmatrix} \in SO_{2}(\R)$. One may verify that this map is bijective, continuous, homomorphic, with a continuous inverse as well.
\end{proof}

As above, linear groups of dimension $3$ include $SU_{2}(\C)$, $SO_{3}(\R)$, and $SL_{3}(\R)$. One can show that $SU_{2}(\C)$ is exactly the set of matrices $\begin{pmatrix}
    a & b \\ -\bar{b} & \bar{a}
\end{pmatrix}$ with $\abs{a}^{2} + \abs{b}^{2} = 1$. From here, we get
\begin{align}
    SU_{2}(\C) \cong \{(a,b) \in \C^{2} \mid \abs{a}^{2} + \abs{b}^{2} = 1\} \cong \{(x_{1},x_{2},x_{3},x_{4}) \in \R^{4} \mid x_{1}^{2} + x_{2}^{2} + x_{3}^{2} + x_{4}^{2} = 1\} = S^{3}.
\end{align}
Thus, $S^{3}$ can be given a group structure via relating by $SU_{2}(\C)$. Note that the binary operator induced is also continuous. $S^{2}$, however, cannot be given such a continuous group structure. There is a conjecture that $S^{n}$ for $n \geq 4$ cannot be given a continuous group structure.

\begin{definition}
    In $S^{3}$, we define latitudes, for $-1 < c < 1$, to be the surface cut out by $x_{1} = c$ on $x_{1}^{2} + x_{2}^{2} + x_{3}^{2} + x_{4}^{2} = 1$. This surface turns out to be a sphere of radius $1-c^{2}$.
\end{definition}
Thus the latitudes of $S^{3}$ are homeomorphisms of $S^{2}$. The north pole $x_{1} = 1$ tells us $I \in SU_{2}$, and the south pole $x_{1} = -1$ tells us $-I \in SU_{2}$.

\textit{October 14th.}

\begin{proposition}
    The latitudes of $S^{3}$ are conjugancy classes in $SU_{2}$. Conversely, other than $\pm I$, all conjugacy classes are latitudes.
\end{proposition}

\begin{proof}
    The centre of $SU_{2}$ is $\{\pm I\}$. Let $P = \begin{pmatrix}
        a & b \\ -\bar{b} & \bar{a}
    \end{pmatrix} \in SU_{2}$, that is, $\abs{a}^{2} + \abs{b}^{2} = 1$. The characteristic polynomial of $P$ is given as
    \begin{align}
        \chi_{P}(t) = (t-a)(t-\bar{a}) + b \bar{b} = t^{2} - 2\Re(a) t + 1.
    \end{align}
    This is a real polynomial; if $\lambda$ is a root, then so is $\bar{\lambda}$. By spectral theorem, $P$ is conjugate to $\begin{pmatrix}
        \lambda & 0 \\ 0 & \bar{\lambda}
    \end{pmatrix}$ where $\lambda + \bar{\lambda} = 2\Re(a)$, via a unitary $Q$. If $Q \in SU_{2}$, then done. Otherwise, let $\delta = \det Q$ and let $\epsilon = \sqrt{\delta}$. Then one can verify that $\bar{\epsilon} Q \in SU_{2}$, and \begin{align}
        (\bar{\epsilon} Q) P (\bar{\epsilon} Q)^{\ast} = Q P Q^{\ast} = \begin{pmatrix}
            \lambda & 0 \\ 0 & \bar{\lambda}
        \end{pmatrix}.
    \end{align}
    Note that $\tr P = \lambda + \bar{\lambda} = 2\Re(a)$, so $\Re(a)$ determines the conjugacy class. The latitudes are also determined by $x_{1} = c$, which is also $\Re(a)$. Thus the latitudes are exactly the conjugacy classes.
\end{proof}

\begin{definition}
    In $S^{3}$, we define longitudes to be the intersection of $S^{3}$ with the hyperplanes containing the $x_{1}$-axis (specifically, containing $\pm I_{3}$). These are `circles' going through the north and south poles.
\end{definition}

Let $P \in SU_{2}$ be different from $\pm I$. Then $P$ and $I$ are linearly independent vectors in $\R^{4}$. Then they span a $2$-dimensional subspace $V$ of $\R^{4}$ and $V \cap S^{3}$ is the unique longitude containing $P$. Consider the longitude $S^{3} \cap W$ where $W = \{(x_{1},x_{2},x_{3},x_{4}) \mid x_{3} = x_{4} = 0\}$. That is, $T = SU_{2} \cap W = \left\{ \begin{pmatrix}
    a & 0 \\ 0 & \bar{a}
\end{pmatrix} \Big| \abs{a}^{2} = 1 \right\}$. Note that $T$ is a subgroup of $SU_{2}$ and $T \cong U_{1}(\C)$.

\begin{proposition}
    $T \sbg SU_{2}$. Moreover, longitudes of $SU_{2}$ are exactly the conjugates of the subgroup $T$.
\end{proposition}
\begin{proof}
    $T$ being a subgroup is clear. Let $T'$ be a different longitude with $P' \in T'$ and $P' \neq \pm I$. Then $-1 < \tr P' < 1$. Let $P \in T$ such that $\tr P = \tr P'$. By the previous proposition, there exists $Q \in SU_{2}$ such that $QPQ^{\ast} = P'$. We now show that $QTQ^{\ast}$ is a longitude containing $P'$. Let $Q = \begin{pmatrix}
        a & b \\ -\bar{b} & \bar{a}
    \end{pmatrix}$ for some $a,b \in \C$ with $\abs{a}^{2} + \abs{b}^{2} = 1$, and let $T = \left\{ \begin{pmatrix}
        w_{1} + iw_{2} & 0 \\ 0 & w_{1} - iw_{2}
    \end{pmatrix} \Big| w_{1},w_{2} \in \R, w_{1}^{2} + w_{2}^{2} = 1 \right\}$. If we take such a matrix $W \in T$ with $z = w_{1}+iw_{2}$, then computing $QWQ^{\ast}$ gives a plane viewed in $\R^{4}$. Thus, $QTQ^{\ast} = SU_{2} \cap QWQ^{\ast}$ since $T = SU_{2} \cap W$.
\end{proof}
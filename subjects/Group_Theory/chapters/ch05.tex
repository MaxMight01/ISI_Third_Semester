\chapter{MATRIX GROUPS}

\section{Spheres and Linear Groups}

We begin with $GL_{n}(\R) \subseteq M_{n}(\R)$, the set of all invertible $n \times n$ matrices with real entries. If we instead look at $M_{n}(\R)$ as $\R^{n^{2}}$ instead, then $GL_{n}(\R)$ is the set of all matrices such that the polynomial $\det(A) \neq 0$. Thus $GL_{n}(\R)$ is the complement of the zero set of a polynomial, and hence is open in $\R^{n^{2}}$. Note that $GL_{n}$ acts on $M_{n}$ via matrix conjugation as
\begin{align}
    GL_{n} \times M_{n} \to M_{n}, \quad (P,A) \mapsto PAP^{t}.
\end{align}
Let us find the stabilizer of the identity. We have
\begin{align}
    \Stab (I_{n}) = \{P \in GL_{n} \mid PP^{t} = I_{n}\} = O_{n},
\end{align}
the \eax{orthogonal group}. Sylvester's law of inertia states that $GL_{n}(\R)$ orbits of real symmetric matrices consist of a unique matrix of the form $\diag(I_{p},-I_{m},0_{z})$ where $p+m+z = n$. One may also look at $O_{n-1,1}(\R)$ is defined as $\Stab (I_{n-1,1})$ where $I_{n-1,1} = \diag(I_{n-1},-1)$. The group $O_{3,1}(\R)$ is also called the \eax{Lorentz group}.\\ \\
\textit{October 9th.}

\begin{definition}
    A (complex) matrix $A$ is termed a \eax{normal matrix} if $AA^{\ast} = A^{\ast}A$. It is termed a \eax{Hermitian matrix} if $A = A^{\ast}$, and a \eax{unitary matrix} if $AA^{\ast} = I$.
\end{definition}

\begin{theorem}[The \eax{spectral theorem}]
    If $A$ is a normal matrix, then there exists a unitary matrix $P$ such that $PAP^{\ast}$ is diagonal.
\end{theorem}

Let $A$ be a $n \times n$ real matrix. Then $A$ defines a bilinear form on $\R^{n}$, that is, a map $\R^{n} \times \R^{n} \to \R$ is given as $(x,y) \mapsto x^{t}Ay$. The map is typically denoted by $\ip{\cdot,\cdot}_{A}$. If $A$ is symmetric, then $\ip{\cdot,\cdot}_{A}$ is a symmetric bilinear form. If $A$ is skew-symmetric, then $\ip{\cdot,\cdot}_{A}$ is a skew-symmetric bilinear form. Moreover if $A$ is non-singular, then $\ip{\cdot,\cdot}_{A}$ is non-degenerate. 

Just like before, $GL_{n}(\C)$ acts on $M_{n}(\C)$ via $PAP^{\ast}$, the conjugation action. In this case, we have
\begin{align}
    \Stab(I_{n}) = \{P \in GL_{n}(\C) \mid PP^{\ast} = I_{n}\} = U_{n},
\end{align}
the \eax{unitary group}. We also have $SL_{n} = \{A \mid \det A = 1\}$, the special linear group.

\begin{definition}
    A subset $X \subseteq \R^{N}$ is called a $n$-dimensional \eax{manifold} if for every $x \in X$, there exists an open neighbourhood $U \subseteq \R^{N}$ of $x$, which is homeomorphic to $\R^{n}$. That is, there exists a map $\varphi : U \to \R^{n}$ such that $\varphi$ is continuous, bijective, and $\varphi^{-1}$ is also continuous.
\end{definition}

The group $SO_{n} = SL_{n} \cap O_{n}$ is termed the special orthogonal group. The group $SU_{n} = SL_{n} \cap U_{n}$ is termed the special unitary group.

\begin{example}
    We have $GL_{1}(\R) = \R^{\ast}$, $GL_{1}(\C) = \C^{\ast} \cong \R^{2} \setminus \{0\}$. One can also infer
    \begin{align}
        SL_{1}(\R) = \{1\} = SL_{1}(\C), \quad O_{1}(\R) = \{1,-1\} = O_{1}(\C),\quad U_{1}(\C) = S_{1},\quad SO_{1}(\R) = SU_{1}(\C) = \{1\}.
    \end{align}
\end{example}

\begin{example}
    $GL_{2}(\R)$ is a $4$-dimensional manifold since it is an open subset of $\R^{4}$. $SL_{2}(\R)$ is a $3$-dimensional manifold since it is the zero set of the polynomial $\det(A) - 1$. $SO_{2}(\R) = SL_{2}(\R) \cap O_{2}(\R)$ is the set of matrices of the form
    \begin{align}
        SO_{2}(\R) = \left\{ \begin{pmatrix}
            a & b \\ c & d
        \end{pmatrix} \Big| AA^{t} = A^{t}A = I_{2},\; \det A = 1 \right\} = \left\{ \begin{pmatrix}
            x_{1} & -x_{2} \\ x_{2} & x_{1}
        \end{pmatrix} \Big| (x_{1},x_{2}) \in S^{1}\right\}
    \end{align}
\end{example}

\begin{proposition}
    There is an isomorphism $U_{1}(\C) \to SO_{2}(\R)$, which is also a homeomorphism.
\end{proposition}
\begin{proof}
    Simply send $z = x_{1}+ix_{2} \in U_{1}(\C)$ to the matrix $\begin{pmatrix}
        x_{1} & -x_{2} \\ x_{2} & x_{1}
    \end{pmatrix} \in SO_{2}(\R)$. One may verify that this map is bijective, continuous, homomorphic, with a continuous inverse as well.
\end{proof}

As above, linear groups of dimension $3$ include $SU_{2}(\C)$, $SO_{3}(\R)$, and $SL_{3}(\R)$. One can show that $SU_{2}(\C)$ is exactly the set of matrices $\begin{pmatrix}
    a & b \\ -\bar{b} & \bar{a}
\end{pmatrix}$ with $\abs{a}^{2} + \abs{b}^{2} = 1$. From here, we get
\begin{align}
    SU_{2}(\C) \cong \{(a,b) \in \C^{2} \mid \abs{a}^{2} + \abs{b}^{2} = 1\} \cong \{(x_{1},x_{2},x_{3},x_{4}) \in \R^{4} \mid x_{1}^{2} + x_{2}^{2} + x_{3}^{2} + x_{4}^{2} = 1\} = S^{3}.
\end{align}
Thus, $S^{3}$ can be given a group structure via relating by $SU_{2}(\C)$. Note that the binary operator induced is also continuous. $S^{2}$, however, cannot be given such a continuous group structure. There is a conjecture that $S^{n}$ for $n \geq 4$ cannot be given a continuous group structure.

\begin{definition}
    In $S^{3}$, we define latitudes, for $-1 < c < 1$, to be the surface cut out by $x_{1} = c$ on $x_{1}^{2} + x_{2}^{2} + x_{3}^{2} + x_{4}^{2} = 1$. This surface turns out to be a sphere of radius $1-c^{2}$.
\end{definition}
Thus the latitudes of $S^{3}$ are homeomorphisms of $S^{2}$. The north pole $x_{1} = 1$ tells us $I \in SU_{2}$, and the south pole $x_{1} = -1$ tells us $-I \in SU_{2}$.

\textit{October 14th.}

\begin{proposition}
    The latitudes of $S^{3}$ are conjugancy classes in $SU_{2}$. Conversely, other than $\pm I$, all conjugacy classes are latitudes.
\end{proposition}

\begin{proof}
    The centre of $SU_{2}$ is $\{\pm I\}$. Let $P = \begin{pmatrix}
        a & b \\ -\bar{b} & \bar{a}
    \end{pmatrix} \in SU_{2}$, that is, $\abs{a}^{2} + \abs{b}^{2} = 1$. The characteristic polynomial of $P$ is given as
    \begin{align}
        \chi_{P}(t) = (t-a)(t-\bar{a}) + b \bar{b} = t^{2} - 2\Re(a) t + 1.
    \end{align}
    This is a real polynomial; if $\lambda$ is a root, then so is $\bar{\lambda}$. By spectral theorem, $P$ is conjugate to $\begin{pmatrix}
        \lambda & 0 \\ 0 & \bar{\lambda}
    \end{pmatrix}$ where $\lambda + \bar{\lambda} = 2\Re(a)$, via a unitary $Q$. If $Q \in SU_{2}$, then done. Otherwise, let $\delta = \det Q$ and let $\epsilon = \sqrt{\delta}$. Then one can verify that $\bar{\epsilon} Q \in SU_{2}$, and \begin{align}
        (\bar{\epsilon} Q) P (\bar{\epsilon} Q)^{\ast} = Q P Q^{\ast} = \begin{pmatrix}
            \lambda & 0 \\ 0 & \bar{\lambda}
        \end{pmatrix}.
    \end{align}
    Note that $\tr P = \lambda + \bar{\lambda} = 2\Re(a)$, so $\Re(a)$ determines the conjugacy class. The latitudes are also determined by $x_{1} = c$, which is also $\Re(a)$. Thus the latitudes are exactly the conjugacy classes.
\end{proof}

\begin{definition}
    In $S^{3}$, we define longitudes to be the intersection of $S^{3}$ with the hyperplanes containing the $x_{1}$-axis (specifically, containing $\pm I_{3}$). These are `circles' going through the north and south poles.
\end{definition}

Let $P \in SU_{2}$ be different from $\pm I$. Then $P$ and $I$ are linearly independent vectors in $\R^{4}$. Then they span a $2$-dimensional subspace $V$ of $\R^{4}$ and $V \cap S^{3}$ is the unique longitude containing $P$. Consider the longitude $S^{3} \cap W$ where $W = \{(x_{1},x_{2},x_{3},x_{4}) \mid x_{3} = x_{4} = 0\}$. That is, $T = SU_{2} \cap W = \left\{ \begin{pmatrix}
    a & 0 \\ 0 & \bar{a}
\end{pmatrix} \Big| \abs{a}^{2} = 1 \right\}$. Note that $T$ is a subgroup of $SU_{2}$ and $T \cong U_{1}(\C)$.

\begin{proposition}
    $T \sbg SU_{2}$. Moreover, longitudes of $SU_{2}$ are exactly the conjugates of the subgroup $T$.
\end{proposition}
\begin{proof}
    $T$ being a subgroup is clear. Let $T'$ be a different longitude with $P' \in T'$ and $P' \neq \pm I$. Then $-1 < \tr P' < 1$. Let $P \in T$ such that $\tr P = \tr P'$. By the previous proposition, there exists $Q \in SU_{2}$ such that $QPQ^{\ast} = P'$. We now show that $QTQ^{\ast}$ is a longitude containing $P'$. Let $Q = \begin{pmatrix}
        a & b \\ -\bar{b} & \bar{a}
    \end{pmatrix}$ for some $a,b \in \C$ with $\abs{a}^{2} + \abs{b}^{2} = 1$, and let $T = \left\{ \begin{pmatrix}
        w_{1} + iw_{2} & 0 \\ 0 & w_{1} - iw_{2}
    \end{pmatrix} \Big| w_{1},w_{2} \in \R, w_{1}^{2} + w_{2}^{2} = 1 \right\}$. If we take such a matrix $W \in T$ with $z = w_{1}+iw_{2}$, then computing $QWQ^{\ast}$ gives a plane viewed in $\R^{4}$. Thus, $QTQ^{\ast} = SU_{2} \cap QWQ^{\ast}$ since $T = SU_{2} \cap W$.
\end{proof}

\textit{October 15th.}

\begin{proposition}
    There exists a surjective continuous homomorphic map $\varphi : SU_{2} \to SO_{3}(\R)$ with kernel $\{\pm I\}$.
\end{proposition}

\begin{proof}
    We have already recognized the action of $SU_{2}$ on latitudes via conjugacy; in particular on $L = \{(0,x_{2},x_{3},x_{4}) \mid x_{2}^{2} + x_{3}^{2} + x_{4}^{2} = 1\} = \left\{ \begin{pmatrix}
        ix_{2} & x_{3} + ix_{4} \\ -x_{3} + ix_{4} & -ix_{2}
    \end{pmatrix} \Big| x_{2}^{2} + x_{3}^{2} + x_{4}^{2} = 1 \right\}$. Without the spherical condition, a basis for this 3-dimensional vector space is given as
    \begin{align}
        \cB = \left\{ A_{1} = \begin{pmatrix}
            i & 0 \\ 0 & -i
        \end{pmatrix},\; A_{2} = \begin{pmatrix}
            0 & 1 \\ -1 & 0
        \end{pmatrix},\; A_{3} = \begin{pmatrix}
            0 & i \\ i & 0
        \end{pmatrix} \right\}.
    \end{align}
    Let $V = \spanof \cB$, and let $P \in SU_{2}$ (note that $SU_{2}$ is a subset of $V$, but \textit{not} a linear subspace). Then $P$ acts on $V$ via $A \mapsto PAP^{\ast}$. Note that for $A \in V$, $\tr A = 0$ and $A^{\ast} = -A$; these two conditions characterise $V \subseteq M_{2}(\C)$. One can verify that $\tr (PAP^{\ast}) = \tr A = 0$ and $(PAP^{\ast})^{\ast} = -PAP^{\ast}$, so $PAP^{\ast} \in V$. Hence we obtain a group homomorphism $\varphi : SU_{2} \to \Bij V$. Let $P \in SU_{2}$ and suppose $\varphi(P):V \to V$ maps $A \mapsto PAP^{\ast}$. One can verify that $\varphi(P)$ is linear, so $\varphi(P) \in GL(V) \cong GL_{3}(\R)$. We now claim that $\varphi(P) \in SO_{3}(\R)$. If $\varphi(P)$ preserves an inner product on $V$, then it is orthogonal. One can verify that $\ip{A,B} = -\frac{1}{2}\tr(AB)$ is an inner product on $V$, and $\ip{\varphi(P)(A),\varphi(P)(B)} = \ip{A,B}$ for all $A,B \in V$. Thus $\varphi(P) \in O_{3}(\R)$. We are now left with showing that $\det \varphi(P) = 1$. Let us call the composition map $SU_{2} \xrightarrow{\varphi} GL_{3} \xrightarrow{\det} \R$ as $f = \det \circ \varphi$. Note that $f$ is a continuous homomorphism. The only possible image of $f$ are $\{1,-1\}$. So $f^{-1}(-1)$ and $f^{-1}(1)$ are both open in $SU_{2}$. If $f^{-1}(1)$ is a non-empty proper (open) subset of $S^{3}$, then $S_{3}\setminus f^{-1}(1) = f^{-1}(-1)$ is not open, which is a contradiction. Hence $f^{-1}(1)$ is either empty or $S^{3}$. Since $f(I) = 1$, we have $f^{-1}(1) = S^{3}$, so $\det \varphi(P) = 1$ for all $P \in SU_{2}$. Thus $\varphi(P) \in SO_{3}(\R)$ for all $P \in SU_{2}$.
\end{proof}
We shall show the surjectivity and the kernel of $\varphi$ in the next section.

\section{Basic Topology}

Note that in the above, we have used the subspace topological definition.

\begin{definition}
    For $X \subseteq \R^{n}$, an \eax{open set} in $X$ is a set of the form $U \cap X$ where $U$ is open in $\R^{n}$.
\end{definition}

\begin{definition}
    $X \subseteq \R^{n}$ is said to be \eax{disconnected} if there exist non-empty open set $U,V \subseteq X$ such that $U \cap V = \emptyset$ and $U \cup V = X$. If $X$ is not disconnected, then it is said to be \eax{connected}.
\end{definition}

We show that $S^{3}$ is connected. Let $U \subsetneq S^{3}$ be open and non-empty. Let $x \in V = S^{3}\setminus U$ such that $x$ is the limit of a sequence $\{x_{n}\} \subseteq U$. Since $x \in V$, which we assume to be open for the sake of contradiction, there exists an open set $B$ such that $x \in B \subseteq V$. But since $x_{n} \to x$, there are infinitely many $x_{i}$'s contained in $B$ and hence in $U_{2}$. But then this shows $U \cap V \neq \emptyset$, which is a contradiction. Thus $S^{3}$ is connected.

\begin{proposition}
    Let $f:X \to Y$ be a continuous map where $X \subseteq \R^{n}$ and $Y \subseteq \R^{m}$. If $X$ is connected, then $f(X)$ is also connected.
\end{proposition}

Note that $A \in SO_{3}$ implies that it fixes a non-zero vector in $\R^{3}$. One can show that $1$ is an eigenvalue of $A$. Let $v$ be a corresponding eigenvector, that is, $Av = v$, and let $\{v,v_{1},v_{2}\}$ be an orthogonal basis of $\R^{3}$. Then $A$ with respect to this basis is given as $\begin{pmatrix}
    1 & 0 & 0 \\ 0 & \cos \theta & \sin \theta \\ 0 & -\sin \theta & \cos \theta
\end{pmatrix}$ for some $\theta$. Thus, $SO_{3}(\R)$ is the set of all such matrices.

We now continue the proof of \textbf{Proposition 5.11}.

\begin{proof}[Proof, continued.]
    We had shown $\varphi: SU_{2} \to SO_{3}(\R)$ is a continuous homomorphism. Trivially, $\varphi(I) = I$ and $\varphi(-I) = I$, so $\{\pm I\} \subseteq \ker \varphi$. Conversely, if $P \in \ker \varphi$, then $PAP^{\ast} = A$ for all $A \in V$. In particular, for $A = A_{1}$, we have
    \begin{align}
        \begin{pmatrix}
            a & b \\ -\bar{b} & \bar{a}
        \end{pmatrix} \begin{pmatrix}
            i & 0 \\ 0 & -i
        \end{pmatrix} \begin{pmatrix}
            \bar{a} & -b \\ \bar{b} & a
        \end{pmatrix} = \begin{pmatrix}
            i & 0 \\ 0 & -i
        \end{pmatrix}.
    \end{align}
    We get $b = 0$. Similarly, for $A = A_{2}$, we get $a = \bar{a}$. Together with $\abs{a}^{2} + \abs{b}^{2} = 1$, we get $a = \pm 1$. Thus $\ker \varphi = \{\pm I\}$. We now show surjectivity. Let $H \in SO_{3}(\R)$, and let its matrix form be given as above with respect to the basis $\{v,v_{1},v_{2}\}$. We want to show that there exists $P \in SU_{2}$ such that $\varphi(P) = H$. Letting $P = \begin{pmatrix}
        a & 0 \\ 0 & \bar{a}
    \end{pmatrix}$, we have $\varphi(P)(A_{1}) = A_{1}$, and a linear combination of $A_{2}$ and $A_{3}$ given as $\begin{pmatrix}
        0 & z \\ -\bar{z} & 0
    \end{pmatrix}$ is mapped as $\varphi(P) \begin{pmatrix}
        0 & z \\ -\bar{z} & 0
    \end{pmatrix} = \begin{pmatrix}
        0 & a^{2} z \\ -\bar{a}^{2} \bar{z} & 0
    \end{pmatrix}$. Since $\varphi(P)$ is special orthogonal, we have $\abs{a}^{2} = 1$ or $a = e^{-i\theta}$ or $a^{2} = e^{-2i\theta}$. Thus, $\varphi(P)$ is exactly the rotation by angle $\theta$ fixing $v$ given as
    \begin{align}
        \varphi(P) = \begin{pmatrix}
            1 & 0 & 0 \\ 0 & \cos 2\theta & \sin 2\theta \\ 0 & -\sin 2\theta & \cos 2\theta
        \end{pmatrix}.
    \end{align}
    Thus, $H \in \Img \varphi$.
\end{proof}

\textit{October 23rd.}

\begin{proposition}
    Let $G$ be a closed and connected subgroup of $GL_{n}(\C)$. Let $H \sbg G$ such that it contains an open non-empty subset $U$ of $G$. Then $H = G$.
\end{proposition}
\begin{proof}
    We claim that for any $g \in G$, the coset $gH$ is open in $G$. Since $U \subseteq H$ is non-empty and open, there exists some $y \in U$. For any $x \in gH$, we have $xH = gH$. Consider the set $xy^{-1}U$, which is an open subset of $xH$ (and hence of $gH$), and note that $x \in xy^{-1}U$. Thus, $gH$ is open in $G$. 
    Now, if $H$ were a proper subgroup of $G$, the collection of cosets $\{ gH \mid g \in G \}$ would contain at least two distinct elements. The union $\bigcup_{g \in G} gH$ would then be a non-empty open set, and by connectedness of $G$, this union must be all of $G$. This leads to a contradiction, as a proper subgroup cannot cover the entire group. Therefore, $H = G$.
\end{proof}

\begin{theorem}
    Let $N \nsbg SU_{2}$ be a normal subgroup, and let $A \in N$ be such that $A \neq \pm I$. Then $N = SU_{2}$. In particular, $SO_{3}$ is simple.
\end{theorem}

\begin{proof}
    Let $P \in N \nsbg SU_{2}$ with $P \notin \{\pm I\}$. Then $-2 < \tr P < 2$. Then $N$ contains the latitude containing $P$; let $P_{0} \neq P$ be another elmenet of the latitude. Now consider the path $f:[0,1] \to SU_{2}$ such that $f(t) \in L$ for all $t$ and $f(0) = P_{0}$ and $f(1) = P$. Now define $\theta(t) = P_{0}^{-1}f(t)$ for $0 \leq t \leq 1$. Then $\theta(0) = I$ and $\theta(1) = P_{0}^{-1}P \neq I$. Note that $\theta(t)$ is a continuous path in $N$, with $\tr \theta(0) = 2$ and $(2-\varepsilon = ) \tr \theta(1) < 2$. Since the trace is continuous, $N$ contains some matrices of trace $[2-\varepsilon,2]$. But latitudes are conjugacy classes, so $N$ contains all matrices of $SU_{2}$ with trace in $[2-\varepsilon,2]$. Thus, $N$ contains an open neighbourhood of $I$ in $SU_{2}$. By the previous proposition, $N = SU_{2}$.

    Moreover, if $N_{0} \nsbg SO_{3}(\R)$ and is non-trivial, then $\varphi^{-1}(N_{0}) \nsbg SU_{2}$ is non-trivial. By the above, $\varphi^{-1}(N_{0}) = SU_{2}$, so $N_{0} = SO_{3}(\R)$. Thus, $SO_{3}(\R)$ is simple.
\end{proof}

\subsection{More on Simple Linear Groups}

We define $PSL_{n} = SL_{n}/Z(SL_{n})$, where $Z(SL_{n}) = \{\lambda I_{n} \mid \lambda^{n} = 1\}$ is the centre of $SL_{n}$.

\begin{theorem}
    The groups $PSL_{n}(\C)$ for $n \geq 2$, $SO_{n}(\C)/Z(SO_{n}(\C))$ for $n \geq 5$, and $\mathrm{sp}_{2n}(\C)/Z(\mathrm{sp}_{2n}(\C))$ for $n \geq 1$ are all simple.
\end{theorem}

\begin{theorem}
    Let $F$ be a field of characteristic different from $p$, and let $\# F \geq 7$. Then $PSL_{2}(F)$ is simple.
\end{theorem}

\begin{proof}
    Let $N \nsbg SL_{2}(F)$ such that $N$ contains $A \neq \pm I$. Note here that $Z(SL_{2}(F)) = \{\lambda I_{2} \mid \lambda^{2} = 1\} = \{\pm I_{2}\}$. We claim that $N$ contains an upper-triangular matrix different from $\pm I$. Let $A = \begin{pmatrix}
        a & b \\ c & d
    \end{pmatrix} \in N$ with $A \neq \pm I$. If $c = 0$, then done. Otherwise, consider
    \begin{align}
        \begin{pmatrix}
            1 & x \\ 0 & 1
        \end{pmatrix} \begin{pmatrix}
            a & b \\ c & d
        \end{pmatrix}
        \begin{pmatrix}
            1 & -x \\ 0 & 1
        \end{pmatrix} = \begin{pmatrix}
            a + cx & \ast \\ c & d - cx
        \end{pmatrix} \in N.
    \end{align}
    There exists $x \in F$ such that $a + xc = 0$. Thus, a matrix of the from $A = \begin{pmatrix}
        0 & b \\ c & d
    \end{pmatrix} \in N$ exists (these $b,c,d$ are altered values). Then $\det A = 1$ implies $bc = -1$. Let $P = \begin{pmatrix}
        u & 0 \\ 0 & u^{-1}
    \end{pmatrix}$ for some $u \in F^{\times}$. Then
    \begin{align}
        N \ni P^{-1}A^{-1} P A = \begin{pmatrix}
            u^{-1} & 0 \\ 0 & u
        \end{pmatrix} \begin{pmatrix}
            d & -b \\ -c & 0
        \end{pmatrix} \begin{pmatrix}
            u & 0 \\ 0 & u^{-1}
        \end{pmatrix} \begin{pmatrix}
            0 & b \\ c & d
        \end{pmatrix} = \begin{pmatrix}
            u^{-2} & bd(1-u^{-2}) \\ 0 & u^{2}
        \end{pmatrix} = A' \in N.
    \end{align}
    We have to verify that this matrix is different from $\pm I$. If it were, we must have $u^{2} = \pm 1$ or $u^{4} = 1$. Since $\# F \geq 7$, there exists $u \in F^{\times}$ such that $u^{4} \neq 1$. Thus, we have found an upper-triangular matrix $A' \in N$ different from $\pm I$. Our next claim is that $N$ contains an upper-triangular matrix with $1$'s on the diagonal and some non-zero $u$ in the upper-right entry. We have already found a matrix $A = \begin{pmatrix}
        a & b \\ 0 & d
    \end{pmatrix} \in N$ different from $\pm I$. If $a = d$ then $a^{2} = 1$ implies $a = d = \pm 1$. If $a = d = 1$, then $A$ is of the desired form, and if $a = d = -1$, then $-A$ is of the desired form. If $a \neq d$, then consider
    \begin{align}
        N \ni A' = \begin{pmatrix}
            1 & 1 \\ 0 & 1
        \end{pmatrix} \begin{pmatrix}
            a & b \\ 0 & d
        \end{pmatrix}
        \begin{pmatrix}
            1 & -1 \\ 0 & 1
        \end{pmatrix} = \begin{pmatrix}
            a & b' \\ 0 & d
        \end{pmatrix} \in N
    \end{align}
    where $b' = b + d - a$. Since $a \neq d$, we have $b' \neq b$. Then look at
    \begin{align}
        A'^{-1} A = \begin{pmatrix}
            1 & ad-d^{2} \\ 0 & 1
        \end{pmatrix} \in N
    \end{align}
    which is of the desired form since $a \neq d \implies ad-d^{2} \neq 0$.

    We make another claim that the conjugacy classes of $\begin{pmatrix}
        1 & u \\ 0 & 1
    \end{pmatrix}$ in $SL_{2}(\F)$ for $u \in F^{\times}$ contains matrices of the form $\begin{pmatrix}
        1 & a^{2}u \\ 0 & 1
    \end{pmatrix}$ for all $a \in F^{\times}$ and $\begin{pmatrix}
        1 & 0 \\ -u & 1
    \end{pmatrix}$. The latter matrix is easily obtained by conjugating with $\begin{pmatrix}
        0 & 1 \\ -1 & 0
    \end{pmatrix}$. The former matrices are obtained by conjugating with $\begin{pmatrix}
        a & 0 \\ 0 & a^{-1}
    \end{pmatrix}$ for all $a \in F^{\times}$. Before proceeding further, we show a small lemma: that if $\mathrm{char} F \neq 2$, then $(F,+)$ is generated by $\{a^{2} \mid a \in F^{\times}\}$. Easy to see since for all $x \in F$, we can write $x = \frac{(x+1)^{2} - (x-1)^{2}}{4}$.

    One last claim is that since $\mathrm{char} F \neq 2$, and there is a matrix of the form $\begin{pmatrix}
        1 & u \\ 0 & 1
    \end{pmatrix} \in N$ for some $u \neq 0$, then $N$ contains all matrices of the form $\begin{pmatrix}
        1 & v \\ 0 & 1
    \end{pmatrix}$ for all $v \in F$. Let $H = \{x \mid \begin{pmatrix}
        1 & x \\ 0 & 1
    \end{pmatrix} \in N\}$. Then $H$ is a subgroup of $(F,+)$ since if $x,y \in H$, then $x-y \in H$ (verify this). Also $0 \neq u \in H$ by assumption. By the previous claim, $a^{2}u \in H$ for all $a \in F$. But since $\{a^{2} \mid a \in F\}$ generates $(F,+)$, we have $H = F$. Similar to this claim, one can show that $N$ contains all matrices of the form $\begin{pmatrix}
        1 & 0 \\ u & 1
    \end{pmatrix}$ for all $u \in F$. We now show that $SL_{2}(F)$ is generated by these two types of matrices. Let $A = \begin{pmatrix}
        a & b \\ c & d
    \end{pmatrix} \in SL_{2}(F)$. Look at
    \begin{align}
        \begin{pmatrix}
            1 & u \\ 0 & 1
        \end{pmatrix} \begin{pmatrix}
            a & b \\ c & d
        \end{pmatrix} = \begin{pmatrix}
            a+uc & b+ud \\ c & d
        \end{pmatrix},\quad \begin{pmatrix}
            1 & 0 \\ u & 1
        \end{pmatrix} \begin{pmatrix}
            a & b \\ c & d
        \end{pmatrix} = \begin{pmatrix}
            a & b \\ c + ua & d + ub
        \end{pmatrix}.
    \end{align}
    If $c = 0$, then $A_{1} = \begin{pmatrix}
        1 & 0 \\ 1 & 1
    \end{pmatrix} A = \begin{pmatrix}
        a_{1} & b_{1} \\ c_{1} & d_{1}
    \end{pmatrix}$. Then $c_{1} \neq 0$. Now there exists $c' \in F$ such that $c_{1} c' = 1-a_{1}$, or $c' = c_{1}^{-1}(1-a_{1})$. Then
    \begin{align}
        A_{2} = \begin{pmatrix}
            1 & c' \\ 0 & 1
        \end{pmatrix} A_{1} = \begin{pmatrix}
            a_{1} + c'c_{1} & b_{1} + c'd_{1} \\ c_{1} & d_{1}
        \end{pmatrix} = \begin{pmatrix}
            1 & b_{2} \\ c_{2} & d_{2}
        \end{pmatrix}.
    \end{align}
    Similarly,
    \begin{align}
        A_{3} = \begin{pmatrix}
            1 & 0 \\ -c_{2} & 1
        \end{pmatrix} A_{2} = \begin{pmatrix}
            1 & b_{3} \\ 0 & d_{3}
        \end{pmatrix}, \quad A_{4} = A_{3} \begin{pmatrix}
            1 & -b_{3} \\ 0 & 1
        \end{pmatrix} = \begin{pmatrix}
            1 & 0 \\ 0 & d_{4}
        \end{pmatrix}.
    \end{align}
    Since $A_{1},A_{2},A_{3},A_{4} \in SL_{2}(F)$, we must have $d_{3} = d_{4} = 1$. Taking inverses and rearranging, we have expressed $A$ as a product of the two types of matrices. Thus, $N = SL_{2}(F)$.
\end{proof}

\begin{corollary}
    $PSL_{2}(\F_{p})$ is simple for all primes $p \geq 5$.
\end{corollary}
\chapter{FREE GROUPS}

Let $S = \{a_{1},a_{2},\ldots\}$ be a non-empty set of letters. Our objective is to define a (free) group on $S$. Note that $S$ need not be finite or countable. Let $\bar{S} = \{a_{1},a_{2},\ldots\} \sqcup \{a_{1}^{-1},a_{2}^{-1},\ldots\} \sqcup \{\square\}$ be the set of letters and their formal inverses, and another `null' element $\square$. An element of $\bar{S}$ is called a \eax{letter}. A \eax{word} on $S$ is a finite sequence of letters from $\bar{S}$, i.e., an element of the set $\bigcup_{n=0}^{\infty} \bar{S}^{n}$, where $\bar{S}^{0} = \{\square\}$. If $\Omega$ denotes the set of all words,
then clearly
\begin{align}
    \Omega = \{\square\} \sqcup \bar{S} \sqcup \bar{S}^{2} \sqcup \bar{S}^{3} \sqcup \cdots = \bigcup_{n=0}^{\infty} \bar{S}^{n} = \{f:\N \to \bar{S} \mid f(i) = \square \text{ for all but finitely many } i\}.
\end{align}
A word $w \in \Omega$ is said to be a \eax{reduced word} if the associated function $f:\N \to \bar{S}$ if the following are satisfied:
\begin{itemize}
    \item if there exists $k$ such that $f(k) = \square$, then $f(i) = \square$ for all $i > k$;
    \item for all $i$, if $f(i) = a_{j}$ for some $j$, then $f(i+1) \neq a_{j}^{-1}$, and vice versa.
\end{itemize}
Essentially, null characters do not appear in the middle of a reduced word, and no letter is immediately followed by its formal inverse. The set of all reduced words is denoted by $\bar{\Omega}$. For example, if $S = \{a,b\}$, then $ab^{-1}a^{-1}b$ is a reduced word, but $ab^{-1}\square a^{-1}b$ and $abb^{-1}a^{-1}$ are not. Note that $\square$ is a reduced word.

Now define a binary operation $\ast:\bar{\Omega}\times\bar{\Omega} \to \bar{\Omega}$, termed \eax{concatenation}, as follows: for $u,v \in \bar{\Omega}$, let $u$ and $v$ have the associated functions $f,g:\N \to \bar{S}$ respectively. Let $k \in \N$ be the least integer such that $f(k+1) = \square$. Then define $u \ast v$ to be the reduced word whose associated function $h:\N \to \bar{S}$ is given by
\begin{align}
    h(i) = \begin{cases}
        f(i) & \text{if } 1 \leq i \leq k,\\
        g(i-k) & \text{if } k < i
    \end{cases}
\end{align}
where $h$ is reduced to ensure that $u \ast v \in \bar{\Omega}$; this reduction is ensured by the following lemma.

\begin{lemma}
    Given $w \in \Omega$, there exists a unique $w_{\emph{\text{red}}} \in \bar{\Omega}$ such that $w_{\emph{\text{red}}}$ is obtained from $w$ by repeatedly deleting adjacent pairs of the form $a_{j}a_{j}^{-1}$ or $a_{j}^{-1}a_{j}$, for some $j$, until no such pair exists, and deleting all occurrences of $\square$ if a non-$\square$ letter appears anywhere after it.
\end{lemma}

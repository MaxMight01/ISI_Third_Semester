\chapter{INTRODUCTION TO GROUP THEORY}

\section{Set Theory}
\textit{July 22nd.}

We begin with some basic assumptions to introduce set theory. The symbol $\in$ is used to denote membership in a set. A statement using this in set theory may be stated as $x \in y$, which can be either true or false. Once we have developed this language to discuss sets, we can introduce some axioms.
\begin{axiom}
    There exists a set with no elements, the \eax{empty set} $\emptyset$.
\end{axiom}
Formally, the above axiom is $\exists x (\forall y (y \notin x))$.
\begin{axiom}
    Two sets are equal if they have the same elements.
\end{axiom}
From the above two axioms, we can infer a unique empty set. A notion of subsets may also be declared.
\begin{definition}
    We say the set $A$ is a \eax{subset} of the set $B$, denoted $A \subseteq B$, if every element of $A$ is also an element of $B$.
\end{definition}
We also have a bunch of similarity axioms stated below.
\begin{axiom}[Similarity axioms]
    We have the following:
    \begin{enumerate}
        \item If $x, y$ are sets, then $\{x, y\} \Rightarrow \{x, \{x, y\}\}$ (not an ordered pair).
        \item If $A$ is a set, then $\bigcup A = \{x \mid \exists y \in A,\ x \in y\}$ is a set.
        \item There exists a \eax{power set} for every set; given a set $A$, there exists a set $P(A)$ such that for all $B \subseteq A$, $B \in P(A)$. Formally, $\forall A \exists P(A) (\forall B \subseteq A, B \in P(A))$.
        \item The \eax{infinite axiom}: Formally, $\exists I (\emptyset \in I \wedge \forall y \in I (P(y) \in I))$.
        \item If $A$ and $B$ are sets, then $A \times B = \{(x,y) \mid x \in A, y \in B\}$ is a set.
    \end{enumerate}
\end{axiom}
Before discussing the last axiom, we define a relation on sets.
\begin{definition}
    A \eax{relation} $R$ on a set $A$ is a subset $R \subseteq A \times A$. If $(x,y) \in R$, we write $xRy$.
\end{definition}
\begin{axiom}[The \eax{axiom of choice}]
    Let $A$ be a collection of non-empty and disjoint sets. Then there exists a set $C$ consisting of exactly one element from each set in $A$.
\end{axiom}


\begin{definition}
    A relation $R$ on a set $A$ is said to be:
    \begin{itemize}
        \item \eax{reflexive} if $xRx \forall x \in A$,
        \item \eax{symmetric} if $xRy \Rightarrow yRx$,
        \item \eax{transitive} if $xRy \wedge yRz \Rightarrow xRz$,
        \item \eax{antisymmetric} if $xRy \wedge yRx \Rightarrow x = y$.
    \end{itemize}
\end{definition}

\begin{definition}
    A \eax{partial order} on a set $A$ is a reflexive, transitive, and antisymmetric relation on $A$.
\end{definition}
Some examples of partially ordered sets include $(R, \leq)$, $(P(\R), \subseteq)$.

\begin{definition}
    A \eax{total order} $R$ on a set $A$ is a partial order such that for all $x,y \in A$, either $xRy$ or $yRx$.
\end{definition}
Again, $(R, \leq)$ is a totally ordered set, but not $(P(\R), \subseteq)$.
\begin{definition}
    A total order $\leq$ on a set $A$ is said to be a \eax{well-order} if given any non-empty subset $B \subseteq A$, there exists $x \in B$ such that for all $y \in B$, $x \leq y$.
\end{definition}

The below theorem may be derived from the above definitions and axioms.

\begin{theorem}[The \eax{well-ordering principle}]
    Every set can be well-ordered.
\end{theorem}
We may note that the well-ordering principle and the axiom of choice are equivalent.

\begin{definition}
    A \eax{chain} in partially ordered set $A$, with relation $\prec$, is a subset of $A$ which is totally ordered with respect to $\prec$.
\end{definition}
\begin{definition}
    Let $C \subseteq A$ be a subset in a partially ordered set $(A, \prec)$. An element $x \in A$ is an \eax{upper bound} of $C$ if for all $y \in C$, $y \prec x$.
\end{definition}
\begin{definition}
    An element $x \in A$ is a \eax{maximal element} of a partially ordered set $(A, \prec)$ if for all $y \in A$, $x \prec y \Rightarrow x = y$.
\end{definition}
\begin{lemma}[\eax{Zorn's lemma}]
    Let $A$ be a set and let $\prec$ be a partial order on $A$ such that every chain in $A$ has an upper bound. Then $A$ has a maximal element.
\end{lemma}

\begin{theorem}
    The following are equialent:
    \begin{enumerate}
        \item The axiom of choice,
        \item The well-ordering principle,
        \item Zorn's lemma.
    \end{enumerate}
\end{theorem}

\begin{proof}
    We begin with 2.~implies 3.; let $A$ be a non-empty set. Consider 
    \begin{align}
        \cC = \{(B,\leq) \mid B \subseteq A \text{ and }\leq \text{ is a well-order on } B\}.
    \end{align}
    We note that $\cC$ is non-empty since if we pick $B = \{x\}$ for some $x \in A$, then $x \leq x$ and $(B,\leq) \in \cC$. Let $(B,\leq), (C,\leq') \in \cC$. We say $(B, \leq) \preceq (C,\leq')$ if there exists $y \in C$ such that 
    \begin{align}
        \text{$B = \{x \in C \mid x \lneq' y\}$ $(=I(c,y))$ and $\leq = \leq'\mid_{B}$, or $(B,\leq) = (C,\leq')$}
    \end{align}
    Note that $\preceq$ is a partial order on $\cC$ and is clearly reflexive.
    
    For transitivity, if we take $B \preceq C$ and $C \preceq D$, then $B = C$ or $B = I(C,y)$ for some $y \in C$, and $C = D$ or $C = I(D,z)$ for some $z \in D$. If equality holds in either case, then clearly $B \preceq D$. If $B = I(C,y)$ and $C = I(D,z)$. Clearly, $B = I(D,y)$.
    
    Now let $T$ $=(\{(B_{i},\leq_{i}) \mid i \in I\})$ be a chain in $\cC$. Let $B = \bigcup_{i \in I} B_{i}$, and $\leq = \bigcup_{i \in I} \leq_{i}$. Note that this makes sense since if $x \in B_{i}$ and $y \in B_{j}$ with $B_{i} \preceq B_{j}$, then $x,y \in B_{j}$. So, we assign $x \leq y$ if $x \leq_{j} y$. Now let $C \subseteq B$ be non-empty. Also let $x \in C$; then $x \in B_{i}$ for some $i \in I$. Let $w = \min (B_{i} \cap C)$. We claim that $w = \min C$. For $y \in C$, if $y \in B_{i}$ then $w \leq y$. If $y \notin B_{i}$ then $y \in B_{j} \in T$. Since $T$ is a chain, either $B_{i} \preceq B_{j}$ or $B_{j} \preceq B_{i}$; the latter is not possible since $y \notin B_{i}$. Thus, $B_{i} = I(B_{j},z)$, for some $z \in B_{j}$, and for any $x \in B_{i}$, $w \leq x \leq y$. 

    So $(B,\leq) \in \cC$ and it is an upper bound of $T$; to realize it is an upper bound, we show that $B_{i} \preceq B$ for all valid $i$. If $B_{i} = B$, we are done. Otherwise, let $x = \min (B \setminus B_{i})$. Then $B_{i} = I(B,x)$, and $B_{i} \preceq B$. Thus, by Zorn's lemma, $\cC$ has a maximal element--- cal it $(M, \leq)$.

    We now claim that $M = A$. If $M \subsetneq A$, then let $a \in A \setminus M$. If we let $\hat{M} = (M \cup \{a\}, \leq')$ where $x \leq' a$ for all $x \in M$, then $M = I(\hat{M},a)$ but this is a contradiction to the fact that $(M,\leq)$ is a maximal element. Thus, $A = M$.\\

    Next comes 1.~implies 3. Let $X$ be a partially ordered set such that every chain has an upper bound. Suppose $X$ has no maximal element; we will utilise the axiom of choice to arise at a contradiction. For every chain $T$ in $X$, there exists a strict upper bound $c_{T}$. Define a function $f$ sending chains $T$ in $X$ to $X$ as $f(T) = c_{T} \notin T$. Such a function $f$ exists by the axiom of choice. A subset $A \subseteq X$ is called a \eax{conforming subset} if $A$ is well-ordered, with respect to order on $X$, and for all $x \in A$, $f(I(A,x)) = x$. We claim that if $A$ and $B$ are conforming subsets of $X$, then $A = B$ or one is the initial segment of the other. For now, let us take this claim to be true. We shall prove it later.

    If $f(\emptyset) = x$ then $A = \{x\}$. Note that $A$ is conforming. But $I(A,x) = \emptyset \implies f(I(A,x)) = x$. Let $U$ be the union of all conforming subsets of $X$. Then $U$ is conforming since if $x \in U$ then $x \in B$ for some $B$ conforming and $x = f(I(B,x)) = f(I(U,x))$. Let $f(U) = w$. Define a new set $\tilde{U} = U \sqcup \{w\}$, which is well-ordered and conforming. Then $U = I(\tilde{U},w)$, which is a contradiction.

    Coming back to the claim, suppose $x \in A \setminus B$. We wish to show that $B = I(A,x)$ for some $x \in A$. Let $x = \min(A\setminus B)$. We claim that this $x$ works. $I(A,x) \subseteq B$ holds since if $y \in A$ and $y < x$ then $y \in B$, or else $x \neq \min (A\setminus B)$. Suppose, now, that the equality does not hold. Take $y = \min (B \setminus I(A,x))$ and $z = \min (A \setminus I(B,y))$. We claim that $I(A,z) = I(B,y)$. Take $v \in I(A,z)$; then $v < z$ implies $v \in I(B,y)$ since $z = \min(A\setminus I(B,y))$. Taking $u \in I(B,y)$, we have $u \in I(A,x) \implies u < x$ since $y = \min(B\setminus I(A,x))$. If $z \leq u$, then $z \in I(A,x) \subseteq B \implies z \in I(B,y)$ contrading the fact that $z = \min (A\setminus I(B,y))$. Thus, $z > u$ and $y \in I(A,z)$. Finally, $z = f(I(A,z)) = f(I(B,y)) = y$ implies $z = x = y$. But this a contradiction since $x \in A\setminus B$ and $y \in B$.
\end{proof}

\begin{definition}
    A relation $R$ on a set $A$ is said to be an \eax{equivalence relation} if it is reflexive, symmetric, and transitive. Let $x \in A$. Then $[x] = \{yRx \mid y \in A\} \subseteq A$ is called the \eax{equivalence class} of $x$.
\end{definition}

We note that $\bigcup_{x \in A} [x] = A$ and for $x,y \in A$, either $[x] \cap [y] = \emptyset$ or $[x] = [y]$. Thus, we get a partition of $A$ into equivalence classes.

Let $I$ be an indexing set, and let $A_{i}$ be sets for all $i \in I$. Then the existence of $\text{X}_{i \in I} A_{i} = \{f:I \to \overset{\cdot}{\bigsqcup} A_{i} \mid f(i) \in A_{i} \text{ for all } i \in I\}$ is another way of stating the axiom of choice.

\begin{theorem}[The \eax{principle of induction}]
    Let $S(n)$ be statements about the naturals $n \in \N$. Suppose $S(1)$ holds and for all $k \in \N$, $S(k) \Rightarrow S(k+1)$. Then $S(n)$ holds true for all $n \in \N$.
\end{theorem}

Let $I$ be a well-ordered set and let $S(i)$ be statements for all $i \in I$. Suppose that if $S(j)$ holds for all $j < i$, then $S(i)$ holds. Then $S(i)$ holds for all $i \in I$. This is the \eax{principle of transfinite induction}, which is also equivalent to the axiom of choice. We now properly introduce the theory of groups.

\section{Groups}
We first define a group.
\begin{definition}
    A \eax{group} is a triple $(G, \cdot, e)$ where $G$ is a set, $\cdot: G \times G \to G$ is a binary operation on $G$, and $e \in G$ is an element of $G$ satisfying the following axioms:
    \begin{itemize}
        \item The property of \eax{associativity}: For $a,b,c \in G$, $(a \cdot b) \cdot c = a \cdot (b \cdot c)$.
        \item The property of the \eax{identity element}: For all $a \in G$, $a \cdot e = e \cdot a = a$. $e$ is referred to as the identity element.
        \item The existence and property of the \eax{inverse element}: For all $a \in G$, there exists $b \in G$ such that $a \cdot b = b \cdot a = e$.
    \end{itemize}
    In addition, $(G,\cdot,e)$ is also termed an \eax{abelian group} if for all $a,b \in G$, $a \cdot b = b \cdot a$, that is, commutativity holds.
\end{definition}
A group may also be rewritten as $(G,\cdot)$, or just $G$. Some examples include $(\Z, +), (\Q, +), (\R, +), (\C, +)$. The set $(\Q, \cdot)$ is not a group since $0$ does not have an inverse. However, $(\Q^{\ast}, \cdot)$ is a group, where $\Q^{\ast} = \Q\setminus\{0\}$. All these groups are also abelian. An example of a non-abelian group is $S_{n}$, the set of all bijections from $\{1,2,\ldots,n\}$ to itself, under the binary operation of composition of functions. Another non-abelian group is $(GL_{n}(\R), \cdot)$, for $n \geq 2$, the set of all invertible real $n \times n$ matrices.

\textit{July 24th.}

From the axioms, arise basic properties related to groups.
\begin{proposition}
    Let $(G,\cdot,e)$ be a group.
    \begin{enumerate}
        \item Let $a \in G$ be such that $a \cdot b = b$ for all $b \in G$. Then $a = e$; the identity element is unique.
        \item Each element $a \in G$ has a unique inverse. Thus, the inverse of $a$ is then termed $a^{-1}$.
        \item $(a^{-1})^{-1} = a$ holds for all $a \in G$.
        \item For all $a,b \in G$, $(a \cdot b)^{-1} = b^{-1} \cdot a^{-1}$.
        \item  Let $a \in G$ be such that $a \cdot b = b$ for some $b \in G$. Then $a = e$.
    \end{enumerate}
\end{proposition}
\begin{proof}
    \begin{enumerate}
        \item Choose $b$ to be $e$. Then $a \cdot e = e$ by hypothesis, and $a \cdot e = a$ by the property of the identity element. Thus, $a = e$.
        
        \item Let $a \in G$ and $b \in G$ be such that $a \cdot b = b \cdot a = e$. Let $c \in G$ be also such that $c \cdot a = e$. Thus, $(c \cdot a) \cdot b = e \cdot b \Rightarrow c \cdot (a \cdot b) = e \cdot b \Rightarrow c \cdot e = b \Rightarrow c = b$.
        
        \item Easy to see since $a^{-1} \cdot a = a \cdot a^{-1} = e$ which just means that the inverse of $a^{-1}$ is $a$.
        
        \item Also easy since $(a \cdot b) \cdot (b^{-1} \cdot a^{-1}) = (b^{-1} \cdot a^{-1}) \cdot (a \cdot b) = e$.
        
        \item Finally, right multiplying $b^{-1}$ leads to $a = a \cdot b \cdot b^{-1} = b \cdot b^{-1} = e$.
    \end{enumerate}
\end{proof}

\noindent \textit{July 29th.}

\begin{definition}
    The \eax{order of a group} $G$ is the cardinality of the set $G$, and is denoted by $\abs{G}$, $o(G)$, or $\ord(G)$. If $\abs{G}$ is finite, we say $G$ is a \eax{finite group}.
\end{definition}

We provide some examples.

\begin{example}
    \begin{itemize}
        \item The \eax{trivial group} is $G = \{e\}$, with $e \cdot e = e$. Here, $\abs{G} = 1$, and it is the smallest possible finite group. Similarly, one can form a group with two elements as $G = \{e,a\}$, with $a \cdot a = e$ and $a \cdot e = e \cdot a = a$.
        
        \item Another important example is the set of all bijections of a set $X$, denoted by $S(X)$. It forms a group under composition. Here, if $f,g \in S(X)$, then $f \circ g \in S(X)$. Similarly, the bijection $\id_{X}(x) = x$ for all $x \in X$ is the identity element of $S(X)$. Associativity also holds, and the inverse of $f \in S(X)$ is simply the inverse mapping $f^{-1} \in S(X)$ to get $f \circ f^{-1} = f^{-1} \circ f = \id_{X}$. If $X = \{1,2,\ldots,n\}$, then $S(X)$ is also denoted by $S_{n}$, with $\abs{S_{n}} = n!$. If the set $X$ is infinite, then so is $S(X)$.
        
        \item The set $\sfrac{\Z}{n\Z}$ is a group when equipped with the binary operation of addition $(+)$. Here, $\abs{\sfrac{\Z}{n\Z}} = n$.
        
        \item The set $\mu_{n} = \{e^{2\pi im/n} \mid 1 \leq m \leq n\}$ is a group with respect to multiplication. Again, $\abs{\mu_{n}} = n$.
    \end{itemize}
\end{example}

Order is also defined for elements.

\begin{definition}
    Let $(G,\cdot,e)$ be a group. The \eax{order of an element} $a \in G$, denoted $o(a)$, $\ord(a)$, or $\abs{a}$, is the least $n \geq 1$ such that $a^{n} = e$. If no such $n$ exists, then we term $\abs{a} = \infty$.
\end{definition}

Examples follow.

\begin{example}
    \begin{itemize}
        \item In $\mu_{n}$, $o(e^{2\pi i/n}) = n$.
        
        \item Similarly, in $\sfrac{\Z}{n\Z}$, $o([1]_{n}) = n$. For a general element $[a]_{n} \in \sfrac{\Z}{n\Z}$, the order is $o([a]_{n}) = \frac{n}{\gcd(a,n)}$.
    \end{itemize}
\end{example}

\begin{proposition}
    Let $G$ be a finite group. For all $a \in G$, $o(a)$ is finite.
\end{proposition}
\begin{proof}
    Let $a \in G$. We look at $a,a^{2},a^{3},\ldots \in G$. Since $G$ is finite, not all are distinct; there exists $m > n$ such that $a^{m} = a^{n}$. Multiplying by $a^{-n}$, we have $a^{m-n} = a^{n-n} = e$, and the order of $a$ is finite.
\end{proof}

\subsection{The $S_{n}$ Group}
To understand the order better, we look specifically at $S_{3}$.

\begin{example}
    The elements in $S_{3}$ are
    \begin{align}
        \begin{pmatrix}
            1 & 2 & 3 \\ 1 & 2 & 3
        \end{pmatrix}, 
        \begin{pmatrix}
            1 & 2 & 3 \\ 2 & 1 & 3
        \end{pmatrix}, 
        \begin{pmatrix}
            1 & 2 & 3 \\ 1 & 3 & 2
        \end{pmatrix}, 
        \begin{pmatrix}
            1 & 2 & 3 \\ 3 & 2 & 1
        \end{pmatrix}, 
        \begin{pmatrix}
            1 & 2 & 3 \\ 2 & 3 & 1
        \end{pmatrix}, \text{ and }
        \begin{pmatrix}
            1 & 2 & 3 \\ 3 & 1 & 2
        \end{pmatrix}.
    \end{align}
    Alternatively, the elements may be (correspondingly) written as
    \begin{align}
        e, \begin{pmatrix}
        1 & 2
        \end{pmatrix}, 
        \begin{pmatrix}
        2 & 3
        \end{pmatrix}, 
        \begin{pmatrix}
        1 & 3
        \end{pmatrix}, 
        \begin{pmatrix}
        1 & 2 & 3
        \end{pmatrix}, \text{ and }
        \begin{pmatrix}
        3 & 2 & 1
        \end{pmatrix}.
    \end{align}
    It is easy to see that the orders of $e, \begin{pmatrix}
        1 & 2
    \end{pmatrix}, \begin{pmatrix}
        1 & 2 & 3
    \end{pmatrix}$ are $1,2,3$, respectively. The elements $\begin{pmatrix}
        1 & 2
        \end{pmatrix}, 
        \begin{pmatrix}
        2 & 3
        \end{pmatrix}, \text{ and }
        \begin{pmatrix}
        1 & 3
        \end{pmatrix}$ are termed transpositions.
    In general, an element $\sigma \in S_{n}$ is called a \eax{transposition} if there exists $1 \leq a \neq b \leq n$ such that $\sigma(a) = b$ and $\sigma(b) = a$, but $\sigma(x) = x$ for all $x \notin \{a,b\}$.

    An element $\sigma \in S_{n}$ is called a \eax{cycle} if there exists distinct $1 \leq a_{1},a_{2},\ldots,a_{m} \leq n$ such that $\sigma(a_{i}) = a_{i+1}$ for $1 \leq i \leq m-1$, $\sigma(a_{m}) = a_{1}$, and $\sigma(x) = x$ for all $x \notin \{a_{1},a{2},\ldots,a_{m}\}$. Thus, a transposition is really just a cycle of length 2. If $\sigma$ is a cycle of length $m$, then $o(\sigma) = m$.

    In the above, $\sigma^{i}(a_{1}) = a_{i+1}$ if $i < m$. Thus, $\sigma^{i} \neq e$ for $i < m$. But for $m$-times composition, we have $\sigma^{m}(a_{i}) = a_{i}$ for all $1 \leq i \leq m$. Hence, the order of $\sigma$ is really $m$.
\end{example}

Note that $S_{3}$ is non-abelian since $\begin{pmatrix}
    1 & 2
\end{pmatrix} \begin{pmatrix}
    1 & 3
\end{pmatrix} = \begin{pmatrix}
    1 & 3 & 2
\end{pmatrix}$, but $\begin{pmatrix}
    1 & 3
\end{pmatrix} \begin{pmatrix}
    1 & 2
\end{pmatrix} = \begin{pmatrix}
    1 & 2 & 3
\end{pmatrix}$.

\begin{definition}
    Let $\sigma, \tau \in S_{n}$ be cycles. They are called \eax{disjoint cycles} if $\sigma = (a_{1},\ldots,a_{m})$ and $\tau = (b_{1},\ldots,b_{k})$, and $\{a_{1},\ldots,a_{m}\} \cap \{b_{1},\ldots,b_{k}\} = \emptyset$.
\end{definition}

If $\sigma$ and $\tau$ are disjoint cycles then they commute; that is, $\sigma \circ \tau = \tau \circ \sigma$.

\begin{proposition}
    Every element of $S_{n}$ can be written as a product of disjoint cycles.
\end{proposition}

\begin{proof}
    Let $\sigma \in S_{n}$, and let $k$ be the least positive integer such that $\sigma^{k}(1) = 1$. Then let $\tau_{1} = \begin{pmatrix}
        1 & \sigma(1) & \sigma^{2}(1) & \cdots & \sigma^{k-1}(1)
    \end{pmatrix}$. Let $S'_{1}$ be the \eax{support} of $\tau_{1}$, defined as $\supp (\tau_{1}) = \{1,\sigma(1),\ldots,\sigma^{k-1}(1)\}$. If $S'_{1} = \{1,2,\ldots,n\}$, we are done. Otherwise, let $a_{2} = \min (\{1,2,\ldots,n\}\setminus S'_{1})$. Let $k_{2}$ be the least positive integer such that $\sigma^{k_{2}}(a_{2}) = a_{2}$, and then let $\tau_{2} = \begin{pmatrix}
        a_{2} & \sigma(a_{2}) & \cdots & \sigma^{k_{2}-1}(a_{2})
    \end{pmatrix}$. Then $\tau_{2}$ is a cycle of length of $k_{2}$. Again, let $S_{2}' = \supp(\tau_{2})$. We claim that $S_{1}' \cap S_{2}' = \emptyset$.

    If $\sigma(a_{2})$ were in $S_{1}'$, then we would have $\sigma^{i}(i) = a_{2} \in S_{1}'$, but $a_{2}$ was taken from $\{1,2,\ldots,n\}\setminus S_{1}'$. Similarly, if $\sigma^{j}(a_{2}) \in S_{1}'$, then a similar problem arises. Thus, the sets have to be disjoint.

    Continue this way to get $\tau_{1},\tau_{2},\ldots,\tau_{l}$ until $S_{1}' \cup S_{2}' \cup \cdots \cup S_{k}' = \{1,2,\ldots,n\}$. The process stops since $S_{1}',S_{2}',\ldots,S_{k}'$ are non-empty. Thus, we conclude that $\tau_{1} \circ \tau_{2} \circ \cdots \circ \tau_{l}$ is the disjoint cycle decomposition of $\sigma$.
\end{proof}

For ease of notation, we will write $\sigma \circ \tau$ as $\sigma\tau$.

\begin{proposition}
    Let $\sigma \in S_{n}$ and $\sigma = \tau_{1}\tau_{2}\cdots\tau_{k}$ be a disjoint cycle decomposition of $\sigma$. Then, $\abs{\sigma} = \lcm (\abs{\tau_{1}}, \abs{\tau_{2}},\ldots,\abs{\tau_{k}})$.
\end{proposition}

\begin{proof}
    The proof of this proposition is left as an exercise to the reader.
\end{proof}


\section{Subgroups}

We begin with the definition.

\begin{definition}
    A non-empty subset $H$ of a group $(G,\cdot)$ is called a \eax{subgroup} if the following properties hold.
    \begin{enumerate}
        \item For all $a,b \in H$, $a \cdot b \in H$.
        \item For all $a \in H$, $a^{-1} \in H$.
    \end{enumerate}
    In such a scenario, we write $H \leqslant G$.
\end{definition}

More properties of a subgroup can be inferred.

\begin{proposition}
    The following properties hold true for a subgroup $H \leqslant G$, where $(G,\cdot,e)$ is a group.
    \begin{enumerate}
        \item $e \in G$.
        \item $(H,\cdot,e)$ is a group.
    \end{enumerate}
\end{proposition}

\begin{proof}
    \begin{enumerate}
        \item $H$ is non-empty, so there exists $a \in G$ such that $a \in H$. From the definition, $a^{-1} \in H$ also. Since $H$ is closed under the binary operation, we have $a \cdot a^{-1} = e \in H$.
        
        \item We show that $(H,\cdot,e)$ satisfies the group axioms. From definition, $\cdot$ is an associative binary operaion on $H$. Also, $e$ is the identity element in $H$. Again, from the definition, each $a \in H$ has an inverse $a^{-1} \in H$.
    \end{enumerate}
\end{proof}

Equivalently, $H$ is a subgroup if the following holds.

\begin{theorem}
    Let $G$ be a group and $H \subseteq G$ be non-empty. Then $H$ is a subgroup of $G$ if and only if $a \cdot b^{-1} \in H$ for all $a,b \in H$.
\end{theorem}
\begin{proof}
    The forward implication is left as an exercise to the reader. If $a \in H$ then $a \cdot a^{-1} \in H$ shows that $e \in H$. Since $e,a \in H$, $e \cdot a^{-1} = a^{-1} \in H$. If $a,b \in H$, then $a,b^{-1} \in H \implies a \cdot (b^{-1})^{-1} \in H \implies ab \in H$
\end{proof}
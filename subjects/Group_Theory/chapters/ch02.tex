\chapter{COSETS AND MORPHISMS}

\section{Cosets}
We start with cosets.

\begin{definition}
    Let $H \sbg G$ and $x \in G$. A \eax{left coset} of $H$ generated by $x$ is $xH = \{xh \mid h \in H\} \subseteq G$. The left coset need not be a subgroup of $G$. Similarly, a \eax{right coset} of $H$ generated by $x$ is $Hx = \{hx \mid h \in H\} \subseteq G$. Again, the right coset need not be a subgroup
\end{definition}

Let $H \sbg G$. For $x,y \in G$, let us write $x \sim y$ if $x^{-1}y \in H$. Then $\sim$ is an equivalence relation. Moreover, $[x] = xH$ for all $x \in G$. Once we have proved, we will be able to partition our group.

\begin{proof}
    Clearly, $\sim$ is reflexive since $x^{-1}x = e \in H$ for all $x \in G$. $\sim$ is symmetric since we have
    \begin{align}
        x \sim y \implies x^{-1}y \in H \implies (x^{-1}y)^{-1} H \implies y^{-1}x \in H \implies y \sim x.
    \end{align}
    Finally, $\sim$ is also transitive since
    \begin{align}
        x \sim y \text{ and } y \sim z \implies x^{-1}y,y^{-1}z \in H \implies x^{-1}y \cdot y^{-1}z = x^{-1}z \in H \implies x \sim z.
    \end{align}
    To show the latter result, we first have
    \begin{align}
        y \in [x] \implies x \sim y \implies x^{-1}y \in H \implies xx^{-1}y = y \in xH \implies y \in xH.
    \end{align}
    So, $[x] \subseteq xH$. For the converse inclusion, we have
    \begin{align}
        y \in xH \implies y = xh \text{ for some } h \in H \implies x^{-1}y = h \in H \implies y \in [x].
    \end{align}
    Thus, $xH \subseteq [x]$ and $xH = [x]$.
\end{proof}

The above results of cosets prove to be useful in the following theorem.

\begin{theorem}[\eax{Lagrange's theorem}]
    Let $G$ be a finite group with $H \sbg G$. Then $\abs{H} \mid \abs{G}$.
\end{theorem}

\begin{proof}
    For $x,y \in G$, if $xH \cap yH \neq \emptyset$, then we must have $xH = yH$. Also, $\bigcup_{x \in G} xH = G$. We now claim that $\abs{xH} = \abs{yH}$ for all $x,y \in G$. To show this, we let $f:xH \to yH$ be defined as $f(a) = yx^{-1}a$, and $g:yH \to xH$ be defined as $g(b) = xy^{-1}b$. Then $f$ and $g$ are inverses of each other since
    \begin{align}
        (f \circ g)(b) = f(xy^{-1}b) = yx^{-1}xy^{-1}b = b \text{ and } (g \circ f)(a) = g(yx^{-1}a) = xy^{-1}yx^{-1}a = a.
    \end{align}
    Let $S = G/\sim$ (also denoted as $G/H$). Since $G = \bigcup_{A \in S} A$, we have $\abs{A} = \abs{H}$ for all $A \in S$, implying $\abs{G} = \abs{S} \abs{H}$.
\end{proof}

\begin{corollary}
    Let $G$ be a finite group, with $a \in G$. Then $o(a) \mid \abs{G}$.
\end{corollary}

\begin{proof}
    If $o(a) = n$, then $\ip{a} = \{a,a,^{2},\ldots,a^{n-1},e\}$. Since this is a subgroup, we have $\abs{\ip{a}} = n \mid \abs{G}$ by Lagrange's theorem.
\end{proof}

\section{Mappings}
\textit{August 5th.}

We now study important mappings between groups and the types of mappings one can define.

\begin{definition}
    A function $f:(G,\ast) \to (H,\circ)$, where $(G,\ast)$ and $(H,\circ)$ are groups, is said to be a (group) \eax{homomorphism} if
    \begin{align}
        f(x \ast y) = f(x) \circ f(y) \text{ for all } x,y \in G.
    \end{align}
\end{definition}

The following is a trivial example of a group homomorphism.

\begin{example}
    For instance, the map $a \mapsto a$ in $(\Z,+) \to (\Q,+)$ is a group homomorphism, trivially. More generally, if $H \sbg G$, then $a \mapsto a$, called the \eax{inclusion map} is a group homomorphism.
\end{example}

Homomorphisms can be classified further if they inherit nicer properties.

\begin{definition}
    The group homomorphism is also called an injective homomorphism, or a \eax{monomorphism}, if the mapping is also injective. Similarly, it is also called a surjective homomorphism, or a \eax{epimorphism}, if the mapping is also surjective. Finally, the group homomorpshim is termed an \eax{isomorphism} if it is bijective.
\end{definition}

\begin{example}
    \begin{enumerate}
        \item The map $q:(\Z,+) \to (\sfrac{\Z}{n\Z},+)$ defined as $q(a) = [a]_{n}$ for $n \geq 1$ is a group homomorphism. Specifically, it is an epimorphism.
        
        \item $f:(G,\ast) \to (\{e\},\cdot)$ with $f(g) = e$ for all $g \in G$ is another epimorphism. This is also a trivial homomorphism.
        
        \item The scaling map $a \mapsto \lambda a$ in $\Z \to \Z$ is a monomorphism for $\lambda \in Z_{\geq 1}$. Similarly, $[a] \mapsto [\lambda a]$ in $\sfrac{\Z}{n\Z} \to \sfrac{\Z}{n\Z}$ is group homomorphism. If $\gcd(n,\lambda) = 1$, then the map is also an isomorphism in this case.
        
        \item The scaling map $f:\Q \to \Q$ with $f(a) = ca$ with $c \in \Q^{\ast}$ is an isomorphism. For $c = 0$, we get the trivial homomorphism.
        
        \item From linear algebra, the map $T:(\Q^{n},+) \to (\Q^{n},+)$ with $T \in M_{n}(\Q)$ defined as $v \mapsto Tv$ is also a group homomorphism. If $T \in GL_{n}(\Q) \subseteq M_{n}(\Q)$, the map is also an isomorphism.
        
        \item Towards more non-trivial examples, one can confirm that the map $\exp:(\R,+) \to (\R_{> 0},\cdot)$ defined as $x \mapsto e^{x}$ is a group homomorphism.
    \end{enumerate}
\end{example}


\subsection{Properties}

Arising from these structure-preserving mappings are some useful properties.

\begin{proposition}
    Let $f:(G,\ast) \to (H,\circ)$ be a group homomorphism. Then
    \begin{enumerate}
        \item $f(e_{G}) = e_{H}$,
        \item $f(a^{n}) = f(a)^{n}$, and
        \item $f(a)^{-1} = f(a^{-1})$.
    \end{enumerate}
\end{proposition}
\begin{proof}
    \begin{enumerate}
        \item Simply work as
        \begin{align}
            f(e_{G}) = f(e_{G} \ast e_{G}) = f(e_{G}) \circ f(e_{G}) \implies e_{H} = f(e_{G})^{-1} \circ f(e_{G}) = f_{e_{G}}.
        \end{align}

        \item We show the base case, then induction may be applied.
        \begin{align}
            f(a^{2}) = f(a \ast a) = f(a) \circ f(a) = f(a)^{2}.
        \end{align}

        \item Again,
        \begin{align}
            f(a^{-1}) \circ f(a) = f(a^{-1} \ast a) = f(e_{G}) = e_{H} = f(a)^{-1} \circ f(a) \implies f(a^{-1}) = f(a)^{-1}.
        \end{align}
    \end{enumerate}
\end{proof}

We show further some properties of bijective homomorphisms.

\begin{proposition}
    Let $f:(G,\ast) \to (H,\cdot)$ be a group isomorphism. Then
    \begin{enumerate}
        \item $f^{-1}:H \to G$ is a group isomorphism,
        \item $o(x) = o(f(x))$ for all $x \in G$,
        \item $\abs{G} = \abs{H}$, and
        \item $G$ is abelian if and only if $H$ is abelian.
    \end{enumerate}
\end{proposition}

\begin{proof}
    \begin{enumerate}
        \item Fix $a,b \in H$, and let $x = f^{-1}(a)$ and $y \in f^{-1}(b)$. We want to show that $f^{-1}(a \cdot b) = f^{-1}(a) \ast f^{-1}(b)$. To this end, we have
        \begin{align}
            f(f^{-1}(a) \ast f^{-1}(b)) = f(x \ast y) \Rightarrow f(f^{-1}(a)) \cdot f(f^{-1}(b)) = a \cdot b = f(x \ast y) \Rightarrow f^{-1}(a \cdot b) = x \ast y.
        \end{align}

        \item Let $o(x) = n$, where $x^{m} \neq e_{G}$ for $1 \leq m < n$ and $x^{n} = e_{G}$. This shows that $f(x)^{n} = f(x^{n}) = e_{H}$. Also, since $x^{m} \neq e_{G}$ for $1 \leq m < n$, we must have $f(x^{m}) \neq e_{H}$ for $1 \leq m < n$ as $f$ is bijective. Thus, $o(f(x)) = o(x)$. If $o(x)$ were not finite, then $x^{n} \neq e_{G}$ for all $n \geq 1$ implies $f(x)^{n} \neq e_{H}$ for all $n \geq 1$.
        
        \item This is trivial.
        
        \item If $G$ is abelian then $a \ast b = b \ast a$ for all $a,b \in G$. Applying $f$, we get $f(a \ast b) = f(b \ast a) \Rightarrow f(a) \cdot f(b) = f(b) \cdot f(a)$ for all $a,b \in G$. If we take $a = f^{-1}(x)$ and $b = f^{-1}(y)$, we get $x \cdot y = y \cdot x$ for all $x,y \in H$. For the converse implication, simply consider the isomorphism $f^{-1}$.
    \end{enumerate}
\end{proof}

Essentially, in group theory, we consider two groups the same if they are isomorphic. Thus, we are equipped to classify groups up to isomorphism seeing as they share basically the same struture and properties. If two groups $G$ and $H$ are isomorphic, we denote it as $G \cong H$.

\begin{proposition}
    Let $(G,\cdot)$ be a group of order $p$, where $p$ is prime. Then $G \cong \sfrac{\Z}{n\Z}$.
\end{proposition}

\begin{proof}
    Let $x \in G$ be a non-identity element. Then $o(x) = p$ since $o(x) \mid p$ and $o(x) \neq 1$. Define the map $f:\sfrac{\Z}{n\Z} \to G$ as $f(a) = x^{a}$. We show that this mapping is an isomorphism. For $\overline{a},\overline{b} \in \sfrac{\Z}{n\Z}$, we have
    \begin{align}
        f(\overline{a} + \overline{b}) = f(\overline{a+b}) = x^{a+b} = x^{a} \cdot x^{b} = f(\overline{a}) \cdot f(\overline{b})
    \end{align}
    showing $f$ is a group homomorphism. Moreover, $G = \ip{x}$ as $o(x) = p$, so $G$ is also surjective. Hence, $f$ is an isomorphism as $G$ is finite.
\end{proof}

\begin{example}
    We find all the groups of order 4 upto isomorphism. The only two possibilities are $\sfrac{\Z}{4\Z}$, and $\left( \sfrac{\Z}{2\Z} \right)^{2}$ with component-wise addition.
\end{example}

\begin{example}
    We list down all the groups of order 6 upto isomorphism. Again, the only two possibilities are $\sfrac{\Z}{6\Z}$ and $S_{3}$.
\end{example}

\textit{August 7th.}

\begin{example}
    For groups of order 8, we have $\sfrac{\Z}{8\Z}$, $(\sfrac{\Z}{2\Z})^{2}$, $\sfrac{\Z}{4\Z} \times \sfrac{\Z}{2\Z}$, $D_{4}$, and $Q_{8}$, the quaternions.
\end{example}

The \eax{quaternions} $Q_{8}$ is the group $\{\pm 1, \pm i, \pm j, \pm k\}$ equipped with the multiplication operation such that
\begin{align}
    i^{2} &= -1, \quad j^{2} = -1, \quad k^{2} = -1, \\
    ij &= k, \quad jk = i, \quad ki = j, \\
    ji &= -k, \quad kj = -i, \quad ik = -j.
\end{align}

\subsection{Kernel and Image}

\begin{definition}
    For a group homomorphism $f:G \to H$, we define the \eax{kernel} of $f$ as $\ker f = \{g \in G \mid f(g) = e_{H}\}$. We also define the \eax{image} of $f$ as $f(G) = \Img f = \{f(g) \mid g \in G\}$.
\end{definition}

The image and kernel are both subgroups; this is our proposition.

\begin{proposition}
    For a group homomorphism $f:G \to H$, $\ker f$ and $\Img f$ are subgroups of $G$ and $H$ respectively.
\end{proposition}

\begin{proof}
    Note that $\emptyset \neq \Img f \subseteq H$; let $a,b \in \Img f$. Then $f(x) = a$ and $f(y) = b$ for some $x,y \in G$. So, $ab = f(x) f(y) = f(xy) \in \Img f$, showing $ab \in \Img f$. Also, $a^{-1} = f(x)^{-1} = f(x^{-1}) \in \Img f$, showing $a^{-1} \in \Img f$. Thus, $\Img f \sbg H$.

    For the kernel, note that $e_{g} \in \ker f$ since $f(e_{G}) = e_{H}$. Let $x,y \in \ker f$. Then $f(x) = f(y) = e_{H}$ implying $f(xy^{-1}) = f(x)f(y)^{-1} = e_{H}e_{H}^{-1} = e_{H}$. Thus, $xy^{-1} \in \ker f$, showing $\ker f \sbg H$.
\end{proof}

\begin{remark}
    Let $f$ be a group homomorphism.
    \begin{enumerate}
        \item If $f$ is an isomorphism, then $\Img f = H$ and $\ker f = \{e_{G}\}$.
        \item If $f$ is a monomorphism, then $\ker f = \{e_{G}\}$.
        \item If $f$ is an epimorphism, then $\Img f = H$.
    \end{enumerate}
\end{remark}

\section{Normal Subgroups and Quotient Groups}

\begin{proposition}
    Let $G$ be a group and $H \sbg G$ be a subgroup. Then the following are equivalent.
    \begin{enumerate}
        \item $gH \subseteq Hg$ for all $g \in G$,
        \item $g^{-1}Hg \subseteq H$ for all $g \in G$,
        \item $gH = Hg$ for all $g \in G$,
        \item $g^{-1}Hg = H$ for all $g \in G$.
    \end{enumerate}
\end{proposition}

Such a subgroup satisfying any (all) of the above conditions is termed a \eax{normal subgroup} of $G$ and is denoted by $H \nsbg G$.

\begin{proof}
    For 1.~implies 2., we are given $g^{-1}H \subseteq Hg^{-1}$ for all $g \in G$. Let $x \in g^{-1}Hg$. Then $x = g^{-1}hg$ for some $h \in H$. Thus, $g^{-1}h \in g^{-1}H \subseteq Hg^{-1}$ which implies $g^{-1}h = h'g^{-1}$ for some $h' \in H$. But then $g^{-1}hg = h' \in H$, showing $x \in H$. Therefore, $g^{-1}Hg \subseteq H$.

    For 2.~implies 3., assume $g^{-1}Hg \subseteq H$ for all $g \in G$. Let $x \in gH$, that is, $x = gh$ for some $h \in H$. Write this as $x = ghg^{-1}g$. But $ghg^{-1} \in gHg^{-1} \subseteq H$, so $ghg^{-1} = h'$ for some $h' \in H$. Thus, $x = h'g \in Hg$. Similarly, if $x \in Hg$, then $x \in gH$. We conclude that $Hg = gH$ for all $g \in G$.

    For 3.~implies 4., we have $gH = Hg$ for all $g \in G$. Let $x \in g^{-1}Hg$, where $x = g^{-1}hg$ for some $h \in H$. Note that $hg = gh'$ for some $h' \in H$ since $gH = Hg$. Thus, $x = g^{-1}hg = g^{-1}(gh') = h' \in H$, giving us $g^{-1}Hg \subseteq H$.

    Finally, for 4.~implies 1., let $x \in gh$; there exists $h \in H$ such that $x = gh$. Thus, $x = ghg^{-1}g = h'g \in Hg$ since $gHg^{-1} = H$. Hence, $gH \subseteq Hg$.
\end{proof}

Note that if $G$ is abelian, then every subgroup is normal.

\begin{proposition}
    The following miscellaneous propositions hold true. Let $G$ be a group.
    \begin{enumerate}
        \item If $g,h \in G$, then $\ord(ghg^{-1}) = \ord(h)$.
        \item The mapping $\varphi_{g}:G \to G$ defined as $\varphi_{h}(h) = g^{-1}hg$ is an isomorphism for all $g \in G$. The inverse isomorphism is given by $\varphi_{g}^{-1} = \varphi_{g^{-1}}$.
        \item Both $G$ and $\{e\}$ are normal subgroups of $G$.
    \end{enumerate}
\end{proposition}
\begin{proof}
    The proofs of these are left as an exercise to the reader.
\end{proof}

\begin{proposition}
    Let $f:G \to H$ be a group homomorphism. Then $\ker f \nsbg G$.
\end{proposition}
\begin{proof}
    For $g \in G$, let $x \in g^{-1} \ker(f) g$; that is, $x = g^{-1}hg$ for some $h \in \ker f$. Then,
    \begin{align}
        f(x) = f(g^{-1}hg) = f(g^{-1})f(h)f(g) = f(g)^{-1}e_{H}f(g) = e_{H}.
    \end{align}
    Thus, $x \in \ker f$, showing $g^{-1} \ker(f) g \subseteq \ker f$ for all $g \in G$; $\ker f \nsbg G$.
\end{proof}

\begin{proposition}
    Let $G$ be a group with $H \sbg G$ a subgroup. Then $H \nsbg G$ if and only if for all $\varphi_{g}:G \to G$, we have $\varphi_{g}|_{H}:H \to H$, an isomorphism.
\end{proposition}
Note that an isomorphism from a group to itself is called an \eax{automorphism}. Thus, the above proposition equates to $\varphi_{g}$ still remaning an automorphism when restricted to $H$.
\begin{proof}
    The statement is simply equivalent to saying $g^{-1}Hg = H$ for all $g \in G$.
\end{proof}

One also defines the notion of \eax{product of groups}. Let $G_{1},G_{2}$ be two groups. Then
\begin{align}
    G_{1} \times G_{2} \defeq \{ (g_{1},g_{2}) \mid g_{1} \in G_{1}, g_{2} \in G_{2} \}
\end{align}
is a group with the equipped operation defined as
\begin{align}
    (g_{1},g_{2}) \cdot (g_{1}',g_{2}') = (g_{1}g_{1}',g_{2}g_{2}').
\end{align}
Here, the identity element is $(e_{G_{1}},e_{G_{2}})$ and the inverse of $(g_{1},g_{2})$ is $(g_{1}^{-1},g_{2}^{-1})$.

Let $G$ be a group and $H,K$ be subgroups of $G$. Let $HK = \{hk \mid h \in H,k \in K\}$. Then $HK$ is a group if either $H$ or $K$ is a normal subgroup of $G$.
\begin{proof}
    Let us assume $H \nsbg G$. Take the elements $h_{1}k_{1},h_{2}k_{2} \in HK$. Since $H \nsbg G$, $k_{1}H = Hk_{1}$. So, $k_{1}h_{2} = h'k_{1}$ for some $h' \in H$ Thus,
    \begin{align}
        h_{1}k_{1}h_{2}k_{2} = h_{1}h'k_{1}k_{2} \in HK.
    \end{align}
    Similarly, $(h_{1}k_{1})^{-1} = k_{1}^{-1}h_{1}^{-1} = h'k_{1}^{-1} \in HK$ for some $h' \in H$, since $H \nsbg G$ and $k_{1}^{-1}H =  Hk_{1}^{-1}$.
\end{proof}

\textit{August 12th.}

We now get familiar with quotient groups.
\begin{definition}
    Let $G$ be a group and $H \nsbg G$ a normal subgroup. The \eax{quotient group} is defined as $G/H = \{gH \mid g \in G\}$ with the operation defined as $gH \ast g'H \defeq (gg')H$ for all $g,g' \in G$.
\end{definition}
Of course, it still remains to verify that the groups axioms are not violated and the operation is indeed well-defined.
\begin{proof}
    Let $gH = kH$ and $g'H = k'H$ for $k,k' \in G$. We wish to show that $gg'H = kk'H$. Since $gH = kH$, we have $k^{-1}g \in H$. Similarly, $k'^{-1}g' \in H$, and $k'^{-1}g'(k^{-1}g) \in H$. Thus, $(k g')^{-1}gg' \in H$ and $gg'H = kg'H$. Hence, the operation is well-defined. We verify the group axioms now.
    \begin{enumerate}
        \item \textit{Associativity:} We have
        \begin{align}
            (gH \ast hH) \ast (kH) = (ghH) \ast kH = (gh)kH = g(hk)H = gH \ast (hkH) = (gH) \ast (gH \ast kH).
        \end{align}

        \item \textit{Existence of Identity:} The identity here is $e_{G/H} = H$ since
        \begin{align}
            gH \ast H = (g e) H = gH = (e g) H = H \ast gH.
        \end{align}

        \item \textit{Existence of Inverse:} For $gH \in G/H$, we have $(gH)^{-1} = g^{-1}H$ since
        \begin{align}
            (gH) \ast (g^{-1}H) = (g g^{-1})H = H = (g^{-1} g) H = (g^{-1}H) \ast (gH).
        \end{align}
    \end{enumerate}
\end{proof}
\noindent Note that the map $q:G \to G/H$ defined as $g \mapsto gH$ is a group epimorphism, with $\ker q = H$. The proof of showing surjectivity and preservation of group structure is left as an exercise the reader.

\begin{example}
    \begin{itemize}
        \item As a familiar example, $n\Z \nsbg \Z$ is a normal subgroup, and the quotient group $\Z/n\Z$ is the group of integers modulo $n$.
        \item If one sets $H = \{e\} \nsbg G$, then $G/H = \{\{g\} \mid g \in G\}$ is the group of singletons, and the quotient map $q:G \to G/H$ becomes an isomorphism.
        \item If $H = G \nsbg G$, then $G/H = \{G\}$ is the trivial group.
    \end{itemize}
\end{example}

\subsection{Centre}

\begin{definition}
    The \eax{centre of a group} $G$, denoted $Z(G)$, is the set of all elements in $G$ that commute with every element of $G$:
    \begin{align}
        Z(G) = \{ g \in G \mid gx = xg \text{ for all } x \in G \}.
    \end{align}
\end{definition}

One can show that $Z(G) \sbg G$ always holds true. In fact, the centre is a subgroup of $G$.

\begin{example}
    \begin{itemize}
        \item Since $Z(G)$ is a normal subgroup, the quotient group makes sense. However, in general, $Z(G/Z(G))$ is not trivial.
        \item $G$ is abelian if and only if $Z(G) = G$.
        \item $Z(GL_{2}(\R)) = \{\lambda I \mid \lambda \in \R^{\ast}\}$, where $I$ is the identity matrix.
    \end{itemize}
\end{example}

One can also define a centre for individual elements and subsets in a group.

\begin{definition}
    The \eax{centre of a subset} $H \subseteq G$ is defined as
    \begin{align}
        C_{G}(H) = \{g \in G \mid gh = hg \text{ for all } h \in H\}.
    \end{align}
    Note that $Z(G) \subseteq C_{G}(H)$ and $C_{G}(G) = Z(G)$. For $g \in G$, we define the \eax{centralizer} of $g$ as $C_{G}(g) \defeq C_{G}(\{g\})$. Additionally, if $H \sbg G$, then
    \begin{align}
        N_{G}(H) = \{g \in G \mid gH = Hg\}
    \end{align}
    is termed the \eax{normalizer} of $H$ in $G$.
\end{definition}

\begin{remark}
    The following may be shown, for a subset $H \subseteq G$.
    \begin{itemize}
        \item $C_{G}(H) = \bigcap_{g \in H} C_{G}(g)$.
        \item $C_{G}(H) \sbg G$ holds.
    \end{itemize}
    The following may be shown, for a subgroup $H \sbg G$.
    \begin{itemize}
        \item $C_{G}(H) \subseteq N_{G}(H)$ holds.
        \item $H \nsbg N_{G}(H)$ holds.
        \item $N_{G}(H) \sbg G$ holds.
    \end{itemize}
\end{remark}

The proofs of the above are left as an exercise to the reader.

\section{The Isomorphism Theorems}

These are important theorems that hold regarding isomorphisms. In particular, they describe the relationships between different quotient groups and subgroups. Before we encounter the actual theorems, we establish a minor result.

\begin{proposition}
    Let $f:G \to H$ be a group homomorphism and let $K = \ker f$. Then $K \nsbg G$. Moreover, $K = \{e\}$ if and only if $f$ is injective.
\end{proposition}
\begin{proof}
    The first part follows from the fact that $g^{-1}Kg \subseteq K$ for all $g \in G$. For the second part, if $f$ is injective, then $\ker f = \{e\}$ since $f(g) = e_{H}$ implies $g = e_{G}$. Conversely, suppose $K = \{e\}$ and let $f(g) = f(g')$ for some $g,g' \in G$. Then
    \begin{align}
        f(g^{-1}g') = f(g^{-1})f(g') = f(g)^{-1}f(g) = e_{H} \implies g^{-1}g' = e_{G} \implies g' = g.
    \end{align}
\end{proof}

\subsection{First Isomorphism Theorem}

\begin{theorem}[The \eax{first isomorphism theorem}]
    Let $f:G \to H$ be a group homomorphism. Then the map $\tilde{f}:G/K \to \Img f$ sending $gK \mapsto f(g)$ is a well-defined isomorphism where $K = \ker f$. Bluntly,
    \begin{align}
        G/\ker f \cong \Img f.
    \end{align}
\end{theorem}
\begin{proof}
    We first show that $\tilde{f}$ is well-defined. Suppose $gK = g'K$ for some $g,g' \in G$. Then
    \begin{align}
        g^{-1}g' \in K \implies f(g^{-1}g') = e_{H} \implies f(g) = f(g').
    \end{align}
    To show $\tilde{f}$ is a homomorphism, let $aK,bK \in G/K$. Then
    \begin{align}
        \tilde{f}(aK \cdot bK) = \tilde{f}((ab)K) = f(ab) = f(a)f(b) = \tilde{f}(aK) \cdot \tilde{f}(bK).
    \end{align}
    Finally, we show $\tilde{f}$ is bijective. Let $h \in \Img f$. Then there exists $g \in G$ such that $f(g) = h$. We claim that $\tilde{f}(gK) = h$. Indeed,
    \begin{align}
        \tilde{f}(gK) = f(g) = h.
    \end{align}
    Thus, $\tilde{f}$ is surjective. To show injectivity, suppose $\tilde{f}(gK) = \tilde{f}(g'K)$ for some $g,g' \in G$. Then
    \begin{align}
        f(g) = f(g') \implies g^{-1}g' \in K \implies gK = g'K.
    \end{align}
    Therefore, $\tilde{f}$ is injective. We conclude that $\tilde{f}$ is a bijection.
\end{proof}

\textit{August 14th.}

\begin{example}
    We discuss the \eax{Heisenberg group}
    \begin{align}
        H = \left\{ \begin{pmatrix}
              1 & a & b \\
              0 & 1 & c \\
              0 & 0 & 1
           \end{pmatrix} \;\middle|\; a,b,c \in \R \right\}.
    \end{align}
    The group operation here is matrix multiplication, and one can see that $H \sbg SL_{3}(\R)$. Here, the center of $H$ is precisely 
    \begin{align}
        Z(H) = \left\{ \begin{pmatrix}
              1 & 0 & b \\
              0 & 1 & 0 \\
              0 & 0 & 1
           \end{pmatrix} \;\middle|\; b \in \R \right\}.
    \end{align}
    If we look at the map $f:H \to \R^{2}$ that sends
    \begin{align}
        \begin{pmatrix}
              1 & a & b \\
              0 & 1 & c \\
              0 & 0 & 1
           \end{pmatrix} \mapsto (a,c)
    \end{align}
    then $f$ is a group homomorphism since
    \begin{align}
        f(AA') = (a+a',c+c') = (a,c) + (a',c') = f(A) + f(A').
    \end{align}
    Moreover, $\Img f = \R^{2}$ and $\ker f = Z(H)$. Hence, by the first isomorphism theorem, we have
    \begin{align}
        H/Z(H) \cong \R^{2}
    \end{align}
    with the map $\tilde{f}:H/Z(H) \to \R^{2}$ given by $\tilde{f}(A Z(H)) = f(A)$. Note that $\R^{2}$ is an abelian group, and so is $H/Z(H)$. Hence, $Z(H/Z(H)) = H/Z(H)$.
\end{example}


\subsubsection{Commutator Subgroup}

\begin{definition}
    Let $G$ be a group. Then
    \begin{align}
        (G:G) \defeq \ip{\{ghg^{-1}h^{-1} \mid g,h \in G\}}
    \end{align}
    is termed the \eax{commutator subgroup} of $G$.
\end{definition}

\noindent Of course, the name suggests that $(G:G)$ is a subgroup of $G$. In fact, it is actually a normal subgroup of $G$.

\begin{proposition}
    $(G:G)$ is a normal subgroup of $G$. Moreover $G/(G:G)$ is abelian. Let $f:G \to A$ be a group homomorphism with an abelian group $A$. Then $(G:G) \subseteq \ker f$ and there exists $\bar{f}:G/(G:G) \to A$ such that $f = \bar{f} \circ q$ where $q:G \to G/(G:G)$ is the quotient map.
\end{proposition}

\begin{proof}
    Let $x \in G$. Then we have
    \begin{align}
        x^{-1}(ghg^{-1}h^{-1})x &= x^{-1}gxx^{-1}hxx^{-1}g^{-1}xx^{-1}h^{-1}x = (x^{-1}gx)(x^{-1}hx)(x^{-1}g^{-1}x)(x^{-1}h^{-1}x) \notag \\
        &= aba^{-1}b^{-1} \in (G:G) \text{ where } a = x^{-1}gx \text{ and } b = x^{-1}hx.
    \end{align}
    Thus, if $S = \{ghg^{-1}h^{-1} \mid g,h \in G\}$, then $x^{-1}Sx \subseteq (G:G)$ for all $x \in G$. We now show that $(G:G)$ is a subgroup. Let $a \in (G:G)$. Then $a = b_{1} \cdots b_{n}$, where $b_{i} \in S$, and
    \begin{align}
        x^{-1}ax = x^{-1}b_{1}xx^{-1}b_{2}x \cdots x^{-1}b_{n}x \in (G:G)
    \end{align}
    since $x^{-1}b_{i}x \in (G:G)$ for all $i$. Thus, $x^{-1}(G:G)x \subseteq (G:G)$ for all $x \in G$, showing $(G:G) \nsbg G$. Hereforth, in this example, let $C = (G:G)$. We now show $G/C$ is abelian. This is simple enough since $ghg^{-1}h^{-1} = gh(hg)^{-1} \in C$ implies $gChC = ghC = hgC = hCgC$.

    Now let $f:G \to A$ be a group homomorphism with $A$ an abelian group. We then have
    \begin{align}
        f(ghg^{-1}h^{-1}) = f(g)f(h)f(g^{-1})f(h^{-1}) = f(g)f(h)f(g)^{-1}f(h)^{-1} = e_{A}.
    \end{align}
    Thus, $S \subseteq \ker f$ implying $(G:G) \subseteq \ker f$. Finally, we show that $\bar{f}:G/C \to A$ defined as $gC \mapsto f(g)$ is an isomorphism. To show $\bar{f}$ is well-defined, we have $gC = hC \Leftrightarrow gh^{-1} \in C \subseteq \ker f$ showing $f(gh^{-1}) = e_{A}$ or $f(g) = f(h)$. To show a homomorphism, for $g,h \in G$, we have
    \begin{align}
        \bar{f}(gChC) = f(gh) = f(g)f(h) = \bar{f}(gC)\bar{f}(hC).
    \end{align}
    Moreover, $\bar{f} \circ q(g) = \bar{f}(gC) = f(g)$ for all $g \in G$, so $\bar{f} \circ q = f$.
\end{proof}

\begin{remark}
    A few corollaries, we have
    \begin{itemize}
        \item If $G$ is abelian then $(G:G) = \{e\}$.
        \item For $H$, the Heisenberg group, we have $(H:H) = Z(H)$.
        \item $(G/(G:G):G/(G:G)) = \{e\}$.
        \item If $H \sbg G$ then $\abs{H} \mid \abs{G}$. The \eax{index} of $H$ in $G$ is defined as $[G:H] = \abs{G/H}$. It is also equal to $\frac{\abs{G}}{\abs{H}}$ if $\abs{G}$ is finite.
    \end{itemize}
\end{remark}


\subsection{Second and Third Isomorphism Theorems}

\begin{theorem}[The \eax{second ismorphism theorem}]
    Let $G$ be a group and $A,B \sbg G$ be subgroups such that $A \sbg N_{G}(B)$. Then $AB \sbg G$, $B \nsbg AB$, and $A \cap B \nsbg A$. Moreover,
    \begin{align}
        AB/B \cong A/(A \cap B).
    \end{align}
\end{theorem}

\begin{proof}
    Firstly, we have $A \sbg N_{G}(B)$ and $B \nsbg N_{G}(B) = \{g \in G \mid gB = Bg\}$. Thus, $AB \sbg N_{G}(B)$. Of course, this also means $AB \sbg G$. Also, $B \nsbg AB$ since $B \nsbg N_{G}(B)$ and $AB \subseteq N_{G}(B)$.

    We now define a map $f:A \to AB/B$ that maps $a \mapsto aB$ since $a \in A \subseteq AB$. $f$ is a group homomorphism. Thus, by the first isomorphism theorem,
    \begin{align}
        \bar{f} : A/\ker f \to \Img f
    \end{align}
    is an isomorphism. It is enough to show that $\ker f = A \cap B$ and $f$ is surjective. Again, simple to see since
    \begin{align}
        \ker f = \{a \in A \mid aB = B\} = \{a \in A \mid a \in B\} = A \cap B;
    \end{align}
    for surjectivity, let $x \in AB/B$. Then $x = abB$ for some $a \in A$ and $b \in B$. But $abB$ is simply $aB$ so we simply have $abB = aB = f(a)$.
\end{proof}

\begin{corollary}
    For $G$ a group with $A,B \sbg G$, we have $[AB:B] = [A:A \cap B]$.
\end{corollary}

Finally, we move on to the third theorem.

\begin{theorem}[The \eax{third isomorphism theorem}]
    Let $G$ be a group and let $H,K \nsbg G$ such that $K \sbg H$. Then $H/K \nsbg G/K$ and
    \begin{align}
        \frac{G/K}{H/K} \cong G/H.
    \end{align}
\end{theorem}

\begin{proof}
    We define a map $f:G/K \to G/H$ via $gK \mapsto gH$. This is well-defined since if $gK = g'K$, then $g^{-1}g' \in K \subseteq H$ and hence $gH = g'H$. We now show that $f$ is a group homomorphism. For $gK,hK \in G/K$, we have
    \begin{align}
        f(gKhK) = f(ghK) = ghH = gHhH = f(gK)f(hK).
    \end{align}
    $f$ is also surjective as $gH = f(gK)$ for all $g \in G$. By the first isormorphism theorem, we have
    \begin{align}
        \frac{G/K}{\ker f} \cong \Img f = G/H
    \end{align}
    where the isomorphism is given by the map $gK \ker f \mapsto gH$. Since $\ker f = \{gK \mid gH = H\} = \{gK \mid g \in H\} = H/K$, our proof is complete.
\end{proof}
\chapter{GROUP ACTIONS}

\section{An Overview}

Let $G$ be a group and $S$ be a set. A \eax{(left) group action} or $G$-action on $S$ is a function $\theta : G \times S \to S$ satisfying
\begin{enumerate}
    \item $\theta(g_{1}g_{2},x) = \theta(g_{1},\theta(g_{2},x))$ for all $g_{1},g_{2} \in G$ and $x \in S$.
    \item $\theta(e_{G},x) = x$ for all $x \in S$.
\end{enumerate}
In practice, we prefer to write $\theta$ simply as $(g,x) \mapsto gx$. Thus, the axioms are simply $g_1(g_2x) = (g_1g_2)x$ and $ex = x$ for all $g_1,g_2 \in G$ and $x \in S$. Unless there are multiple different group actions on $S$ in context, we will stick with this notation instead. In this case, $S$ is called a $G$-set.\\

\textit{August 19th.}

\begin{remark}
    \begin{itemize}
        \item Let $\varphi_{g}:S \to S$ sending $x \mapsto gx$ for all $g \in G$ and $x \in S$. Then $\varphi_{g}$ is a bijection for each $g \in G$. This can be shown simply by considering the maps $\varphi_{g^{-1}} \circ \varphi_{g}$.
        \item Let $G$ be a group and $X$ be a $G$-set. Then the map $\psi_{\theta}:G \to \Bij(X)$ given by $g \mapsto \theta(g,\ast) = \varphi_{g}(\ast)$ is a group homomorphism. Here, $\Bij(X)$ denotes the set (group) of bijections from $X$ to itself.
        \item Let $\psi:G \to \Bij(X)$ be a group homomorphism. Also let $\theta_{\psi}:G \times X \to X$ be a map defined as $\theta_{\psi}(g,x) = \psi(g)(x)$. Then $\theta$ is a $G$-action on $X$.
    \end{itemize}
\end{remark}
One can show that $\theta_{\psi_{\theta}} = \theta$ and $\psi_{\theta_{\psi}} = \psi$; if one starts with the action, the homomorphism can be obtained and vice versa.
\begin{align}
    \theta_{\psi_{\theta}}(g,x) &= \psi_{\theta}(g)(x) = \theta(g,x) \text{ for all } g \in G, x \in X,\\
    \psi_{\theta_{\psi}}(g)(x) &= \theta_{\psi}(g,x) = \psi(g)(x) \text{ for all } g \in G, x \in X.
\end{align}

\begin{example}
    \begin{itemize}
        \item Let $H \sbg S_{n}$. Then $H$ acts on $\{1,2,\ldots,n\}$ as $(\sigma,i) \mapsto \sigma(i)$ for all $\sigma \in H$ and $i \in \{1,2,\ldots,n\}$.
        \item Naturally, $GL_{n}(\R) \times \R^{n} \to \R^{n}$ acts as $(A,x) \mapsto Ax$ for all $A \in GL_{n}(\R)$ and $x \in \R^{n}$.
        \item More generally, let $f:G \to GL_{n}(\R)$ be a homomorphism of groups. Then the induced action of $G$ on $\R^{n}$ is given by $(g,x) \mapsto f(g)(x)$ for all $g \in G$ and $x \in \R^{n}$.
        \item Let $G$ be a group and $H \sbg G$. Also let $X = G/H = \{gH \mid g \in G\}$ be the set of left cosets of $H$ in $G$. Then $G$ acts on $X$ as $(g,g'H) \mapsto gg'H$ for all $g,g' \in G$. 
    \end{itemize}
\end{example}

\section{Orbits and Stabilizers}

\begin{definition}
    Let $G$ be a group and $X$ a $G$-set. Then, for $x \in X$, the $G$-orbit (or \eax{orbit}) of $x$ is defined as
    \begin{align}
        \cO(x) = Gx \defeq \{gx \mid g \in G\} = \{\theta(g,x) \mid g \in G\} \subseteq X.
    \end{align}
    If $Gx = X$ for some $x \in X$, then the $G$-action on $X$ is said to be a \eax{transitive action}.
\end{definition}

For example, if $H \sbg G$, then the $G$-action on $G/H$ is transitive.

\begin{proposition}
    Let $G$ be a group acting on a set $X$. For $x,y \in X$, we say $x \sim y$ if there exists $g \in G$ such that $gx = y$. Then $\sim$ is an equivalence relation on $X$, and the equivalence classes are simply the orbits;
    \begin{align}
        [x] = \{y \in X \mid y \sim x\} = Gx.
    \end{align}
\end{proposition}
\begin{proof}
    $g \sim g$ since $eg = g$. If $x \sim y$, then $gx = y$ for some $g \in G$ implies $x = g^{-1}y$ or $y \sim x$. If $x \sim y$ and $y \sim z$, then $gx = y$ and $hy = z$ for some $g,h \in G$, so $(hg)x = z$ showing $x \sim z$.
\end{proof}
These orbits of $G$-action on $X$ give a partition of $X$.

\begin{definition}
    Let a group $G$ act on a set $X$. For $x \in X$, the \eax{stabilizer} of $x$ in $G$ is defined as
    \begin{align}
        \Stab_{G}(x) = \Stab(x) \defeq \{g \in G \mid gx = x\} = \{g \in G \mid \theta(g,x) = x\} \subseteq G.
    \end{align}
\end{definition}

\begin{proposition}
    $\Stab_{G}(x) \sbg G$ for all $x \in X$.
\end{proposition}
\begin{proof}
    Clearly, $e_{G} \in \Stab_{G}(x)$ since $ex = x$. If $g,h \in \Stab_{G}(x)$, then $(gh)x = g(hx) = gx = x$, so $gh \in \Stab_{G}(x)$. Note that if $g \in \Stab_{G}(x)$, then $gx = x$ implies $x = g^{-1}x$, or $g^{-1} \in \Stab_{G}(x)$. $\Stab_{G}(x)$ is therefore a subgroup.
\end{proof}

\begin{example}
    \begin{itemize}
        \item If $G$ acts on $G/H$, then $\Stab(H) = H$ and $\Stab(gH) = \{x \in G \mid xgH = gH\} = gHg^{-1}$.
        \item Regarding the group action $S_{n}$ on $\{1,2,\ldots,n\}$, $\Stab(n) = \{ \sigma \in S_{n} \mid \sigma(n) = n\} = S_{n-1}$.
    \end{itemize}
\end{example}

\begin{proposition}
    We state two results.
    \begin{enumerate}
        \item Let a group $G$ act on a set $X$. For $x \in X$, the \textit{set} $G/\Stab(x)$ is in bijection with $Gx$, the orbit of $x$.
        \item Let $x,y \in X$ be in the same orbit. Then $\Stab_{G}(x) = g\Stab_{G}(y)g^{-1}$, where $g \in G$ is such that $x = gy$.
    \end{enumerate}
\end{proposition}

\begin{proof}
    \begin{enumerate}
        \item Let $H = \Stab(x)$. Let $gH \in G/H$. We show that the map $\psi:G/H \to Gx$ defined as $gH \mapsto gx$ is the required bijection. $\psi$ is well=defined since if $gH = g'H$ for some $g,g' \in G$, then $g^{-1}g'x = x$ implies $g^{-1}g' \in H = \Stab(x)$, so $g'x = gx$. The converse argument is also true, showing $\psi$ is well-defined and one-one. Also, for $y \in Gx$, there exists $g \in G$ such that $y = gx = \psi(gH)$; $\psi$ is also onto.

        \item Let $x,y$ be in the same orbit; that is, there exists $g \in G$ such that $x = gy$. Suppose $h \in \Stab(x)$, or $hx = x$. This implies $hgy = gy \Rightarrow g^{-1}hgy = y$, so $g^{-1}hg \in \Stab(y)$, showing $h \in g\Stab(y)g^{-1}$. The converse argument also holds.
    \end{enumerate}
\end{proof}


\begin{corollary}[The \eax{orbit-stabilizer theorem}]
    $\abs{G/\Stab(x)} = \abs{Gx}$, that is, $\abs{G} = \abs{\Stab(x)} \cdot \abs{Gx}$.
\end{corollary}


\begin{definition}
    Let $G$ be a group acting on a set $X$. The action is termed a \eax{faithful action} if $\{g \in G \mid gx = x \text{ for all } x \in X\} = \{e_{G}\}$, that is, the only element of $G$ that fixes every point in $X$ is the identity element.
\end{definition}
The above definition is equivalent to saying that the induced group homomorphism $G \to \Bij(X)$ is injective.

\begin{definition}
    The \eax{kernel of an action} $G$ acting on $X$ is defined as
    \begin{align}
        K = \{g \in G \mid gx = x \text{ for all } x \in X\}.
    \end{align}
    Note that if the action is $\theta$, then $K = \ker \psi_{\theta}$.
\end{definition}

\noindent One may show that
\begin{align}
    K = \bigcap_{x \in X} \Stab_{G}(x).
\end{align}
For example, the kernel of $G$-action on $G/H$ is $K = \bigcap _{g \in G} gHg^{-1} \nsbg G$. Note that, here in this example, $K$ is the largest subgroup of $G$ contained in $H$.\\ \\
\textit{August 21st.}

\begin{proposition}
    Let $G$ be a finite group. Let $H \sbg G$ be a subgroup such that $[G:H]$ is the smallest prime dividing $\abs{G}$. Then $H$ is a normal subgroup of $G$.
\end{proposition}

\begin{proof}
    Suppose $G$ acts on $G/H$ via $(g,xH) \mapsto gxH$. Letting $p = [G:H]$, this induces a group homomorphism $\varphi:G \to \Bij(G/H) \cong S_{p}$. Also $K = \ker \varphi \subseteq H$ with $K \nsbg G$. By the first isomorphism theorem, we work as
    \begin{align}
        G/K \cong \Img \varphi \Rightarrow \abs{\Img \varphi} = \abs{G/K} = \frac{\abs{G}}{\abs{K}} = \frac{\abs{G}}{\abs{H}} \frac{\abs{H}}{\abs{K}} = [G:H][H:K].
    \end{align}
    But $\abs{\Img \varphi}$ divides $\abs{S_{p}} = p!$, so $p[H:K] \mid p! \implies [H:K] \mid (p-1)!$. Since $p$ is the smallest prime dividing $\abs{G}$, we must have $[H:K] = 1$, showing $K = H$.
\end{proof}

One can show a different proof for a proposition we studied earlier.
\begin{proposition}
    Let $\sigma \in S_{n}$. Then $\sigma$ is a product of disjoint cycles.
\end{proposition}

\begin{proof}
    Let $H = \ip{\sigma} \subseteq S_{n}$. Then $H$ acts on $\{1,2,\ldots,n\}$. Let $O_{1},\ldots,O_{r}$ be the orbits of $H$-action on $\{1,2\ldots,n\}$. We claim that $O_{i} = \{x_{i},\sigma x_{i},\sigma^{2}x_{i},\ldots,\sigma^{d_{i}-1}x_{i}\}$, where $\abs{O_{i}} = d_{i}$ for $1 \leq i \leq r$. If it so happens, then
    \begin{align}
        \sigma = (x_{1}\;\sigma x_{1}\;\cdots\;\sigma^{d_{1}-1}x_{1}) (x_{2}\;\sigma x_{2}\;\cdots\;\sigma^{d_{2}-1}x_{2}) \cdots (x_{r}\;\sigma x_{r}\;\cdots\;\sigma^{d_{r}-1}x_{r}).
    \end{align}
    Note that $\{x_{i},\sigma x_{i},\ldots,\sigma^{d_{i}-1}x_{i}\} \subseteq O_{i}$; if $\sigma^{a}x_{i} \neq \sigma^{b}x_{i}$ for $a \neq b$, then the elements are distinct. Otherwise, $\sigma^{a}x_{i} = \sigma^{b}x_{i}$ implies $\sigma^{b-a}x_{i} = x_{i}$. Then $O_{i} = \{x_{i},\sigma x_{i},\ldots,\sigma^{b-a-1}x_{i}\}$ which contradicts $\abs{O_{i}} = d_{i}$ and $\sigma^{d_{i}}x_{i} = x_{i}$. The disjoint cycle decomposition then follows since $\bigcup_{i=1}^{n} O_{i} = \{1,2,\ldots,n\}$ and $O_{i} \cap O_{j} = \emptyset$ for $i \neq j$.
\end{proof}

\section{Conjugation}

The \eax{conjugation action} of $G$ on $\cP(G)$, the power set of $G$, is defined as $G \times \cP(G) \to \cP(G)$ given by $(g,S) \mapsto gSg^{-1} = \{gsg^{-1} \mid s \in S\}$. This is, indeed, an action since
\begin{align}
    (g_{1}g_{2}) \cdot S = (g_{1}g_{2})S(g_{1}g_{2})^{-1} = g_{1}(g_{2}Sg_{2}^{-1})g_{1}^{-1} = g_{1} \cdot (g_{2}Sg_{2}^{-1}) = g_{1} \cdot (g_{2} \cdot S) \text{ and } e_{G} \cdot S = S.
\end{align}
This action gives one an action on $G$ on itself as
\begin{align}
    G \times G \to G, \quad (g,x) \mapsto gxg^{-1}.
\end{align}

Orbits of this action are termed \eax{conjugacy classes}. For example, the conjugacy class of $e_{G}$ is $\{e_{G}\}$. In general, the conjugacy class of $x \in G$ is $\{x\}$ if and only if $x \in Z(G)$, that is, it commutes with every element.

\begin{example}
    In $S_{3}$, the conjugacy classes are as follows:
    \begin{itemize}
        \item The class of the identity: $\{e_{S_{3}}\}$.
        \item The class of the transpositions: $\{(1\;2),(2\;3),(1\;3)\}$.
        \item The class of the 3-cycles: $\{(1\;2\;3),(1\;3\;2)\}$.
    \end{itemize}
    Note that elements belonging to the same conjugacy class have the same order; this is always true for conjugacy classes of any group. A necessary, but not sufficient, condition for a subset to be a conjugacy class is that the cardinality of the subset must divide the order of the group by the orbit-stabilizer theorem.
\end{example}

\begin{proposition}
    Two permutations $\sigma,\tau \in S_{n}$ belong to the same conjugacy class if their disjoint cycle decompositions are of the same type; that is, they have the same number of cycles and they have the same cycle lengths.
\end{proposition}

\begin{proof}
    We may assume $\sigma = (1\;\cdots\;d_{1})(d_{1}+1\;\cdots\;d_{1}+d_{2})\cdots(d_{1}+\cdots+d_{r-1}+1\;\cdots\;n)$ and $\tau = (i_{1}\;\cdots\;i_{d_{1}})(i_{d_{1}+1}\;\cdots\;i_{d_{1}+d_{2}})\cdots(i_{d_{1}+\cdots+d_{r-1}+1}\;\cdots\;n)$, where $1 \leq i_{1},\ldots,i_{n}$ are distinct, and the cycle lengths in $\sigma$ are $d_{1},d_{2},\ldots,d_{r}$ such that $d_{1}+d_{2}+\cdots+d_{r} = n$. Moreover, we may assume that $d_{1} \leq d_{2} \leq \cdots \leq d_{r}$.

    Let $g \in S$ such that $g(j) = i_{j}$ for all $1 \leq j \leq n$. Then
    \begin{align}
        g\sigma g^{-1}(i_{j}) = g(\sigma(j)) = \begin{cases}
            g(j+1) &\text{ if } j \notin \{d_{1},d_{1}+d_{2},\ldots,d_{1}+\cdots+d_{r-1},n\},\\
            g(d_{1}+\cdots+d_{l-1}+1) &\text{ if } j = d_{1}+\cdots+d_{l} \text{ for } l = 1,2,\ldots,r.
        \end{cases}
    \end{align}
    The cases are essentially $i_{j+1}$ and $i_{d_{1}+\cdots+d_{l-1}+1}$. But this is $\tau(i_{j})$ for every $i_{j}$, showing $g\sigma g^{-1} = \tau$.
\end{proof}

Let $G$ be a finite group. Then
\begin{align}
    \abs{G} = \abs{Z(G)} + \sum_{\substack{[g] \text{ non-trivial}\\ \text{conjugacy classes}}} [G:C_{G}(g)].
\end{align}
This is known as the \eax{class-equation}.

\begin{proof}
    $G$ acts on $G$ via conjugacy. If $x \in Z(G)$, then the conjugacy class of $x$, $[x] = \{x\}$. If $x$ is not in the centre, then $[x]$ is non-trivial. Moreover, by the orbit-stabilizer theorem, $\abs{[x]} = [G:\Stab(x)]$. But $\Stab(x) = \{g \in G \mid gxg^{-1} = x\} = C_{G}(x)$, the centralizer of $x$ in $G$. Thus, $\abs{[x]} = [G:C_{G}(x)]$. The class equation follows by summing over all non-trivial conjugacy classes and adding the size of the centre.
\end{proof}

The above result seem very trivial and straightforward, but proves to be useful in various contexts.

\begin{remark}
        \item Let $H \sbg G$; the normalizer is then $N_{G}(H) = \{g \in G \mid gHg^{-1} = H\}$. If $G$ acts on $\cP(G)$ via conjugation, then $N_{G}(H) = \Stab(H)$ with respect to this action.
\end{remark}

Recall that $G$ acts on $G$ trivially via $(g,x) \mapsto gx$. This is then a transitive action; moreover, it is also a faithful action. This leads to a very important theorem in group theory.

\begin{theorem}[\eax{Cayley's theorem}]
    Every finite group is isomorphic to a subgroup of $S_{n}$ for some $n$.
\end{theorem}

\begin{proof}
    The above $G$-action on itself is faithful; hence, the group homomorphism $\psi:G \to \Bij(G) \cong S_{\abs{G}}$ is injective, and $G \cong \Img \psi \cong H \sbg S_{\abs{G}}$.
\end{proof}

\begin{definition}
    Let $p$ be a prime. A finite group $G$ is called a \eax{$p$-group} if $\abs{G} = p^{n}$ for some $n \geq 0$.
\end{definition}
Of course, for $n = 1$, $G$ is isomorphic to $\Z/p\Z$. For $\abs{G} = p^{2}$, $G$ is also abelian.
\begin{proposition}
    If $G$ is a $p$-group then $Z(G)$ is non-trivial.
\end{proposition}
\begin{proof}
    This follows immediately from the class-equation since if $Z(G)$ is trivial, then $p$ divides $\abs{G}$ and the summation, but $p \nmid \abs{Z(G)} = 1$, contradicting the class-equation.
\end{proof}
Thus, $Z(G)$ contains at least $p$ elements for a $p$-group $G$. One can show that if $G/Z(G)$ is cyclic then $G$ is abelian.

\textit{August 26th.}

\begin{proposition}
    Let $G$ be a group and $N \nsbg G$. Then there is a natural bijection between subgroups of $G/N$ and subgroups of $G$ containing $N$.
\end{proposition}
\begin{proof}
    Define a map $q:G \to G/N$. Let $H \sbg G$ such that $N \subseteq H$ and $q(H) \sbg G/N$; subgroups of $G$ containing $N$ are mapped to subgroups of $G/N$. Let the image be $\bar{H} \sbg G/N$. Then $q^{-1}(\bar{H}) = \{h \in G \mid q(h) \in \bar{H}\}$ is a subgroup of $G$ containing $q^{-1}(e_{G/N}) = N$. Let $g, g' \in q^{-1}(\bar{H})$, that is, $q(g), q(g') \in \bar{H}$. Then $q(gg') = q(g)q(g') \in \bar{H}$, so $gg' \in q^{-1}(\bar{H})$. Also, if $g \in q^{-1}(\bar{H})$, then $q(g) \in \bar{H}$ implies $q(g^{-1}) = q(g)^{-1} \in \bar{H}$, so $g^{-1} \in q^{-1}(\bar{H})$. Thus, $q^{-1}(\bar{H})$ is a subgroup of $G$ containing $N$. The maps $H \mapsto q(H)$ (denoted by $\phi$) and $\bar{H} \mapsto q^{-1}(\bar{H})$ (denoted by $\psi$) are inverses of each other. To show this, let $N \subseteq H \sbg G$. Then $\phi(H) = q(H)$, and $H \subseteq q^{-1}(q(H)) = \psi(q(H))$. 

    Conversely, let $\bar{H} \sbg G/N$. Then $\psi(\bar{H}) = q^{-1}(\bar{H})$ is a subgroup of $G$ containing $N$. Applying $\phi$, we have $\phi(\psi(\bar{H})) = q(q^{-1}(\bar{H})) = \bar{H}$. Similarly, for $H \sbg G$ containing $N$, $\psi(\phi(H)) = q^{-1}(q(H)) = H$. Thus, $\phi$ and $\psi$ are inverses, establishing the bijection.
\end{proof}

\begin{theorem}[\eax{Cauchy's theorem}]
    Let $p \mid \abs{G}$ for a prime $p$. Then there exists $x \in G$ such that $\abs{x}=p$.
\end{theorem}
\begin{proof}
    We perform induction on $\abs{G}$. If $G$ is abelian, pick a non-identity $x \in G$. If $\abs{x}$ is $mp$ for some $m$, then $x^{m}$ has order $p$. If $\abs{x}$ is not a multiple of $p$, then $H = \ip{x} \nsbg G$ and $p \mid \abs{G/H}$. By induction hypothesis, there exists $yH \in G/H$ such that $\abs{yH} = p$. Thus, $(yH)^{p} = e_{G/H} = H$ or $y^{p}H = H$; $y^{p}$ must belong in $H$. This tells us that $y^{p} = x^{r}$ for some $r$, and $m = \abs{x^{r}}$ is coprime to $p$ as $(\abs{x},p) = 1$. Thus, $(y^{p})^{m} = e_{G}$. We claim that $p \mid \abs{y}$. If not, then $y^{r} = e$ for some $p \nmid r$ and $(yH)^{r} = H$. But $\abs{yH} = p$, a contradiction. Hence, $\abs{y} = pn$ for some $n$, and $\abs{y^{n}} = p$.

    For the general case, let $Z(G) \nsbg G$ be the centre of $G$. If $\abs{(Z(G))} \neq 1$, then consider two cases; if $p \mid Z(G)$, then there exists $x \in Z(G) \subseteq G$ such that $\abs{x} = p$. If $p \nmid \abs{Z(G)}$, then $p \mid \abs{G/Z(G)}$ and there exists $yZ(G) \in G/Z(G)$ such that $\abs{yZ(G)} = p$ by induction hypothesis. The same argument as before works leading to the claim that $\abs{y}$ is a multiple of $p$ where $y$ is such that $q(y) = yZ(G)$, and that $\abs{y^{n}} = p$ for some $n$.

    If $Z(G) = \{e\}$, the class equation gives $\abs{G} = \abs{Z(G)} + \sum_{[x]:[G:C_G(x)]\neq 1} [G:C_G(x)]$. SO there exists $x \in G$ such that $p \nmid [G:C_{G}(x)] > 1$ implying that $p \mid \abs{C_{G}(x)} < \abs{G}$ since $\abs{G} = \abs{C_{G}(x)}[G:C_{G}(x)]$. By induction hypothesis there exists $y \in C_{G}(x) \sbg G$ such that $\abs{y} = p$
\end{proof}

\section{Sylow's Theorems}

\begin{definition}
    Let $G$ be a finite group and $p$ be a prime. Let $\abs{G} = p^{n}m$ where $(m,p) = 1$ and $n \geq 0$. A subgroup $H \sbg G$ is called a \eax{$p$-Sylow subgroup} of $G$ if $\abs{H}=p^{n}$.
\end{definition}

In fact, such a subgroup always exists.

\begin{theorem}[\eax{Sylow's first theorem}]
    Let $G$ be a finite group and $p$ be a prime. Then $G$ has a $p$-Sylow subgroup.
\end{theorem}

\begin{proof}
    We work similar to the proof of Cauchy's theorem; perform induction on $\abs{G}$. We may assume $n \geq 1$. If $p \nmid Z(G)$, the class equation tells us 
    \begin{align}
        \abs{G} = \abs{Z(G)} + \sum_{\substack{[x] \text{ non-trivial}\\ \text{conjugacy classes}}} [G:C_G(x)].
    \end{align}
    So there must exists $x \in G$ such that $p \nmid [G:C_G(x)] > 1$ implying that $\abs{C_{G}(x)} < \abs{G}$. Also, $\abs{C_{G}(x)} = p^{n}k$ for some $k$. By the induction hypothesis, there exists $P \sbg C_{G}(x)$ such that $\abs{P} = p^{n}$ and $P \sbg G$.

    If $P \mid \abs{Z(G)}$, then there exists $H \sbg Z(G)$ such that $\abs{H} = p$. Then $H \nsbg G$ since $H \subseteq Z(G)$, giving us $\abs{G/H} = p^{n-1}m$. So by the induction hypothesis there exists subgroup $\overline{P} \sbg G/H$ such that $\abs{\overline{P}} = p^{n-1}$. Let $q:G \to G/H$ be the quotient map. Let $P = q^{-1}(\overline{P})$. then $\abs{P} = \abs{P/H}\abs{H} = p^{n-1} \cdot p = p^{n}$. Thus, $P \sbg G$ and $\abs{P} = p^{n}$.
\end{proof}


\begin{remark}
    \begin{itemize}
        \item Note that $S_{n}$ acts on $\{1,2,\ldots,n\}$ by permuting the elements. Any subgroup $H \sbg S_{n}$ is termed a \eax{transitive subgroup} if the action of $H$ on $\{1,2,\ldots,n\}$ is transitive. For example, $\ip{(1\;2\;3)} \sbg S_{3}$ is transitive. $H$ is termed a \eax{2-transitive subgroup} if given any $1 \leq i,j \leq n$ with $i \neq j$, there exists a permutation $h \in H$ such that $h(1) = i$ and $h(2) = j$. Since there does not exists $\sigma \in \ip{(1\;2\;3)}$ such that $\sigma(1) = 1$ and $\sigma(2) = 3$, the subgroup is not 2-transitive.

        One can show that $H \sbg S_{n}$ is 2-transitive if and only if $\Stab_{H}(i)$ acts transitively on $\{1,2,\ldots,n\} \setminus \{i\}$ for all $i \in \{1,2,\ldots,n\}$.


        \item Suppose $H$ acts on $A \defeq \{1,2,\ldots,n\}$. A subset $B \subseteq A$ is called a \eax{block} of for all $\sigma \in H$, we have either $\sigma(B) = B$ or $\sigma(B) \cap B = \emptyset$. Note that $\{i\}$ is a block for all $i \in A$. $A$ is also a block. $H$ is termed a \eax{primitive subgroup} if $\{i\}$ and $A$ are the only blocks. One can show that 2-transitive subgroups are primitive.
    \end{itemize}
\end{remark}

\textit{August 28th.}

\begin{theorem}[\eax{Sylow's theorem}]
    Let $G$ be a finite group and $P$ be a $p$-Sylow subgroup of $G$ for a prime $p$.
    \begin{itemize}
        \item Let $H$ be a $p$-subgroup of $G$. Then there exists $g \in G$ such that $H \sbg g^{-1}Pg$.
        \item All $p$-Sylow subgroups of $G$ are conjugate to each other.
        \item Let $n_{p}$ be the number of $p$-Sylow subgroups of $G$. Then $n_{p} \equiv 1 \mod p$ and $n_{p} \mid [G:P]$.
    \end{itemize}
\end{theorem}

\begin{proof}
    Let $S = \{g^{-1}Pg \mid g \in G \}$. Then $G$ acts on $S$ via conjugation. $H$, being a subgroup of $G$, also acts on $S$. Let $Q \in S$. Then $\Stab_{H}(Q) \sbg H$; if this stabilizer is equal to $H$, then $H \sbg N_{G}(Q)$. Also $Q \sbg N_{G}(Q)$. Noting $GQ = S$, the orbit stabilizer theorem gives us
    \begin{align}
        \abs{S}\abs{\Stab_{G}(Q)} = \abs{G} \implies \abs{S} = [G:N_{G}(Q)] \implies \abs{S} \mid [G:Q] \;([G:Q]=[G:P] = m).
    \end{align}
    Suppose $\Stab_{H}(Q)$ were a proper subgroup of $H$ for all $Q \in S$. Then $p$ divides the order of $\{hQh^{-1}|h \in H\}$ since $H$ is a $p$-subgroup. Since $\abs{S}$ is co-prime to $p$, the above is not possible; there exists $P_{0} \in S$ such that $\Stab_{H}(P_{0}) = H$. This shows $H \sbg N_{G}(P_{0})$ and $P_{0} \sbg N_{G}(P_{0})$ ($P_{0}$ is a normal subgroup, in fact). Thus, $HP_{0} \sbg N_{G}(P_{0}) \sbg G$. $[HP_{0}:P_{0}]$ is co-prime to $p$, and $\abs{H/(H \cap P_{0})}$ is a power of $p$, and $HP_{0}/P_{0} \cong H/(H \cap P_{0})$ leads us to conclude that $[HP_{0}:P_{0}] = 1$. Thus, $H \subseteq P_{0}$, or $H \sbg g^{-1}Pg$ for some $g \in G$. Second part follows from the first part by taking $H$ to be a $p$-Sylow subgroup. In particular, $S$ turns out to be the set of all $p$-Sylow subgroups.

    $P \sbg G$ acts on $S$ via conjugation. $\{P\}$ is one of the orbits since $gPg^{-1} = P$ for all $g \in P$. Let $Q \in S$ and $Q \neq P$. If $gQg^{-1} = Q$ for all $g \in P$, then $P \sbg N_{G}(Q)$ and $PQ \sbg N_{G}(Q) \sbg G$. $[PQ:Q]$ is co-prime to $p$ and $PQ/Q \cong P/(P \cap Q)$. Since $P \neq Q$, $\abs{P/P \cap Q} \neq 1$ which implies that $gQg^{-1} \neq Q$ for all $g \in P$ or $\Stab_{P}(Q)$ is a proper subgroup of $P$. Thus $\abs{PQ} = \abs{\{gQg^{-1} \mid q \in P\}}$ is a multiple of $p$. So 1 $P$-orbit is $\{P\}$ and $P$-orbits are of size multiple of $p$; thus the class equation gives us $n_{p} = \# S = \sum \abs{\text{orbits}} \equiv 1 \mod p$.
\end{proof}

\subsection{Simple Groups}
\begin{definition}
    A group $G$ is called a \eax{simple group} if the only normal subgroups are $G$ and $\{e_{G}\}$.
\end{definition}

An example is $\Z/p\Z$ for a prime $p$. One can show that $A_{n}$ is simple for $n \geq 5$.

\begin{corollary}
    Let $G$ be a finite group and $p$ a prime. If $n_{p}$, the number of $p$-Sylow subgroups of $G$, is unity, then $P \nsbg G$ where $P$ is a $p$-Sylow subgroup.
\end{corollary}
$p$-groups different from $\Z/p\Z$ are not simple, since their center is non-trivial (and proper).

\begin{example}
    Let $G$ be a group of order 12. We show that $G$ is not simple. For $p = 3$, the conditions of $n_{p} \equiv 1 \mod p$ and $n_{p} \mid [G:P]$ tell us that $n_{3}$ is $1$ or $4$. If $n_{3} = 1$, then $P \nsbg G$ where $P$ is a $3$-Sylow subgroup. If $n_{3} = 4$, then $P$ is a $3$-group and the number of elements of order $3$ in $P$ is $2$. So there are $8$ elements of order $3$. Then the $2$-Sylow subgroup is of order $4$ containing the remaining $4$ elements. Calling this group $Q$, we have $Q \nsbg G$.
\end{example}


\begin{proposition}
    Let $G$ be a group of order $pq$ or $p^{2}q$, where $p$ and $q$ are distinct primes. Then $G$ is not simple.
\end{proposition}
\begin{proof}
    We may assume $p < q$. If $\abs{G} = pq$, then $n_{q} \equiv 1 \mod q$ and $n_{q} \mid p$ implies $n_{q} = 1$. Thus, $Q \nsbg G$ where $Q$ is a $q$-Sylow subgroup. For $\abs{G} = p^{2}q$, if $q < p$, the above can be used. If $p < q$, then $n_{q}$ is one of $1,p,p^{2}$. If $n_{q} = 1$, we are done. $n_{q}$ cannot be $p$ since $n_{q} \equiv 1 \mod q$. The third case, where $n_{q} = p^{2}$, tells us that $q \mid (p^{2}-1)$ or $q \mid (p-1)(p+1)$. Since $q > p$, $q \mid (p+1)$ or $q = p+1$. The only such case is when $q = 3$ and $p = 2$. It follows from the previous example that a group $G$ of order $p^{2}q = 12$ cannot be simple.
\end{proof}

\begin{proposition}
    Let $G$ be a group of order $pqr$, where $p$, $q$, and $r$ are primes. Then $G$ is not simple.
\end{proposition}
\begin{proof}
    We may assume they are distinct; if not, the above proposition works. So we may take $p < q < r$. For $G$ to be simple, $n_{p},n_{q},n_{r}$ must be different from $1$. The possibilities, thus, for $n_{r}$ are $p$, $q$, or $pq$. But $n_{r} \equiv 1 \mod r$ implies that $n_{r} = pq$. If $R$ and $R'$ are $r$-Sylow subgroups, then $R = R'$ or $R \cap R' = \{e\}$. Thus the number of elements of order $r$ in $G$ is at least $pq(r-1)$. Also, $n_{q}$ is either $r$ or $pr$ since $p < q$. By the same argument, the number of elements of order $q$ in $G$ is at least $(q-1)r$. Similarly, the number of elements of order $p$ in $G$ is at least $(p-1)q$. Summing up (and a 1 for the identity), we get
    \begin{align}
        pqr - pq + qr - r + pq - q + 1 = pqr + qr - r - q + 1 > pqr
    \end{align}
    which is a contradiction to the fact that $\abs{G} = pqr$. Hence, (at least) one of $n_{p}$,$n_{q}$, or $n_{r}$ must be $1$, showing $G$ is not simple.
\end{proof}

\begin{example}
    For $\abs{G} = 24$, $G$ is not simple. Let $P$ be a $2$-Sylow subgroup of $G$. Then $P$ has order $8$ and $n_{2}$ is either $1$ or $3$. If $n_{2} = 1$, we are done, so let us take $n_{2} = 3$ and $n_{3} = 4$. $8$ elements have order $3$. Let $P_{1}$, $P_{2}$, and $P_{3}$ be distinct. Then $\abs{P_{1} \cap P_{2}} \leq 4$ and $\abs{P_{1} \cup P_{2}} \geq 12$. $G$ acts trasitively on $A = \{P_{1},P_{2},P_{3}\}$ via conjugation. This gives a group homomorphism via $\phi:G \to S_{3}$ and $\Img \phi \neq \{e\}$. Thus $\ker \phi$ is not the entirety of $G$ and it's non-trivial. So $G$ is not simple.
\end{example}
\noindent \textit{September 2nd.}

Recall that the alternating groups $A_{n}$ are simple for $n \geq 5$; this is a group of order $n!/2$, or $60$ and above. One can show that any group of order less than $60$ is simple if and only if it is isomorphic to $\Z/p\Z$ for some prime $p$. If the order of $G$ is $pqr$, $pq$, or $p^{n}$, then $G$ is not simple. We use this to show that the previous statement.

If $\abs{G} = 2^{3} \cdot 3 = 24$, then we have already shown $G$ is not simple. If $\abs{G} = 2^{2} \cdot 3^{2} = 36$, then $n_{3}$ is $1$ or $4$ and $n_{1}$ is $1$, $3$, or $9$. For simplicity, assume both are non-unity. If $n_{3} = 4$, then $G$ acts on $3$-Sylow subgroups transitively which produces a non-trivial group homomorphism $\phi:G \to S_{4}$, with $\ker \phi$ non-trivial and a proper subgroup of $G$, since $\abs{S_{4}} = 24$ and $\abs{G} = 36$. Thus, $G$ cannot be simple.

If we then look at $\abs{G} = 2^{3} \cdot 5 = 40$, then $n_{2} = 1,5$ and $n_{5} = 1$. Thus, $G$ is not simple. For $\abs{G} = 2^{4} \cdot 3 = 48$, we have $n_{2} = 1$ or $3$. If $n_{2} = 3$, we proceed as before showing $G$ not simple. $\abs{G} = 50,51,52,54$ also involves casework. For $\abs{G} = 2^{3} \cdot 7 = 56$, $n_{7} = 1$ or $8$. If $n_{7} = 8$, then the number of elements of order $7$ in $G$ is $8 \cdot 6 = 48$. This means that the $2$-Sylow subgroup $H$ is made of these remaining $8$ elements, showing $n_{2} = 1$.

\subsubsection{Alternating Groups}

\begin{proposition}
    Let $N \nsbg G$. Then $N$ is the union of conjugacy classes, that is, if $C$ is a conjugacy class in $G$ then $C \subseteq N$ or $C \cap N = \emptyset$.
\end{proposition}
\begin{proof}
    This is simple to see since $g^{-1}Ng = N$ for all $g \in G$.
\end{proof}

We show that $A_{n}$'s are simple for $n \geq 5$ using the above proposition. For sake of completion, we note that $A_{n}$ is defined to the subgroup of $S_{n}$ containing all even permutations, where a permutation is deemed even if it can be written as the product of an even number of transpositions.

\begin{proposition}
    Let $\sigma \in S_{n}$ be a permutation. If $\sigma = \tau_{1}\tau_{2}\cdots\tau_{r} = \tau_{1}'\tau_{2}'\cdots\tau_{s}'$ are two different expressions of $\sigma$ as a product of transpositions, then $r \equiv s$ modulo $2$.
\end{proposition}
By the above proposition, the sign of a permutation $\sgn(\sigma) = (-1)^{r}$ is well-defined.
\begin{proof}
    Let $\Delta = \Delta(x_{1},\ldots,x_{n}) = \prod_{1 \leq i < j \leq n} (x_{i}-x_{j})$. For $\sigma \in S_{n}$, define
    \begin{align}
        \sigma(\Delta) \defeq \Delta(x_{\sigma(1)},\ldots,x_{\sigma(n)}).
    \end{align}
    Let $\tau = (1\;2)$. Then $\tau(\Delta) = -\Delta$. Similarly, for any transposition $\tau$, we have $\tau(\Delta) = -\Delta$. If $\sigma \in S_{n}$ with $\sigma = \tau_{1}\tau_{2}\cdots\tau_{r} = \tau_{1}'\tau_{2}'\cdots\tau_{s}'$, then we have
    \begin{align}
        (-1)^{r} = \tau_{1}\cdots\tau_{r}(\Delta) = \sigma(\Delta) = \tau_{1}'\cdots\tau_{s}'(\Delta) = (-1)^{s}.
    \end{align}
    Thus, $r \equiv s$ modulo $2$.
\end{proof}

\begin{proposition}
    $A_{n}$ is generated by $3$-cycles.
\end{proposition}
\begin{proof}
    If $\sigma \in A_{n}$, then $\sigma = \tau_{1}\tau_{2}\cdots\tau_{2r-1}\tau_{2r}$ for some $r$, where each $\tau_{i}$ is a transposition. If $\tau_{1} = (1\;2)$ and $\tau_{2} = (1\;3)$, then $\tau_{1}\tau_{2} = (1\;3\;2)$. If $\tau_{1} = (1\;2)$ and $\tau_{2} = (3\;4)$, then $\tau_{1}\tau_{2} = (1\;2)(3\;4) = ((1\;2)(2\;3))((2\;3)(3\;4))$, where each term is a $3$-cycle. Thus, $\sigma$ is a product of $3$-cycles.
\end{proof}

\begin{proposition}
    For $n \geq 5$, $3$-cycles in $A_{n}$ form a conjugacy class in $A_{n}$.
\end{proposition}
\begin{proof}
    Let $\tau \in A_{n}$ be a $3$-cycle. Then there exists a $\sigma \in S_{n}$ such that $\sigma^{-1}(1\;2\;3)\sigma = \tau$. If $\sigma \in A_{n}$, we are done. If not, let $\sigma' = (4\;5)\sigma$. Then $\sigma' \in A_{n}$ and $\sigma'^{-1}(1\;2\;3)\sigma' = \tau$.
\end{proof}

\begin{theorem}
    $A_{5}$ is a simple group.
\end{theorem}
\begin{proof}
    Here, $\abs{A_{5}} = 60$. If $N \nsbg A_{5}$, then it is a union of conjugacy classes. The conjugacy classes in $A_{5}$ are $\{e\}$, the set of $3$-cycles (of size $20$), $[(1\;2)(3\;4)]$ (of size $15$). Via the orbit-stabilizer theorem, 
    \begin{align}
        \abs{\Stab_{S_{5}}((1\;2)(3\;4))} = \frac{5!}{15} = 8
    \end{align}
    where $S_{n}$ acts on itself via conjugation. $A_{n}$ also acts on $S_{n}$ via conjugation, and $\Stab_{A_{5}}((1\;2)(3\;4)) \sbg \Stab_{S_{5}}((1\;2)(3\;4)),\; A_{5}$. Thus the stabilizer in $A_{5}$ can be of order $1$ or $2$ or $4$ or $8$. Since $\abs{A_{5}(1\;2)(3\;4)} \leq 15$, the stabilizer cannot be $1$ or $2$, and since $\abs{A_{5}} = 60$, the stabilizer cannot be $8$. Thus the stabilizer in $A_{5}$ must be of order $4$. Thus, $[(1\;2)(3\;4)]$ is a conjugacy class in $A_{5}$ of size $15$.

    The number of $5$-cycles in $A_{5}$ is $4! = 24$. By a similar arguments, there are $2$ conjugacy classes in $A_{n}$ of size $12$ each consisting of $5$-cycles.

    Now suppose $N \neq \{e\}$. If $N$ contains the $3$-cycles, then $N$ must be $A_{5}$ (since $3$-cycles generate $A_{5}$), so we let $N \cap \text{$3$-cycles} = \emptyset$. Let $(1\;2)(3\;4) \in N$. Then $(1\;2)(3\;5) \in N$ and $(1\;2)(3\;4)(1\;2)(3\;5) = (3\;5\;4) \in N$ implies $N$ contains all $3$-cycles and must be $A_{5}$. So we also let the intersection of $N$ with this conjugacy class be empty. If $N$ contains half of the $5$-cycles, then $\abs{N} = 13$ which is not possible since $\abs{N} \mid 60$. If $N$ contains all $5$-cycles, then $\abs{N} = 25$, which is again not possible. The only real possibility left is that $A_{5}$ must be simple.
\end{proof}

\begin{theorem}
    $A_{n}$ is a simple group for $n \geq 5$.
\end{theorem}
\begin{proof}
    Let $n \geq 6$, and let $\{e\} \neq N \sbg A_{n}$. It is enough to show that $N$ must contain a $3$-cycle. Let $\sigma \in N$ be a non-identity element. Suppose $\sigma = \tau_{1}\cdots\tau_{r}$ where $\tau_{i}$ are disjoint cycles. We deal with cases.

    Case I, where at least two disjoint cycles are transpositions. We can assume that $\tau_{1} = (1\;2)$ and $\tau_{2} = (3\;4)$, and $\sigma = (1\;2)(3\;4)\tau$ and $\tau$ does not contain any elment in $\{1,2,3,4\}$. Then
    \begin{align}
        N \ni (1\;2\;3) \sigma (3\;2\;1) \sigma^{-1} = (1\;2\;3)(1\;2)(3\;4)\tau(3\;2\;1)\tau^{-1}(1\;2)(3\;4) = (1\;3)(2\;4).
    \end{align}
    Conjugating by a $3$-cycle, we get
    \begin{align}
        N \ni (1\;3\;5)(1\;3)(2\;4)(5\;3\;1)(1\;3)(2\;4) = (2\;4)(3\;5)(1\;3)(2\;4) = (1\;3\;5).
    \end{align}
    Thus, $N$ contains a $3$-cycle, and it must be equal to $A_{n}$. Case II, where exactly one of the cycles is a transposition, say, $\tau_{1}$. This implies that one of $\tau_{1},\ldots,\tau_{r}$ has length at least $4$. Thus, $\sigma = (1\;2\;3\;4\;\cdots)\tau = \tau'\tau$ where $\tau$ does not contain any element from $\tau'$. Thus,
    \begin{align}
        N \ni (1\;2\;3)(1\;2\;3\;4\;\cdots)\tau(3\;2\;1)\tau^{-1}(1\;2\;3\;4\;\cdots)^{-1} = (1\;2\;3)(2\;4\;3) = (1\;2\;4).
    \end{align}
    $N$, again, contains a $3$-cycle implying it must be $A_{n}$. Case III, where none of the $\tau_{i}$'s are transpositions. If one of the $\tau_{i}$'s has length at least $4$, then the same argument in Case II works. So we may assume $\sigma$ is a product of disjoint $3$-cycles. If $\sigma$ is a $3$-cycle, we are done. Otherwise, let $\sigma = (1\;2\;3)(4\;5\;6)\tau$ where $\tau$ is disjoint from the previous cycles. Then
    \begin{align}
        N \ni (1\;4\;5)(1\;2\;3)(4\;5\;6)\tau(5\;4\;1)\tau^{-1}(6\;5\;4)(3\;2\;1) = (1\;4\;5)(6\;5\;2) = (1\;4\;5\;2\;6).
    \end{align}
    The argument for Case II can now be applied, since we have a cycle of length more than $4$ in $N$.
\end{proof}